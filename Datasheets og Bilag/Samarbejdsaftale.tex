\documentclass{article}
\usepackage[utf8]{inputenc}
\usepackage[T1]{fontenc}
\usepackage{lmodern} % load a font with all the characters
\usepackage{parskip}
\usepackage[danish]{isodate}
\begin{document}
	{\Large Samarbejdskontrakt for Projektgruppe 3}
	
	\textbf{Gruppe møder}
	\begin{itemize}
		\item Et møde om ugen
		\item Der indkaldes over Facebook og dagsorden bliver lagt på Github
		\item Mødedage, tider og steder planlægges løbende.
	\end{itemize}
	
	\textbf{Afbud til møder}
	\begin{itemize}
		\item Sygdom meldes til gruppen fra morgenstunden.
		\item Planlagt fravær meldes så hurtigt så muligt.
		\item Uanmeldt fravær godtages KUN en gang.
		\item Anden gang der udeblives vælger gruppen en kontaktperson, der tager kontakt til den udeblevne for at få en snak.
		\item Gentagne udeblivelser resulterer i en gruppesession med Gunvor til stede og i værste fald, at vedkommende ikke får sit navn på projektet.
	\end{itemize}
	
	\textbf{Gruppeledelse}
	\begin{itemize}
		\item Gruppen vælger en koordinator (Pernille) som bliver tovholder. Koordinatoren samler trådene og indkalder til møder, skriver dagsorden, referat og har overblikket.
		\item Arbejdsopgaver uddeles på møderne efter de enkeltes præferencer. Der sigtes efter, at alle bliver udfordret og ikke kun holder sig til deres komfortzoner. 
		\item Det enkelte gruppemedlem har ansvaret for sine egne opgaver, og skal give statusrapport ved hvert møde. Den enkelte har også ansvar for at melde til gruppen, hvis der opstår problemer, forsinkelser eller lignende.
		\item Tidsplanen opdateres ugentligt på møderne.
		\item Ved problemer indkalder koordinator til ekstra møde på gruppen, og gruppen bestemmer samlet om Gunvor skal involveres. Dette gælder også ved brud på samarbejdskontrakten, og værste konsekvens er at vedkommende ikke bliver skrevet på projektet.
	\end{itemize}
	
	\textbf{Gruppens ambitionsniveau}
	\begin{itemize}
		\item En god og gennemført rapport.
		\item En fungerende prototype
		\item God læring for alle.
		\item Godt samarbejde.
		
	
	\end{itemize}
\end{document}
