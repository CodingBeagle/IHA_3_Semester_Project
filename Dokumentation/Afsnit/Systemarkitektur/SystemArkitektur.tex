\chapter{Systemarkitektur}

\section{Signalbeskrivelse}
	\begin{tabular}{|>{\hspace{0pt}}p{3cm} | >{\hspace{0pt}}p{3cm} | p{3cm} | p{3cm} |}
		\hline
		\textbf{Blok-navn} & \textbf{Funktionsbeskrivelse} & \textbf{Signaler} & \textbf{Signalbeskrivelse} \\ \hline
		Devkit8000 & Fungerer som grænsflade mellem bruger og systemet. & masterSPI & Type: SPI \\ \cline{3-4}
		 & & touch & Type: touch \newline Tryk på DevKit8000 display \\ \hline
		 PSoC0 & Fungerer som I2C master for systemet samt SPI slave til DevKit8000. & slaveSPI & Type: SPI \\ \cline{3-4}
		 & & masterI2C & Type: I2C\\ \hline
		Motorstyring & Modtager input fra Wii-Nunchuck og omsætter det til PWM signaler. & motorSlave & Type: I2C \newline Indeholder Wii-Nunchuck data der skal bruges til motorstyring  \\ \cline{3-4}
		 & & power & Type: \(V_{CC}\) \newline Strømforsyning til motorstyringen \\ \hline
		PSoC1 & Modtager input fra Wii-Nunchuck og omsætter det til PWM signaler. & MotorI2C & Type: I2C \\ \cline{3-4} 
		& & PWM & Type: PWM \newline PWM signal til styring af moternes hastighed. \\ \hline
		Motor & movings & pwm & Type: PWM \\ \cline{3-4}
		 & & power & Type: CMOS \\ \hline
		PSoC2 & stuff & wiiSlave & Type: I2C \\ \hline
		Wii-nunchuck & wiiiiiii & nunchuckSlave & Type: I2C \\ \hline
		SPI & spiderman & MOSI & Type: SPI \\ \cline{3-4}
		 & & MISO & Type: SPI \\ \cline{3-4}
		 & & SCLK & Type: SPI \\ \cline{3-4}
		 & & SS & Type: SPI \\ \hline
		 I2C & jsjjjas & SDA & Type: I2C \\ \cline{3-4}
		 & & SCL & Type: I2C \\ \hline
		\end{tabular}
\subsection{Specifikation og Analyse}