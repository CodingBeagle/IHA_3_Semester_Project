\documentclass{article}

% -- PREAMBLE START --
\usepackage[utf8]{inputenc}
\usepackage[T1]{fontenc}
\usepackage{lmodern} % load a font with all the characters

\usepackage{parskip}

\usepackage[danish]{isodate}

\usepackage{array}

\begin{document}
	\title{Kravspecifikation}
	\maketitle
	
	\section{Fully dressed use case}
	\begin{tabular}{|>{\hspace{0pt}}p{3cm}  |>{\hspace{0pt}}p{9cm}|}
		\hline
		\textbf{Navn} & Spil Goofy Candygun 3000\\ \hline
		\textbf{Mål} & At spille spillet\\ \hline
		\textbf{Initiering} & Bruger\\ \hline
		\textbf{Aktører} & Bruger\\ \hline
		\textbf{Antal samtidige forekomster} & Ingen \\ \hline
		\textbf{Prækondition} & Spillet og kanonen er operationelt \\ \hline
		\textbf{Postkondition} &  Brugeren har færdiggjort spillet \\ \hline
		\textbf{Hovedscenarie} & \begin{enumerate}
			\item Bruger vælger spiltype på brugergrænsefladen
			\subitem [Extension 1: Brugeren vælger 2 player mode] 
			\item Brugeren vælger antal skud til runden
			\item Brugeren fylder magasin med slik tilsvarende antal skud
			\item Brugeren indstiller kanon med analog stick på Wii-nunchuck
			\item Bruger udløser kanonen med Wii-nunchucks trigger
			\item Brugergrænsefladen opdateres med spillets statistikker
			\item Punkt 4 til 6 gentages indtil skuddene er opbrugt 
			\subitem[Extension 2: Bruger afslutter det igangværende spil]
			\item Brugergrænseflade viser afslutningsinfo for runden
			\item Brugeren afslutter runden
			\item Brugergrænsefladen vender tilbage til starttilstanden
		\end{enumerate}\\ \hline
		\textbf{Udvidelser/ undtagelser} & \textbf{[Extension 1: Brugeren vælger 2 player mode]} \newline \begin{enumerate} 
			\item Punkt 4 til 6 i hovedscenariet gennemføres
			\item Brugeren overdrager Wii-nunchuck til den anden bruger
			\item Punkt 1 til 2 gentages indtil skuddene er opbrugt
			\item Use casen genoptages fra punkt 8
			\end{enumerate}
			\textbf{[Extension 2: Bruger afslutter det igangværende spil]} \newline \begin{enumerate}
			\item Brugergrænsefladen vender tilbage til starttilstanden
			\item Use casen afsluttes
			\end{enumerate}\\ \hline
	\end{tabular}
\end{document}