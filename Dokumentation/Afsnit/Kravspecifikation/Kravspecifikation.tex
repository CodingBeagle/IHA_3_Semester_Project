\renewcommand{\labelenumii}{\theenumii}
\renewcommand{\theenumii}{\theenumi.\arabic{enumii}.}
\chapter{Kravspecifikation}
\section{Aktør kontekst diagram}
\begin{figure}[H]
	\centering
	\includegraphics[]{Kravspecifikation/images/kontekstDiagram}
	\caption{Kontekst diagram for slikkanonen}
	\label{ref:kontekstDiagram}
\end{figure}

\section{Use case diagram}
\begin{figure}[H]
	\centering
	\includegraphics[]{Kravspecifikation/images/usecaseDiagram}
	\caption{Use case diagram for slikkanonen}
	\label{ref:usecaseDiagram}
\end{figure}

\section{Aktør beskrivelse}
I dette system er der en aktør, nemlig brugeren. Brugeren initierer systemet, ved at vælge spiltype på brugergrænsefladen. Derudover har brugeren mulighed for at stoppe spillet igennem brugergrænsefladen. Brugeren vil under spillet interagere med systemet gennem Wii-Nunchucken. 


\section{Fully dressed use case}
\begin{tabular}{|>{\hspace{0pt}}p{3cm}  |>{\hspace{0pt}}p{9cm}|}
	\hline
	\textbf{Navn} & Spil Goofy Candygun 3000\\ \hline
	\textbf{Mål} & At spille spillet\\ \hline
	\textbf{Initiering} & Bruger\\ \hline
	\textbf{Aktører} & Bruger\\ \hline
	\textbf{Antal samtidige forekomster} & Ingen \\ \hline
	\textbf{Prækondition} & Spillet og kanonen er operationel \\ \hline
	\textbf{Postkondition} &  Brugeren har færdiggjort spillet \\ \hline
	\textbf{Hovedscenarie} & \begin{enumerate}
		\item Bruger vælger spiltype på brugergrænsefladen
		\item Brugeren vælger antal skud til runden
		\item Brugeren fylder magasin med slik tilsvarende antal skud
		\item Brugeren indstiller kanon med analog stick på Wii-nunchuck
		\item Bruger udløser kanonen med Wii-nunchucks trigger
		\item Systemet lader et nyt skud
		\item Brugergrænsefladen opdateres med spillets statistikker
		\item Punkt 4 til 6 gentages indtil skuddene er opbrugt 
		\subitem [Extension 1: Brugeren vælger 2 player mode] 
		\subitem[Extension 2: Bruger afslutter det igangværende spil]
		\item Brugergrænseflade viser afslutningsinfo for runden
		\item Brugeren afslutter runden
		\item Brugergrænsefladen vender tilbage til starttilstanden
	\end{enumerate}\\ \hline
	\textbf{Udvidelser/ undtagelser} & \textbf{[Extension 1: Brugeren vælger 2 player mode]} \newline \begin{enumerate} 
		\item Brugeren overdrager Wii-nunchuck til den anden bruger
		\item Punkt 1 til 2 gentages indtil skuddene er opbrugt
		\item Use casen genoptages fra punkt 8
		\end{enumerate}
		\textbf{[Extension 2: Bruger afslutter det igangværende spil]} \newline \begin{enumerate}
		\item Brugergrænsefladen vender tilbage til starttilstanden
		\item Use casen afsluttes
		\end{enumerate}\\ \hline
\end{tabular}

\section{Ikke funktionelle krav}
\begin{enumerate}
	\item Kanonen skal kunne drejes med en nøjagtighed på \(\pm\) 5 \(\degree\)
		\begin{enumerate}
			\item Vertikalt gælder dette for intervallet fra 0 til \(70\degree\)
			\item Horizontalt gælder dette for intervallet fra -45 til \(45\(\degree\)
		\end{enumerate} 
	\item Kanonen skal kunne affyre projektiler med en diameter på 1,25 cm \(\pm\) 2 mm
	\item Kanonen skal kunne affyre sit projektil minimum 1 meter
	\item Kanonens størrelse må maksimalt være 40cm høj, bred og dyb
	\item Fra aftryk på trigger til affyring må der maksimalt gå ti sekunder
	\item Affyring af kanonen skal kunne afvikles tre gange pr. minut
	\item Figur \ref{ref:brugergraesefladeskitse}  viser en skitse af hvordan den grafiskbrugergrænseflade kommer til at se ud
		\begin{figure}[h]
			\centering
			\includegraphics[width=\textwidth]{Kravspecifikation/images/brugergraensefladeskitse}
			\caption{Skitse af brugergrænsefladen}
			\label{ref:brugergraesefladeskitse}
		\end{figure}
\end{enumerate}