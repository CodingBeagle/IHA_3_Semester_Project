\chapter{Accepttestspecifikation}

\section{Use case 1 - Hovedscenarie}
\begin{tabular}{|>{\hspace{0pt}}p{0.6cm} |  >{\hspace{0pt}}p{3.5cm} | >{\hspace{0pt}}p{2.5cm} | p{2.5cm} | p{2cm} |}
	\hline
	Step & Handling & Forventet observation/resultat& Faktisk observation/resultat & Vurdering (OK/FAIL)\\ \hline
	1 & Vælg one-player mode. & Brugergrænsefladen viser spilside for one-player mode og anmoder om valg af antal skud. & & \\ \hline
	
	2 & Vælg ti skud. & Brugergrænseflade anmoder om, at der fyldes ti stykker slik i magasin. & & \\ \hline
	
	3 & Fyld ti stykker slik i magasinet og tryk på knap for at starte spil. & Brugergrænseflade går til spilside og anmoder om, at kanon indstilles. & & \\ \hline
	
	4 & Indstil kanon til affyring med Wii-nunchuck. & Kanon indstiller sig svarende til Wii-nunchucks placering. & & \\ \hline
	
	5 & Udløs kanon med trigger på wii-nunchuck. & Kanon udløses. & & \\ \hline
	
	6 & Gentag punkt 4 og 5 ti gange.  & Punkt 4 og 5 gentages.  & & \\ \hline
	
	7 & Kig på brugergrænsefladen. & Brugergrænsefladen viser info om spillet. & & \\ \hline
	
	8 & Tryk på knap for at vende tilbage til starttilstand. & Brugergrænseflade vender tilbage til startside. & & \\ \hline
\end{tabular}

\subsection{Use case 1 - Extension 1}
\begin{tabular}{|>{\hspace{0pt}}p{0.6cm} |  >{\hspace{0pt}}p{3.5cm} | >{\hspace{0pt}}p{2.5cm} | p{2.5cm} | p{2cm} |}
	\hline
	Step & Handling & Forventet observation/resultat& Faktisk observation/resultat & Vurdering (OK/FAIL)\\ \hline
	
	1 & Vælg two-player mode. & Brugergrænsefladen viser spilside for two-player mode og anmoder om valg af antal skud. & & \\ \hline
	
	2 & Vælg ti skud på brugergrænseflade. & Brugergrænseflade anmoder om, at der fyldes ti stykker slik i magasin. & & \\ \hline
	
	3 & Fyld ti stykker slik i magasinet og tryk på knap for at starte spil. & Brugergrænseflade går til spilside og anmoder om, at kanon indstilles. & & \\ \hline
	
	4 & Indstil kanon til affyring med Wii-nunchuck. & Kanon indstiller sig svarende til Wii-nunchucks placering. & & \\ \hline
	
	5 & Udløs kanon med trigger på wii-nunchuck. & Kanon udløses. & & \\ \hline
	
	6 & Giv Wii-nunchuck til den anden spiller. & Den anden spiller modtager Wii-nunchuck.  & & \\ \hline
	
	7 & Gentag punkt 4 til 6 indtil skud er opbrugt. & Punkt 4 til 6 gentages. & & \\ \hline
	
	8 & Kig på brugergrænseflade. & Brugergrænseflade viser info om spil. & & \\ \hline
	
	9 & Tryk på knap for at vende tilbage til starttilstand. & Brugergrænseflade vender tilbage til startside. & & \\ \hline
\end{tabular}

\subsection{Use case 1 - Extension 2}
\begin{tabular}{|>{\hspace{0pt}}p{0.6cm} |  >{\hspace{0pt}}p{3.5cm} | >{\hspace{0pt}}p{2.5cm} | p{2.5cm} | p{2cm} |}
	\hline
	Step & Handling & Forventet observation/resultat& Faktisk observation/resultat & Vurdering (OK/FAIL)\\ \hline
	
	1 & Vælg one-player mode. & Brugergrænsefladen viser spilside for one-player mode og anmoder om valg af antal skud. & & \\ \hline
	
	2 & Vælg ti skud på brugergrænseflade. & Brugergrænseflade anmoder om, at der fyldes ti stykker slik i magasin. & & \\ \hline
	
	3 & Fyld ti stykker slik i magasinet og tryk på knap for at starte spil. & Brugergrænseflade går til spilside og anmoder om, at kanon indstilles. & & \\ \hline
	
	4 & Tryk på knap for afslutning af spil. & Brugergrænseflade vender tilbage til startside. & & \\ \hline
\end{tabular}

\section{Use case 2 - Hovedscenarie}
\begin{tabular}{|>{\hspace{0pt}}p{0.6cm} |  >{\hspace{0pt}}p{3.5cm} | >{\hspace{0pt}}p{2.5cm} | p{2.5cm} | p{2cm} |}
		\hline
		Step & Handling & Forventet observation/resultat& Faktisk observation/resultat & Vurdering (OK/FAIL)\\ \hline
		1 & Tryk start test på brugergrænseflade & Brugergrænsefladen udskriver at SPI og I2C testen er godkendt. Brugergrænsefladen anmoder bruger om tryk på Z på Wii-nunchuck & & \\ \hline
		
		2 & Tryk Z på Wii- nunchuck & Brugergrænsefladen udskriver at Wii-testen er godkendt & & \\ \hline
		
\end{tabular}

\subsection{Use case 2 - Exception 1}
\begin{tabular}{|>{\hspace{0pt}}p{0.6cm} |  >{\hspace{0pt}}p{3.5cm} | >{\hspace{0pt}}p{2.5cm} | p{2.5cm} | p{2cm} |}
	\hline
	Step & Handling & Forventet observation/resultat& Faktisk observation/resultat & Vurdering (OK/FAIL)\\ \hline
	1 & Fjern SPI-kablet fra DevKittet. & & & \\ \hline
	
	2 & Tryk på start test på brugergrænseflade & Brugergrænsefladen udskriver SPI forbindelses fejlmeddelelse. & & \\ \hline
	
\end{tabular}


\subsection{Use case 2 - Exception 2}
\begin{tabular}{|>{\hspace{0pt}}p{0.6cm} |  >{\hspace{0pt}}p{3.5cm} | >{\hspace{0pt}}p{2.5cm} | p{2.5cm} | p{2cm} |}
	\hline
	Step & Handling & Forventet observation/resultat& Faktisk observation/resultat & Vurdering (OK/FAIL)\\ \hline
	1 & Fjern I2C-kabler fra alle I2C slaver. & & & \\ \hline
	
	2 & Tryk på start test på brugergrænseflade & Brugergrænsefladen udskriver I2C forbindelses fejlmeddelelse. & & \\ \hline
	
\end{tabular}

\subsection{Use case 2 - Exception 3}
\begin{tabular}{|>{\hspace{0pt}}p{0.6cm} |  >{\hspace{0pt}}p{3.5cm} | >{\hspace{0pt}}p{2.5cm} | p{2.5cm} | p{2cm} |}
	\hline
	Step & Handling & Forventet observation/resultat& Faktisk observation/resultat & Vurdering (OK/FAIL)\\ \hline
	1 & Disconnect Wii nun-chucken fra systemet. & & & \\ \hline
	
	2 & Tryk på start test på brugergrænseflade & & & \\ \hline
	
	3 & Vent på timeout. & Brugergrænsefladen udskriver Wii Nunchuck forbindelses fejlmeddelelse& & \\ \hline
\end{tabular}

\newpage
\section{Ikke-funktionelle krav}
\begin{tabular}{|>{\hspace{0pt}}p{0.6cm} |  >{\hspace{0pt}}p{3.5cm} | >{\hspace{0pt}}p{2.5cm} | p{2.5cm} | p{2cm} |}
	\hline
	Krav & Test & Forventet observation/resultat& Faktisk observation/resultat & Vurdering (OK/FAIL)\\ \hline
	
	1.1 & Bruger styrer kanon fra "top" position til "bund" posiion, og måler vinkelforskellen. & Den afmålte vinkelforskel må være 70 \degree  \(\pm\) 5 \degree & & \\ \hline
	
	1.2 & Bruger drejer kanonen fra længst til højre til længst til venstre og måler vinkelforskellen. &Den afmålte vinkelforskel ligger indenfor 70 \degree \(\pm\) 5 \degree & & \\ \hline
	
	2 & Et projektil på 1.25 cm i diameter \(\pm\) 5mm affyres fra kanonen. & Projektilet bliver affyret & & \\ \hline
	
	3 & Et projektil affyres, og distancen mellem kanonen og stedet hvor projektilet lander måles. & Distancen er blevet målt til at være større end 1 meter. & & \\ \hline
	
	4 & Mål kanonens dimensioner med en lineal. & Dimensionerne overstiger ikke 40cm x 40cm x 40cm. & & \\ \hline
	
	5 & Tryk på "triggeren" på Wii Nunchuck, og mål med et stopur hvor lang tid der går fra tryk, til kanonen bliver affyret. &Den målte tid er mindre end 10 sekunder. & & \\ \hline
	
	6 & Kanonen affyres 3 gange, og et stopur startes ved første skud, og stoppes ved det tredje skud. &Den målte tid er mindre end 60 sekunder. & & \\ \hline
	
\end{tabular}