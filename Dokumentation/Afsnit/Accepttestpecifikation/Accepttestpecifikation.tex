\chapter{Accepttestspecifikation}
\label{afsnit:accepttestspecifikation}

\section{Use case 1 - Hovedscenarie}
\begin{tabular}{|>{\hspace{0pt}}p{0.6cm} |  >{\hspace{0pt}}p{3.5cm} | >{\hspace{0pt}}p{2.5cm} | p{2.5cm} | p{2cm} |}
	\hline
	Step & Handling & Forventet observation/resultat& Faktisk observation/resultat & Vurdering (OK/FAIL)\\ \hline
	1 & Vælg single-player mode. & Brugergrænsefladen viser spilside for one-player mode og anmoder om, at der fyldes slik i magasinet og at kanon indstilles. & & \\ \hline
	
	
	3 & Fyld slik i kanon. Indstil kanon til affyring med Wii-nunchuck. & Kanon indstiller sig svarende til Wii-nunchucks placering. & & \\ \hline
	
	4 & Udløs kanon med trigger på wii-nunchuck. & Kanon udløses. & & \\ \hline
	
	5 & Gentag punkt 4 og 5 to gange.  & Punkt 4 og 5 gentages.  & & \\ \hline
	
	
	6 & Tryk på knap for at vende tilbage til starttilstand. & Brugergrænseflade vender tilbage til startside. & & \\ \hline
\end{tabular}

\subsection{Use case 1 - Extension 1}
\begin{tabular}{|>{\hspace{0pt}}p{0.6cm} |  >{\hspace{0pt}}p{3.5cm} | >{\hspace{0pt}}p{2.5cm} | p{2.5cm} | p{2cm} |}
	\hline
	Step & Handling & Forventet observation/resultat& Faktisk observation/resultat & Vurdering (OK/FAIL)\\ \hline
	
	1 & Vælg party mode. & Brugergrænsefladen viser spilside for two-player mode og anmoder om valg af antal skud. & &  \\ \hline
	
	2 & Fyld slik i kanon. Indstil kanon til affyring med Wii-nunchuck. & Kanon indstiller sig svarende til Wii-nunchucks placering. & & \\ \hline
	
	3 & Udløs kanon med trigger på Wii-nunchuck. & Kanon udløses. & & \\ \hline
	
	4 & Giv Wii-nunchuck til den anden spiller. & Den anden spiller modtager Wii-nunchuck.  & & \\ \hline
	
	5 & Gentag punkt 5 til 6 indtil skud er opbrugt. & Punkt 5 til 6 gentages. & & \\ \hline
	
	
	6 & Tryk på knap for at vende tilbage til starttilstand. & Brugergrænseflade vender tilbage til startside. & & \\ \hline
\end{tabular}

\subsection{Use case 1 - Extension 2}
\begin{tabular}{|>{\hspace{0pt}}p{0.6cm} |  >{\hspace{0pt}}p{3.5cm} | >{\hspace{0pt}}p{2.5cm} | p{2.5cm} | p{2cm} |}
	\hline
	Step & Handling & Forventet observation/resultat& Faktisk observation/resultat & Vurdering (OK/FAIL)\\ \hline
	
	1 & Vælg single-player mode. & Brugergrænsefladen viser spilside for one-player mode og anmoder om, at kanon indstilles. & & \\ \hline
	
	2 & Tryk på knap for afslutning af spil. & Brugergrænseflade vender tilbage til startside. & & \\ \hline
\end{tabular}

\section{Use case 2 - Hovedscenarie}
\begin{table}[H]
	\centering
	\begin{tabular}{|l|l|l|l|l|}
		\hline
		Step & Handling                                                                        & \begin{tabular}[c]{@{}l@{}}Forventet \\ observation/resultat\end{tabular}                                                                                                          & \begin{tabular}[c]{@{}l@{}}Faktisk \\ observation/resultat\end{tabular} & \begin{tabular}[c]{@{}l@{}}Vurdering \\ (OK/FAIL)\end{tabular} \\ \hline
		1.   & \begin{tabular}[c]{@{}l@{}}Tryk på \\ "Systemtest" \\ på GUI\end{tabular}                 & \begin{tabular}[c]{@{}l@{}}Systemtest brugergrænsefladen\\ vises på Devkit 8000.\end{tabular}                                                                                        &                                                                         &                                                                \\ \hline
		2.   & \begin{tabular}[c]{@{}l@{}}Tryk på \\ "Start Test \\ på GUI"\end{tabular}                 & \begin{tabular}[c]{@{}l@{}}Brugergrænsefladen udskriver\\ at SPI og I2C testen er\\ godkendt. Brugergrænsefladen\\ anmoder brugereren \\ om tryk på Z på Wii-nunchuck\end{tabular} &                                                                         &                                                                \\ \hline
		3.   & \begin{tabular}[c]{@{}l@{}}Tryk 'Z' knappen\\ på \\ Wii-nunchucken\end{tabular} & \begin{tabular}[c]{@{}l@{}}Brugergrænsefladen udskriver\\ at Wii-testen er godkendt\end{tabular}                                                                                   &                                                                         &                                                                \\ \hline
	\end{tabular}
\end{table}

\subsection{Use case 2 - Exception 1}
\begin{table}[H]
	\centering
	\begin{tabular}{|l|l|l|l|l|}
		\hline
		Step & Handling                                                                 & \begin{tabular}[c]{@{}l@{}}Forventet \\ observation/resultat\end{tabular}                                   & \begin{tabular}[c]{@{}l@{}}Faktisk \\ observation/resultat\end{tabular} & \begin{tabular}[c]{@{}l@{}}Vurdering \\ (OK/FAIL)\end{tabular} \\ \hline
		1.   & \begin{tabular}[c]{@{}l@{}}Fjern SPI-kablet\\ fra Devkit 8000\end{tabular} &                                                                                                             &                                                                         &                                                                \\ \hline
		2.   & \begin{tabular}[c]{@{}l@{}}Tryk på \\ "Systemtest" \\ på GUI \end{tabular}          & \begin{tabular}[c]{@{}l@{}}Systemtest bruger-\\ grænsefladen\\ vises på Devkit 8000.\end{tabular}             &                                                                         &                                                                \\ \hline
		3.   & \begin{tabular}[c]{@{}l@{}}Tryk på \\ "Start Test" \\ på GUI\end{tabular}          & \begin{tabular}[c]{@{}l@{}}Brugergrænsefladen \\ udskriver at\\ SPI forbindelsen\\ mislykkedes\end{tabular} &                                                                         &                                                                \\ \hline
	\end{tabular}
\end{table}


\subsection{Use case 2 - Exception 2}
\begin{table}[H]
	\centering
	\begin{tabular}{|l|l|l|l|l|}
		\hline
		Step & Handling                                                                 & \begin{tabular}[c]{@{}l@{}}Forventet \\ observation/resultat\end{tabular}                                   & \begin{tabular}[c]{@{}l@{}}Faktisk \\ observation/resultat\end{tabular} & \begin{tabular}[c]{@{}l@{}}Vurdering \\ (OK/FAIL)\end{tabular} \\ \hline
		1.   & \begin{tabular}[c]{@{}l@{}}Fjern I2C-kabel\\ fra PSoC0\end{tabular}      &                                                                                                             &                                                                         &                                                                \\ \hline
		2.   & \begin{tabular}[c]{@{}l@{}}Tryk på \\ "Systemtest"\\ på GUI\end{tabular} & \begin{tabular}[c]{@{}l@{}}Systemtest bruger-\\ grænsefladen\\ vises på Devkit 8000.\end{tabular}             &                                                                         &                                                                \\ \hline
		3.   & \begin{tabular}[c]{@{}l@{}}Tryk på \\ "Start Test"\\ på GUI\end{tabular} & \begin{tabular}[c]{@{}l@{}}Brugergrænsefladen \\ udskriver at\\ SPI forbindelsen\\ lykkedes og \\ I2C forbindelsen \\ mislykkedes\end{tabular} &                                                                         &                                                                \\ \hline
	\end{tabular}
\end{table}

\subsection{Use case 2 - Exception 3}
\begin{table}[H]
	\centering
	\begin{tabular}{|l|l|l|l|l|}
		\hline
		Step & Handling                                                                     & \begin{tabular}[c]{@{}l@{}}Forventet \\ observation/resultat\end{tabular}                                        & \begin{tabular}[c]{@{}l@{}}Faktisk \\ observation/resultat\end{tabular} & \begin{tabular}[c]{@{}l@{}}Vurdering \\ (OK/FAIL)\end{tabular} \\ \hline

		1.   & \begin{tabular}[c]{@{}l@{}}Tryk på \\ "Systemtest"\\ på GUI\end{tabular}     & \begin{tabular}[c]{@{}l@{}}Systemtest bruger-\\ grænsefladen\\ vises på Devkit 8000.\end{tabular}                  &                                                                         &                                                                \\ \hline
		2.   & \begin{tabular}[c]{@{}l@{}}Tryk på \\ "Start Test"\\ på GUI\end{tabular}     & \begin{tabular}[c]{@{}l@{}}Brugergrænsefladen \\ udskriver at\\ SPI og I2C forbindelsen\\ lykkedes\end{tabular}  &                                                                         &                                                            \\ \hline
		3.   & Afvent timeout                                                               & \begin{tabular}[c]{@{}l@{}}Brugergrænsefladen \\ udskriver at\\ Nunchuck forbindelsen\\ mislykkedes\end{tabular} &                                                                       &                                                             \\ \hline
	\end{tabular}
\end{table}


\section{Ikke-funktionelle krav}
\begin{tabular}{|>{\hspace{0pt}}p{0.6cm} |  >{\hspace{0pt}}p{3.5cm} | >{\hspace{0pt}}p{2.5cm} | p{2.5cm} | p{2cm} |}
	\hline
	Krav & Test & Forventet observation/resultat& Faktisk observation/resultat & Vurdering (OK/FAIL)\\ \hline
	
	1 & Bruger drejer kanonen så lang til venstre og højre som muligt. & Det observeres at kanonen ikke har roteret 360 \degree  & & \\ \hline
	
	2 & Et projektil på 1.25 cm i diameter \(\pm\) 5mm affyres fra kanonen. & Projektilet bliver affyret & & \\ \hline
	
	3 & Mål produktet dimensioner med en lineal. & Dimensionerne overstiger ikke 50cm x 60cm x 50cm. & & \\ \hline
	
	4 & Tryk på 'Z' knappen på Wii Nunchuck, og mål med et stopur hvor lang tid der går fra tryk, til kanonen bliver affyret. &Den målte tid er mindre end 10 sekunder. & & \\ \hline
	
	5 & Kanonen affyres 3 gange, og et stopur startes ved første skud, og stoppes ved det tredje skud. &Den målte tid er mindre end 60 sekunder. & & \\ \hline
	
\end{tabular}