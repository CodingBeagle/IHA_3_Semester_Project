\chapter{Projektgennemførelse}

Under dette afsnit skal vi have underafsnit om følgende emner.


\section{Samarbejdsaftale}
En samarbejdsaftale er god til at få skabt et fælles grundlag for forventninger og ambitioner til et projektforløb. Som indledning til arbejdet og udviklingen af dette projekt blev der udfærdiget en samarbejdskontrakt, som tog afsæt i en tidligere benytte aftale, som nogle af gruppens medlemmer anvendte i forbindelse med forrige semesterprojekt. Den var velstruktureret og havde fungeret godt, og var derfor et godt udgangspunkt for samarbejdskontrakten til dette projekt. Fokus i aftalen er på forventninger til mødedeltagelse, gruppeledelse og ambitioner for selve projektet. I forhold til gruppeledelsen er personligt ansvar og fælles forpligtelse vægtet højt. Udgangspunktet var, at en koordinator stod for sekretæropgaver og for at holde overblik, men at den egentlige ledelse med tillid og ansvar blev pålagt gruppens medlemmer i fællesskab. Tidligt i projektforløbet blev der indført scrum som udviklingsmodel, og dermed blev koordinatoren erstattet med en scrummaster, som faciliterede udviklingen. Den fælles forpligtelse og det overordnede ansvar lå stadig hos de enkelte gruppemedlemmer.

\section{Arbejdsfordeling}
Som udgangspunkt blev gruppen opdelt i to hovedgrupper; en softwaregruppe på 5 personer og en hardware gruppe på 2 personer. Denne inddeling skete på baggrund af studieretning og dermed som resultat af interesser og kompetencer. Hardwaregruppen stod hovedsageligt for motorstyring, og software gruppen delte sig videre ud i mindre grupper på én og to personer. På trods af den indledende opdeling i hardware og software, som tog udgangspunkt i erfaring og interesser, gav arbejdsfordelingen, stadig store muligheder for at blive udfordret, da flere af opgaverne skulle løses inden den relaterede undervisning havde været afholdt. I tilfælde hvor der opstod problemer, var opdelingen ikke mere fast, end at gruppens medlemmer kunne hjælpe hinanden på tværs af de tildelte opgaver. \\

\section{Planlægning}

\section{Projektledelse}

\section{Projektadministration}
Vores version af scrum.
Herunder brug af fælles fildeling.

\section{Noget om systematic kurset}
Det kom sent i forløbet. 
Hvilke elementer af scrum der giver mening og hvad vi har taget med fra kurset. 

\section{Udviklingsforløb}
	\subsection{Sprint 1}
	\textbf{19/2/16 - 14/3/16}\newline
	\textbf{Varighed:} 3 uger\newline
	Formålet med det første sprint var at dokumentere, implementere og teste use case 2, Test Kommunikationsprotokoller.
	Sprintet startede ud med at lave en backlog i Pivotal Tracker. Der blev lavet userstories på baggrund af brugerønsker samt userstories til de dokumenter, der udgør de tre færdige rapporter. Herefter blev to userstories udvalgt til sprint backloggen, der skulle arbejdes på i dette sprint. Disse userstories består af underopgaver, der hørte til implementeringen af hardware og software. Når en underopgave blev færdiggjort, blev den krydset af i userstory'en. Da vi ikke færdiggjorde nogen af disse userstories til fulde, medførte det at vi tilsyneladende ikke har opnået nogen resultater i dette sprint. På burndown chartet kan man ikke se at vi har lavet noget, selvom vi har fået I2C kommunikation til at lykkedes, en fungerende H-bro til motorstyring og en test GUI. \newline
	
	I kravspecifikationen blev det bestemt der skulle anvendes tre PSoC til I2C kommunikationen mellem GUI, motor og nunchuck. Under arkitektur dannelsen så vi at PSoC2 forbundet til nunchucken kun pollede for information og videredesendte det. Ved at fjerne PSoC2 forbundet til nunchuck'en og forbinde den til PSoC1 forbundet til motoren. Dermed overtager PSoC1 funktionaliteten af PSoC2 og vi undgår unødig kommunikation. \newline
	
	Under sprintet udførte vi diverse opgaver, der ikke stod i sprint backloggen. Dette blev gjort da vi havde glemt at definere dem som userstories da vi udvalgt opgaver til sprintbackloggen. Dermed er der blevet udført meget mere under sprintet end der kan ses på Pivotal Tracker. \newline
	
	Under dette sprint har vi anvendt logbøger, i stedet for daglig stand up møder. Der blev hver morgen noteret det, som man egentligt ville have nævnt til de daglige møder. Det gav en del arbejde til scrum masteren, der skulle tjekke hver dag om alle havde lavet dagens indlæg. Derudover forsvandt kommunikationen mellem gruppensmedlemmer i perioder, da der ikke var nogen andre end scrum masteren, der kiggede logbøgerne igennem. Dette gav anledning til en del miskommunikation omkring arbejdsdage og arbejdsopgaver. For at undgå dette vil vi forsøge os med daglige stand up møder, med vejlederen, i næste sprint. \newline
	
	\textbf{Under dette sprint har vi lært:}
	\begin{itemize}
		\item At user stories skal være mere findelt, for at kunne vise udviklingen af projektet
		\item At strukturen for PSoC opsætningen ikke var hensigtsmæssig, da PSoC1 kunne overtage funktionaliteten af PSoC2
		\item At vi skal holde os til sprint backloggen og kun arbejde på de opgaver vi har defineret for sprintet
		\item At der skal bruges mere tid til at opdatere gruppens status
	\end{itemize}
	
	Da sprintet var færdig er vi gået tilbage og ændret på sprint backloggen for både at findele de userstories vi havde inkluderet og for at dokumentere de ting der er blevet udført i dette sprint, som ikke var inkluderet. Derudover tager den erfaring vi har fået med arkitekturen med til et af de næste sprint hvor use case 1 skal implementeres. Da vi ikke havde så meget erfaring med scrum og denne form for udviklingsmetode mistede vi under sprintet overblikket og dermed kom vi for langsomt i gang med implementeringen. Dette skyldes både vores uerfarenhed med scrum og dårlig tidsestimering. Vi opnåede meget i dette sprint, dog opnåede vi ikke det ønskede mål, som var en færdig implementeret use case 2. 
	
	
	
	\subsection{Sprint 2}
	\textbf{16/3/16 - 6/4/16}\newline
	\textbf{Varighed:} 3 uger\newline
	Formålet med anden sprint var at designe og implementere software og hardware til use case 2, Test Kommunikationsprotokoller. Sprintet startede ud med sprintplanlægningsmøde. Her blev der i gruppen blev aftalt hvilke userstories skulle laves under sprintet. Da disse var fastlagt blev der aftalt et sprintplanlægningsmøde med Gunvor, der hjalp med tidsestimering af disse opgaver. Dette var en langsommelig process, men dette var nødvendigt da vi ikke havde anvendt Pivotal Tracker ordentligt og havde problemer med tidsestimering i sidste sprint. Under mødet blev opgaverne også prioriteret, noget som ikke sketet i sidste sprint, dermed blev der skabt et overblik af de vigtigste opgaver til sprintet. \newline
	
	Som noget nyt forsøgte gruppen sig med stand up meetings. Dette var et forsøg på at anvende scrum i et større omfang end vi tidligere har gjort. I en hel uge blev der afholdt morgen møder med vejlederen inden dagens lektioner. Til disse møder blev der nævnt hvad man ellers ville have været skrevet i logbøgerne. Dette gjorde arbejdet meget lettere for scrum masteren, da denne skulle fungere som ordstyrer til disse møder. Møderne gav gruppens medlemmer et overblik som man ikke havde fået ved logbøgerne, da man ikke aktivt skulle opsøge information omkring hvad de andre medlemmer foretog sig. Gruppen oplevede, at fordi vi havde kontakt med vejlederen dagligt, fik vi hurtigere svar til de problemer der opstod og hjælpen var bedre end det vi kunne modtage fra en mail. Der var dog nogle problemer med møderne, da den uge de kørte over, var en uge hvor der skete meget samarbejde. Dermed blev møderne en samtale omkring hvad hardware- og softwaregruppen havde lavet dagen forinden. Stå op møderne ville have fungeret meget bedre hvis de havde foregået i en uge hvor der blev arbejdet mere individuelt. \newline
	
	En af grundene til at opgaverne fra sprint backloggen ikke var fuldført var fordi at der i midten af dette sprint var påske ferie. Dette gav anledning til diskussioner i gruppen da der ikke var enighed omkring antallet af arbejdstimer i ferien. Løsningen på dette var at der blev aftalt to arbejds dage i ferien, hvor der skulle arbejdes på projektet. Resten af ferien var det op til hver gruppemedlem at bestemme hvilke dage der skulle arbejdes. \newline
	
	\textbf{Under dette sprint har vi lært:}
	\begin{itemize}
		\item At give opgaver realistiske tidsestimater
		\item At prioritere opgaver i backloggen
		\item At stå op møder fungerer ikke optimalt når der sker meget samarbejde
		\item At der skal bruges mere indsats for at kommunikere med gruppen 
	\end{itemize}
	
	Under dette sprint havde gruppen mere erfaring med scrum og anvendelsen af Pivotal Tracker end vi havde tidligere. Eftersom gruppen har arbejdet sammen i et længere stykke tid sker der færrer miskommunikationer. Da der stadig er problemer med tidsestimeringen blev backloggen ikke tømt for opgaver, men vi er kommet længere end forrig sprint. Målet for dette sprint er ikke opnået, men der er blevet udført nok til at vi føler at sprintet var vellykket.
	
	\subsection{Sprint 2}
	\textbf{7/4/16 - ??}\newline

\section{Møder}

\section{Konflikthåndtering}

\section{Opnåede erfaringer}

\section{Fremtidigt arbejde}