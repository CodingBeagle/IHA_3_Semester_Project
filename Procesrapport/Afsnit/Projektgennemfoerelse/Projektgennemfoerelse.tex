\chapter{Udviklingsmodel}
I dette afsnit beskrives den udviklingsmodel der anvendes i forbindelse projektets forløb. I dette projekt er der blevet gjort brug af scrum som udviklingsmetode.

\section{Rollebeskrivelser}
Der er tre hovedroller i scrum; produktejeren, udviklingsholdet og scrum masteren. Disse tre roller udgør scrum teamet. Under dette projekt har gruppevejlederen fungeret som produktejeren, udviklingsholdet har bestået af projekt gruppen og rollen som scrum masteren er varetaget af forskellige medlemmer af gruppen.

\subsection{Produktejer}
Når der udvikles et produkt er der som regel en produktejer, der har stor interesse i det endelige produkt. Produktejeren definerer rammerne omkring produktet og prioriterer vigtigheden af opgaver på backloggen. \newline

I dette projektforløb er rollen som produktejer en kunstig rolle, i den forstand at gruppen har sat rammerne omkring projektet og prioriterer opgaver under forløbet. Gruppens vejleder har påtaget sig rollen som produktejer, men har meget lidt indflydelse på udkommet af projektet. Til sprintenes afslutnings møder fungerede vejlederen som en konventionel produktejer. Her blev resultaterne af sprintet fremført for produktejeren, som skulle forholde sig kritisk i forhold til resultaterne.

Under et sprintplanlægningsmøde havde produktejeren en reel indfyldelse på prioritering og opgaverne på sprintbackloggen. Inden mødet med produktejeren havde gruppen afholdt et møde og udvalgt de vigtigste opgaver til sprintet. Herefter blev mødet med produktejeren afholdt, som hjalp med prioriteringen af opgaver og gav foreslag til opgaver, der ikke var taget højde for.  

\subsection{Udviklingshold}
Udviklingsholdet består af en gruppe individer, der arbejder mod et samlet mål; et færdigt delprodukt for hvert sprint. Udviklingsholdet er selvorganiserende, dermed udpeges der ikke nogen holdleder. \newline

I dette projekt bestod udviklingsholdet af gruppemedlemmerne, der hver især har forskellige egenskaber der giver værdi til holdet. Det er udenfor normerne at scrum masteren er en del af udviklingsholdet. Dette er dog tilfældet under dette projektforløb, da det ville hindre projektet at udelade en person fra udviklingsholdet på grund af sin rolle som scrum master. \par
Under førløbet har udviklingsholdet været mere eller mindre selv organiseret med lille indflydelse af højere instanser. Der er dog nogle deadlines og krav, der skulle overholdes fra IHA's side.


\subsection{Scrum master}
En af hovedrollerne i scrum er scrum masteren. Scrum masterens vigtigste rolle er at sikre at processen i projektforløbet er bevaret. Med dette menes at scrum masteren skal sørge for at udviklingsholdet overholder reglerne for scrum ved at coache dem i at anvende scrum. Derudover skal scrum masteren fungere som bindeled mellem produktejeren og udviklingsholdet. Der er mange andre vigtige opgaver som scrum masteren typisk skal varetage sig, så som ordstyrer til daglige morgenmøder, hjælpe holdet med at afgøre hvad der skal udføres i et sprint og holde et alment overblik over gruppens fremskridt. Da scrum anvendes til et semester projekt og ikke et fuldtidprojekt, er der nogle dele af scrum master rollen vi har taget til os, nogle som vi anvender på vores egen måde og nogle som vi ser bort fra. \newline

Under dette projekt overtog et nyt gruppemedlem rollen som scrum master for hvert sprint. Som scrum master havde man to ansvarsområder; holde overblik over logbøgerne og fungere som bindeled mellem produktejeren og udviklingsholdet. \newline

Under projektforløbet blev der ikke afholdt daglige morgenmøder hvor scrum masteren kunne skabe et overblik over gruppen. Dette er blevet erstattet med logbøger, der blev udfyldt hver morgen inden dagens arbejde påbegyndes. Da der ikke arbejdes på projektet så koncentreret som scrum er bygget til, var daglige morgenmøde valgt fra grundet forskellige mødetider i gruppen. \par
I stedet for at fungere som ordstyrer til morgenmøder, skulle scrum masteren læse hvert gruppemedlems logbog igennem. I disse logbøger blev der besvaret tre spørgsmål og noteret eventuelle bemærkninger. De tre spørgsmål, er de samme tre spørgsmål, man ville blive spurgt hvis der var afholdt morgenmøderne:

\begin{itemize}
	\item Hvad lavede du i går?
	\item Hvad skal du lave i dag?
	\item Skal der bruges hjælp?
\end{itemize} 

Hvis der var opstået problemer for en af gruppens medlemmer var det scrum masterens ansvar at reagere på det. I og med at scrum masteren ikke har nogen reel ansvar for gruppen eller på produktet, har scrum masteren ikke noget ansvar til at løse problemerne. Derimod er scrum masteren med til at afhjælpe problemet ved at gøre resten af gruppen opmærksom på de nyopstandne problemer. 

Som bindeled skulle scrum masteren kommunikere med produktejeren, som i dette tilfælde var gruppens vejleder. Til sprintplanlægningsmøder skulle gruppen bestemme hvad der skulle foretage i det næste sprint. Ved rigtig anvendelse af scrum ville produktejeren være til stede til disse møder for at få afstemt forventninger omkring det næste sprint. Derefter ville scrum masteren, med produktejerens ønsker i tankerne, sætte opgaver på sprintbackloggen sammen med udviklingsholdet. Da dette ville være en tidskrævende process, blev det bestemt at vejlederens tilstedeværelse ikke var nødvendig til denne process og blev først konsulteret efter backloggen var blevet udfyldt med opgaver. Til møderne fungerede scrum masteren som ordstyrer og oprettede opgaver på Pivitol Tracker efter gruppens ønsker. \par 
Under forløbet er gruppens dokumenter til dokumentationsrapporten blevet reviewet af en anden gruppe. Her skulle scrum masteren samle dokumenterne, der skulle reviewes, og lave mødeindkaldelser til reviewmøderne.



\section{Udfordringer ved anvendelse af scrum}
I forbindelse med at det er første gang, der anvendes scrum til et semester projekt, er dele af scrum som vi har haft udfordringer med at få inddraget i udviklingsprocessen, hvilket har hindret processen. \newline

\subsection{Kommunikation}
En vigtig del af scrum er kommunikation. Da gruppen bestod af medlemmer fra forskellige studieretninger, var det ikke muligt at afholde daglige morgen møder, og dette var en stor hindring for kommunikationen. Gruppensmedlemmer havde ikke overblik over hvad der foregik end det man selv var i gang med. Som kommunikationsværktøj blev der oprettet en Facebook gruppe. Her kunne gruppens medlemmer aftale arbejdsdage og informere hinanden om vigtige ting. Dette værktøj blev ikke anvendt så godt som det kunne have gjort, dermed skete der en del miskommunikation omkring arbejdsdage og hvornår man skulle mødes. \par

\subsection{Organisering}
Det er ikke muligt for grupper på tredje semester at få et grupperum. Dette ville have gjort en del for kommunikation at man havde et fælles møderum, hvor der kunne arbejdes og hvor scrumboardet kunne placeres. Da dette ikke var en mulighed blev scrumværktøjet Pivitol Tracker anvendt. Der var uklarhed om hvordan dette værktøj skulle anvendes og det tog et par sprint før dette blev afklaret. Her ville det have været en fordel at have et scrumboard med sticky notes, som man arbejder normalt i scrum, eller eventuelt at gøre brug af et mindre kompliceret værktøj til at holde et online scrum-board. Dette kunne f.eks. være trello eller lign. 

\subsection{Sprintbackloggen}
Anvendelsen af Pivitol Tracker's backlog gik galt fra starten, da der ikke blev lavet en veldefineret liste. For hvert sprint blev der lavet nye opgaver alt efter gruppens ønsker til det kommende sprint. Det at der ikke en veldefineret backlog betød at der ikke kunne udvælges opgaver derfra, da disse opgaver ført blev definieret da vi kom i tanke om dem. \par
Under sprintplanlægningsmøderne manglede der struktur og der blev ikke aflagt nok tid til at afholde disse møde. Dette betød at opgaver ikke var veldefineret og disse blev ændret i løbet af sprintet, som er noget der ikke må ske når der køres et sprint. Udover at opgaverne ikke var veldefineret, var der opgaver der ikke var taget højde for. Dermed blev der arbejdet på opgaver, der ikke var på sprintbackloggen og tog tid væk fra de tidsestimerede opgaver. Dette førte til at alle opgaver op sprintbackloggen aldrig blev fuldtført og tidsplanen blev ved med at skride.  


\chapter{Projektgennemførsel}

Under dette afsnit skal vi have underafsnit om følgende emner.

\section{Gruppedannelse}


\section{Samarbejdsaftale}
En samarbejdsaftale er god til at få skabt et fælles grundlag for forventninger og ambitioner til et projektforløb. Som indledning til arbejdet og udviklingen af dette projekt blev der udfærdiget en samarbejdskontrakt, som tog afsæt i en tidligere benyttet aftale, som nogle af gruppens medlemmer anvendte i forbindelse med forrige semesterprojekt. Den var velstruktureret og havde fungeret godt, og den var derfor et godt udgangspunkt for samarbejdskontrakten til dette projekt. Fokus i aftalen er på forventninger til mødedeltagelse, gruppeledelse og ambitioner for selve projektet. I forhold til gruppeledelsen er personligt ansvar og fælles forpligtelse vægtet højt. Udgangspunktet var, at en koordinator stod for sekretæropgaver og for at holde overblik, men at den egentlige ledelse med tillid og ansvar blev pålagt gruppens medlemmer i fællesskab. Tidligt i projektforløbet blev der indført Scrum som udviklingsmodel, og dermed blev koordinatoren erstattet med en scrummaster, som faciliterede udviklingen. Den fælles forpligtelse og det overordnede ansvar lå stadig hos de enkelte gruppemedlemmer.

\section{Arbejdsfordeling}
Som udgangspunkt blev gruppen opdelt i to hovedgrupper; en softwaregruppe på fem personer og en hardwaregruppe på to personer. Denne inddeling skete på baggrund af studieretning og dermed som resultat af interesser og kompetencer. Hardwaregruppen stod hovedsageligt for motorstyring, og software gruppen delte sig videre ud i mindre grupper på én og to personer. På trods af den indledende opdeling i hardware og software, som tog udgangspunkt i erfaring og interesser, gav arbejdsfordelingen, stadig store muligheder for at blive udfordret, da flere af opgaverne skulle løses inden den relaterede undervisning havde været afholdt. I tilfælde hvor der opstod problemer, var opdelingen ikke mere fast, end at gruppens medlemmer kunne hjælpe hinanden på tværs af de tildelte opgaver. \\

\section{Planlægning}

\section{Projektledelse}
I dette projekt er projektledelse anvendt på den måde, at der i hvert sprint har været en ny scrummaster, som har fungeret som en slags sekretær for gruppen. Det har været scrummasterens opgave at have kontakt til vejleder, at lave dagsorden til møderne og indkalde til disse. Dermed har der været meget fokus på fælles forpligtelse i forhold til, at alle gruppemedlemmer har haft ansvar for, at deres egne opgaver blev udført til tiden, så der ikke har været en decideret projektleder, sm har uddelegeret opgaver. Det har gruppen stået for i fællesskab, hvilket også har virket meget tilfredsstillende. 

\section{Projektadministration}
Herunder brug af fælles fildeling.

\section{Scrumkursus ved Systematic}
Som en del af projektet deltog gruppens medlemmer i et scrumkursus som udbydes af Systematic. Som udgangspunkt havde gruppen arbejdet med scrum i to måneder inden kursets start, dermed var der en forforståelse af hvordan scrum skulle anvendes i forbindelse med projekt arbejde. I de følgende afsnit vil de erfaringer som gruppen har gjort sig under de tre kursus gange. 

\subsection{Planning Poker}
En af øvelserne til kurset var planning poker. Her var opgaven at planlægge en flytning, hvor handlingerne for at nå målet skulle tidsestimeres. Under denne øvelse konstaterede gruppen at der var stor uenighed om hvor meget tid der skulle sættes af for at udføre hver delopgave. Der var nogle medlemmer som var optimistiske og forventede at opgaver var hurtigt overstået uden nogle hindringer. Mens der var nogle medlemmer der var pessimistisk og forventede at noget kunne gå galt under udførelsen af opgaverne, som ville forårsage tidsspild. Under denne øvelse fik gruppen diskuteret begrundelserne for deres tidsestimeringer og dermed har gruppen fået erfaringer med en alternativ måde at tidsestimere på. Denne metode er dog ikke blevet brugt under dette projekt forløb.  

\subsection{Scrum Spillet}
En anden af øvelserne til kurset var scrum spillet. Her blev gruppen præsenteret med en backlog af små, veldefineret opgaver. Hver af disse opgaver havde en pointværdi. Når en opgave var fuldført og godkendt af produktejeren, kunne gruppen lægge denne pointværdi til den totale pointsum. Der blev udført tre sprint hver på 15 minutter. Før hvert sprint var der et sprintplanlægningsmøde på 10 minutter og efter hvert sprint var der aflagt fem minutter til retrospective. \newline

Under sprint planlægningsmøderne valgte hver medlem nogle opgaver, som skulle laves i sprintet. Disse kunne enten udføres individuelt eller i grupper. Dette betød også at hver medlem havde ansvar for de opgaver, som man havde sat sig på og dermed var der en forpligtelse til at levere. \newline

Under sprintene var der meget samarbejde omkring arbejdsopgaverne. Når et gruppemedlem konstaterede at der ikke var nok tid til at fuldføre en opgave, blev der hurtigt set på hvilke medlemmer der kunne assistere. Her var gruppen god til at holde overblik og være selvorganiserende. Dette skyldes at kommunikationen fungerede. I modsætning til projektarbejdet, skulle der ikke ydes en indsats for at kommunikere med gruppens medlemmer, da de var omkring en selv. 
I og med opgaverne var veldefinerede oplevede gruppen at der ikke skulle diskuteres omkring hver enkelt detalje. Når der var tvivl omkring opgaverne kunne man konsultere en produktejer som opklarede tvivlen med en klar melding. Herefter havde man en klar melding som gruppen kunne forholde sig til og rette sig efter.  \newline

Til retrospective fik gruppen talt om hvad der fungerede og hvad der ikke fungerede for sprintet. Under første sprint var der ingen struktur omkring hvordan backloggen var sorteret og hvor sprintopgaverne var placeret. Herefter blev der opsat en struktur i form af lommer der var katagoriseret i \textit{Udførte opgaver} og \textit{Backlog}. I Udførte lommen lå godkendte opgaver og i Backlog så de opgaver der endnu ikke var påbegyndt. Disse opgaver var sorteret, så hurtige og lette opgaver lå øverst og kunne tages hvis der ikke var flere opgaver på sprintbackloggen. De sprintopgaver man havde ansvar på fik man i hånden, så man kunne give den til produktejeren når der skulle godkendes en opgave. \newline

\subsection{Perspektivering}
Under dette kursus har gruppen fået en praktisk forståelse omkring brugen af scrum. Vi oplevede at kommunikationen mellem gruppens medlemmer lykkedes, når vi er fysisk tilstede og har en fysisk og veldefineret sprintbacklog, som vi kan forholde os til. Det ville derfor gavne gruppen at have et grupperum hvor et fysisk scrumboard kunne sættes op. Dermed får vi muligheden for at visualisere de opgaver, der skal udføres i sprintet. \newline
I modsætning til projektarbejdet, var der en ansvarsfølelse for de opgaver der skulle udføres i sprintet. Dette skyldes at vi til scrum spillet valgte opgaver efter vores egne evner og med en sikkerhed om at man kan fuldføre opgaven. Til projektet er opgaver valgt med usikkerhed da projektets opgaver omhandler noget man ikke har kendskab til og er en del af en læringsprocess.

\section{Udviklingsforløb}
I de følgende afsnit vil udviklingsforløbet af projektet beskrives for hvert sprint. Herunder vil der de erfaringer der blev gjort for hvert sprint beskrives.
	\subsection{Sprint 1}
	\textbf{19/2/16 - 14/3/16}\newline
	\textbf{Varighed:} 3 uger\newline
	Formålet med det første sprint var at dokumentere, implementere og teste use case 2, Test Kommunikationsprotokoller.
	Sprintet startede ud med at lave en backlog i Pivotal Tracker. Der blev lavet userstories på baggrund af brugerønsker samt userstories til de dokumenter, der udgør de tre færdige rapporter. Herefter blev to userstories udvalgt til sprint backloggen, der skulle arbejdes på i dette sprint. Disse userstories består af underopgaver, der hørte til implementeringen af hardware og software. Når en underopgave blev færdiggjort, blev den krydset af i userstory'en. Da vi ikke færdiggjorde nogen af disse userstories til fulde, medførte det at vi tilsyneladende ikke har opnået nogen resultater i dette sprint. På burndown chartet kan man ikke se at vi har lavet noget, selvom vi har fået I2C kommunikation til at lykkedes, en fungerende H-bro til motorstyring og en test GUI. \newline
	
	I kravspecifikationen blev det bestemt der skulle anvendes tre PSoC til I2C kommunikationen mellem GUI, motor og nunchuck. Under arkitektur dannelsen så vi at PSoC2 forbundet til nunchucken kun pollede for information og videredesendte det. Ved at fjerne PSoC2 forbundet til nunchuck'en og forbinde den til PSoC1 forbundet til motoren. Dermed overtager PSoC1 funktionaliteten af PSoC2 og vi undgår unødig kommunikation. \newline
	
	Under sprintet udførte vi diverse opgaver, der ikke stod i sprint backloggen. Dette blev gjort da vi havde glemt at definere dem som userstories da vi udvalgt opgaver til sprintbackloggen. Dermed er der blevet udført meget mere under sprintet end der kan ses på Pivotal Tracker. \newline
	
	Under dette sprint har vi anvendt logbøger, i stedet for daglig stand up møder. Der blev hver morgen noteret det, som man egentligt ville have nævnt til de daglige møder. Det gav en del arbejde til scrum masteren, der skulle tjekke hver dag om alle havde lavet dagens indlæg. Derudover forsvandt kommunikationen mellem gruppensmedlemmer i perioder, da der ikke var nogen andre end scrum masteren, der kiggede logbøgerne igennem. Dette gav anledning til en del miskommunikation omkring arbejdsdage og arbejdsopgaver. For at undgå dette vil vi forsøge os med daglige stand up møder, med vejlederen, i næste sprint. \newline
	
	\textbf{Under dette sprint har vi lært:}
	\begin{itemize}
		\item At user stories skal være mere findelt, for at kunne vise udviklingen af projektet
		\item At strukturen for PSoC opsætningen ikke var hensigtsmæssig, da PSoC1 kunne overtage funktionaliteten af PSoC2
		\item At vi skal holde os til sprint backloggen og kun arbejde på de opgaver vi har defineret for sprintet
		\item At der skal bruges mere tid til at opdatere gruppens status
	\end{itemize}
	
	Da sprintet var færdig er vi gået tilbage og ændret på sprint backloggen for både at findele de userstories vi havde inkluderet og for at dokumentere de ting der er blevet udført i dette sprint, som ikke var inkluderet. Derudover tager den erfaring vi har fået med arkitekturen med til et af de næste sprint hvor use case 1 skal implementeres. Da vi ikke havde så meget erfaring med scrum og denne form for udviklingsmetode mistede vi under sprintet overblikket og dermed kom vi for langsomt i gang med implementeringen. Dette skyldes både vores uerfarenhed med scrum og dårlig tidsestimering. Vi opnåede meget i dette sprint, dog opnåede vi ikke det ønskede mål, som var en færdig implementeret use case 2. 
	
	
	
	\subsection{Sprint 2}
	\textbf{16/3/16 - 6/4/16}\newline
	\textbf{Varighed:} 3 uger\newline
	Formålet med anden sprint var at designe og implementere software og hardware til use case 2, Test Kommunikationsprotokoller. Sprintet startede ud med sprintplanlægningsmøde. Her blev der i gruppen blev aftalt hvilke userstories skulle laves under sprintet. Da disse var fastlagt blev der aftalt et sprintplanlægningsmøde med vejlederen, der hjalp med tidsestimering af disse opgaver. Dette var en langsommelig process, men dette var nødvendigt da vi ikke havde anvendt Pivotal Tracker ordentligt og havde problemer med tidsestimering i sidste sprint. Under mødet blev opgaverne også prioriteret, noget som ikke sketet i sidste sprint, dermed blev der skabt et overblik af de vigtigste opgaver til sprintet. \newline
	
	Som noget nyt forsøgte gruppen sig med stand up meetings. Dette var et forsøg på at anvende scrum i et større omfang end vi tidligere har gjort. I en hel uge blev der afholdt morgen møder med vejlederen inden dagens lektioner. Til disse møder blev der nævnt hvad man ellers ville have været skrevet i logbøgerne. Dette gjorde arbejdet meget lettere for scrum masteren, da denne skulle fungere som ordstyrer til disse møder. Møderne gav gruppens medlemmer et overblik som man ikke havde fået ved logbøgerne, da man ikke aktivt skulle opsøge information omkring hvad de andre medlemmer foretog sig. Gruppen oplevede, at fordi vi havde kontakt med vejlederen dagligt, fik vi hurtigere svar til de problemer der opstod og hjælpen var bedre end det vi kunne modtage fra en mail. Der var dog nogle problemer med møderne, da den uge de kørte over, var en uge hvor der skete meget samarbejde. Dermed blev møderne en samtale omkring hvad hardware- og softwaregruppen havde lavet dagen forinden. Stå op møderne ville have fungeret meget bedre hvis de havde foregået i en uge hvor der blev arbejdet mere individuelt. \newline
	
	En af grundene til at opgaverne fra sprint backloggen ikke var fuldført var fordi at der i midten af dette sprint var påske ferie. Dette gav anledning til diskussioner i gruppen da der ikke var enighed omkring antallet af arbejdstimer i ferien. Løsningen på dette var at der blev aftalt to arbejds dage i ferien, hvor der skulle arbejdes på projektet. Resten af ferien var det op til hver gruppemedlem at bestemme hvilke dage der skulle arbejdes. \newline
	
	\textbf{Under dette sprint har vi lært:}
	\begin{itemize}
		\item At give opgaver realistiske tidsestimater
		\item At prioritere opgaver i backloggen
		\item At stå op møder fungerer ikke optimalt når der sker meget samarbejde
		\item At der skal bruges mere indsats for at kommunikere med gruppen 
	\end{itemize}
	
	Under dette sprint havde gruppen mere erfaring med scrum og anvendelsen af Pivotal Tracker end vi havde tidligere. Eftersom gruppen har arbejdet sammen i et længere stykke tid sker der færrer miskommunikationer. Da der stadig er problemer med tidsestimeringen blev backloggen ikke tømt for opgaver, men vi er kommet længere end forrig sprint. Målet for dette sprint er ikke opnået, men der er blevet udført nok til at vi føler at sprintet var vellykket.
	
	\subsection{Sprint 3}
	\textbf{7/4/16 - 27/4/16}\newline
	\textbf{Varighed: 3 uger} \newline
	Formålet med tredje sprint var at færdiggøre use case 2, samt få implemeteret dele af hardware til use case 1. Til sprintplanlægningsmødet oplevede gruppen et stort mandefald, da nogle af gruppens medlemmer var syge. Dette hindrede resten af gruppen i at planlægge sprintet efter alles ønsker. Da der var nogle medlemmer, som var dukket op på trods af sygdom, havde et ønske at tage hjem inden dagens lektioner blev dette møde timeboxed til en time. Under denne time fik gruppen fastlagt målet for sprintet og defineret de opgaver der skulle til for at nå dette mål. Da alle gruppens medlemmer ikke var til stede og der var sat den nødvendige tid af til sprintplanlægningsmødet oplevede vi igen at opgaverne ikke var veldefinerede og der var opgaver der manglede på sprint backloggen. \newline
	
	Under dette sprint deltog alle gruppens medlemmer i et scrumkursus som Systematic udbyder. Da scrum er den udviklingsmodel som anvendes til dette projekt, forekom det naturligt for gruppens deltagere at deltage som en gruppe. Hermed kunne de erfaringer og konklusioner som blev draget til kurset blive anvendt til at styrke gruppens kommunikation og samarbejde. 
	Til kurset blev der undervist i omfattende teori omkring scrum, der blev afholdt diskussioner og scrum blev taget i brug gennem øvelser. Øvelserne fokuserede på at træne evner, som har afgørende betydning for scrum. Disse inkluderede tidsestimering, planlægning, kommunikation og organisering. \newline
	
	Kommunikation er afgørende når der arbejdes med scrum. Dette noget vi oplevede da arbejdede med øvelserne til scrumkurset og da vi så på arbejdsgangen på Systematic. Gruppen har indset at logbøgerne, som blev genoptaget efter det fejlede forsøg med fysiske stand up meetings, ikke fungerede i en grad til at de kunne erstatte de fysiske stand up meetings, som scrum kræver. Kommunikationen i gruppen er ikke optimal, da der arbejdes mere individulet end der er blevet gjort hidtil, og der sker ingen kommunikation på tværs af undergrupperne. Derfor oprettes der en Facebook samtale, som skal sørge for kommunikationen. Med dette anvendes en platform som gruppen i forvejen bruger i dagligdagen til at opretholde kommunikationen.  \newline 
	
	\textbf{Under dette sprint har vi lært:}
	\begin{itemize}
		\item At der skal sættes den nødevendige tid af til afholde sprintplanlægsningsmødet
		\item Hvordan scrum fungerer i praksis i forbindelse med en arbejdsplads
		\item At der fortsat skal ydes en større indsats for at kommunikere i gruppen
	\end{itemize}
	
	Under dette sprint opnåede gruppen sit sprintmål, som var færdiggørelse af use case 2. Alt software og hardware, der skal til, for at kunne udføre use case 2 er færdig implementeret og implementation for dele af hardware til use case 1 er påbegyndt. I dette sprint fik gruppen mulighed for at arbejde intensivt med scrum og kunne tage disse erfaringer med til selve projektet. 

	\subsection{Sprint 4}
	\textbf{27/4/16}\newline
	\textbf{Varighed: 3 uger} \newline
	Formålet med fjerde og sidste sprint er at afslutte alle påbegyndte implementationsopgaver.  
	
\section{Møder}
Gennem hele projektforløbet har der været afholdt et ugentligt vejledermøde. Dette møde har undertiden været brugt til at afholde review- og retrospective-møder i forbindelse med Scrum. Det er til hvert vejledermøde tilstræbt at der blev udarbejdet en dagsorden, som kunne bruges som udgangspunkt for referatet af mødet. På hvert møde blev det aftalt, hvornår det næste vejledermøde skulle afholdes. Scrummasteren sørgede for at indkalde til disse møder via mail til vejleder og over Facebook til resten af gruppen. 

Udover de ugentlige vejledermøder er der blevet afholdt interne møder i gruppen, når det har været nødvendigt. Disse blev der indkaldt til over Facebook, så alle gruppemedlemmer var klar over, at de blev afholdt. Disse møder blev afholdt efter behov, og der var altså ikke et fast antal møder om ugen. 


\section{Konflikthåndtering}

\section{Opnåede erfaringer}

\section{Fremtidigt arbejde}