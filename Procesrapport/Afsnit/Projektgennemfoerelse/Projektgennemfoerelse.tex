\chapter{Udviklingsmodel - Scrum}
Under dette projektforløb er scrum udviklingsmodellen brugt til processtyring. Denne udviklingsmodel er en iterativ metode der bruges til styring, organisering og planlægning af produkt udviklingen.

%I dette afsnit beskrives den udviklingsmodel der anvendes i forbindelse projektets forløb. I dette projekt er der blevet gjort brug af scrum som udviklingsmetode.

\section{Rollebeskrivelser}
Der er tre hovedroller i scrum \cite{scrumGuides}; produktejeren, udviklingsholdet og scrum masteren. Disse tre roller udgør scrumteamet. Under dette projekt har gruppevejlederen fungeret som produktejeren, udviklingsholdet har bestået af projektgruppen og rollen som scrum master er skiftevis varetaget af forskellige medlemmer af gruppen.

\subsection{Produktejer}
Når der udvikles et produkt er der som regel en produktejer, der har stor interesse i det endelige produkt. Produktejeren definerer rammerne omkring produktet og prioriterer vigtigheden af opgaver på backloggen. \cite{scrumGuides} \newline

\noindent I dette projektforløb er rollen som produktejer en kunstig rolle, i den forstand at gruppen har sat rammerne omkring projektet og prioriterer opgaver under forløbet. Gruppens vejleder har påtaget sig rollen som produktejer, men har meget lidt indflydelse på udkommet af projektet. Til sprintenes afslutningsmøder fungerede vejlederen som en konventionel produktejer. Her blev resultaterne af sprintet fremført for produktejeren, som skulle forholde sig kritisk i forhold til resultaterne.\newline 

\noindent Under et sprintplanlægningsmøde havde produktejeren en reel indfyldelse på prioritering af opgaver, og opgaverne på sprintbackloggen. Inden mødet med produktejeren havde gruppen afholdt et møde og udvalgt de vigtigste opgaver til sprintet. Herefter blev mødet med produktejeren afholdt, som hjalp med prioriteringen af opgaverne og gav forslag til opgaver, der ikke var taget højde for.  \newline 

\noindent Vejlederens forudsætning for at prioritere de opgaver, der skulle løses i det kommende sprint var at prioritere de læringsmål, der er opstillet i kursusbeskrivelsen \cite{laeringsmaal}. Det var altså vejlederens opgave at vejlede gruppen, så der ved afslutningen af projektet ville være opfyldt så mange læringsmål som muligt. Læringsmålene skulle afspejles i prioriteringen af opgaver i sprintbackloggen. 

\subsection{Udviklingshold}
Udviklingsholdet består af en gruppe individer, der arbejder mod et samlet mål - et færdigt delprodukt for hvert sprint. Udviklingsholdet er selvorganiserende, og dermed udpeges der ikke nogen holdleder. \cite{scrumGuides} \newline

\noindent I dette projekt bestod udviklingsholdet af projektets gruppemedlemmer, der hver især har forskellige egenskaber, der giver værdi til holdet. Det er udenfor normerne, at scrum masteren er en del af udviklingsholdet. Dette er dog tilfældet under dette projektforløb, da det ville hindre projektet, at udelade en person fra udviklingsholdet på grund af sin rolle som scrum master. \newline

\noindent Under førløbet har udviklingsholdet været mere eller mindre selvorganiseret. Der har dog været nogle deadlines og krav, der skulle overholdes fra IHA's side.

\subsection{Scrum master}
En af hovedrollerne i scrum er scrum masteren. Scrum masterens vigtigste rolle er at sikre, at processen i projektforløbet er bevaret \cite{scrumGuides}. Med dette menes, at scrum masteren skal sørge for at udviklingsholdet overholder reglerne for scrum ved at coache dem i at anvende scrum. Derudover skal scrum masteren fungere som bindeled mellem produktejeren og udviklingsholdet. Scrum masterens fornemmeste opgave er altså at facilitere scrumprocessen for gruppens medlemmer. \newline

\noindent I dette projekt har der været en række administrative opgaver, som vi har valgt at tildele scrum masteren. Derfor har scrum masteren i vores projekt også fungeret som en sekretær for gruppen. Dette har blandt andet indebåret at indkalde til vejledermøder, fungere som ordstyrer til disse møder, holde overblik over gruppens arbejde og fremskridt og holde overblik over de daglige opdateringer i logbøgerne. \newline

\noindent I dette projekt har det også været scrum masterens opgave at være ordstyrer for de daglige stå-op-møder i den uge, der blev gjort brug af disse. Det har også været scrum masterens opgave at sørge for at skaffe hjælp, når der opstod problemer med en opgave. Det er ikke scrum masterens opgave at løse problemerne selv, men derimod at synliggøre problemet for teamet. \newline

\noindent Som bindeled skulle scrum masteren kommunikere med produktejeren, som i dette tilfælde var gruppens vejleder. Til sprintplanlægningsmøder skulle gruppen bestemme, hvad der skulle foretages i det næste sprint. Ved rigtig anvendelse af scrum ville produktejeren være til stede til disse møder for at få afstemt forventninger omkring det næste sprint \cite{scrumProductOwner}. Derefter ville scrum masteren, med produktejerens ønsker i tankerne, sætte opgaver på sprintbackloggen sammen med udviklingsholdet. Da dette ville være en tidskrævende process, blev det bestemt, at vejlederens tilstedeværelse ikke var nødvendig til denne proces og blev først konsulteret efter backloggen var blevet udfyldt med opgaver. Til disse møder fungerede scrum masteren som ordstyrer og oprettede opgaver på Pivotal Tracker (se afsnit \ref{section:pivotalTracker}) efter gruppens ønsker. 

\newpage
\chapter{Projektgennemførsel}
I følgende afsnit beskrives essentielle redskaber der er blevet gjort brug af til gennemførelse af projektet. 

\section{Samarbejdsaftale}
For at få et godt projekt, og en god oplevelse i projektgruppen, blev der udarbejdet en samarbejdskontrakt (Bilag/Procesrapport/Samarbejdsaftale.pdf). Denne indeholder aftaler i forhold til forventning af mødedeltagelse, gruppeledelse, og ambitioner for selve projektet. Samarbejdsaftalen blev udarbejdet før at det blev besluttet at bruge scrum, og er ikke blevet revideret siden.

\section{Arbejdsfordeling}
Som udgangspunkt blev gruppen opdelt i to hovedgrupper: en softwaregruppe bestående af Mia, Michael, Kasper, Tenna og Daniel; og en hardwaregruppe bestående af Mikkel og Pernille. Denne inddeling skete på baggrund af studieretning, og dermed som resultat af interesser og kompetencer. I tilfælde hvor der opstod faglige problemer, var opdelingen ikke mere fastsat end at gruppens medlemmer kunne hjælpe hinanden på tværs af hovedgrupperne.

\section{Projektledelse}
Scrum er en udviklingsmodel, der ikke understøtter en projektleder i en traditionel forstand \cite{scrumGuides}. Der har for hvert sprint været en ny scrum master der skulle holde overblik over scrumprocessen. Under dette projektforløb har der været brug for en til at styre opgaver som sekretær og mødeleder. Derfor forekom det gruppen naturligt, at scrum masteren overtog dette ansvar udover sin rolle som scrum master.

\section{Møder}
Gennem hele projektforløbet har der været afholdt et ugentligt vejledermøde. Dette møde har undertiden været brugt til at afholde review- og retrospectivemøder i forbindelse med scrum. Til hvert vejledermøde, blev der udarbejdet en dagsorden, der kunne bruges som udgangspunkt for referatet af mødet. (Bilag/Procesrapport/Møder/) Ved hvert møde blev det aftalt, hvornår det næste vejledermøde skulle afholdes.\newline

\noindent En central del af scrum, er de daglige stå-op-møder \cite{scrumGuides}. Disse har været en udfordring for os at holde, idét at vi har haft forskellige mødetider i løbet af dagen. For at løse dette problem har vi prøvet flere alternativer. Først prøvede vi at bruge logbøger, hvor scrum masteren havde til opgave at kigge logbøgerne igennem for at finde eventuelle problemer. I logbøgerne skulle følgende spørgsmål besvares:

\begin{itemize}
	\item Hvad lavede du i går?
	\item Hvad skal du lave i dag?
	\item Skal der bruges hjælp?
\end{itemize} 

Gruppen erfarede, at logbøgerne ikke var optimale for udførslen af scrum og blev derfor enige om at prøve at afholde fysiske stå op møder hver dag i en uge. Det var udfordrende at få disse til at hænge sammen med alle gruppemedlemmers skemaer, så derfor blev der oprettet en gruppesamtal på facebook. Her skulle alle gruppens medlemmer hver morgen skrive det samme, som skulle skrives logbøgerne. Det blev erfaret, at måden med at afholde stå op møder over Facebook fungerede bedst, og denne metode blev derfor anvendt i resten af projektforløbet. Denne måde at afholde møderne på gav et hurtigt overblik over, hvad alle gruppemedlemmer havde lavet, hvilke planer de havde for dagen. 

\section{Projektadministration}
Under dette projekt er der anvendt nogle værktøjer til at administrere projektet. Værktøjerne er alle internetbaseret, da disse skulle være lettilgængelige for alle gruppemedlemmer, og kunne anvendes på trods af manglende fast grupperum.

\subsection{GitHub}
Til projektet var det vigtigt at have samlet alle projektets filer og dokumenter, således at alle gruppemedlemmer kunne tilgå, og redigere i filerne. Det var også af stor betydning, at der blev holdt en automatisk versionshistorik for alle filer, så alle ændringer blev dokumenterede. Til dette brugte vi versionsstyrings softwaret \textit{Git}, via \textit{GitHub} \cite{git} \cite{gitRepo}. GitHub er en webbaseret løsning, der stiller en gratis Gitserver til rådighed, samt en brugergrænseflade for at kunne tilgå de uploadede filer. \newline

\noindent På figur \ref{ref:GitHubHistorik} ses et udsnit af en versionshistorik for et af projektets filer. Her kan det ses, at ændringer i filen bliver associeret med et gruppemedlem, en kort beskrivelse af ændringen, samt datoen for ændringen.

\begin{figure}[H]
	\centering
	\includegraphics[width=\textwidth]{Projektgennemfoerelse/images/GitHubHistorik}
	\caption{Versionshistorik for et af projektets filer}
	\label{ref:GitHubHistorik}
\end{figure}

\noindent Måden hvorpå gruppen har brugt Git, er at hvert medlem har synkroniseret lokale ændringer løbende til den centrale server. Ved at gøre dette er den nyeste version af projektet altid tilgængeligt for alle andre. \newline

\noindent Git har været en stor hjælp. Det fungerer som en god sikkerhedsmekanisme, da der altid findes en online backup af projektet, samt ældre revisioner. Versionshistorikken er med til at danne et godt overblik over, hvad alle medlemmer arbejder på og retter i. Ved nogle punkter i projektforløbet blev versionshistorikken også brugt til at gendanne tidligere versioner af filer, hvis der ved uheld var introduceret fejl i for eksempel softwarekomponenter. \newline

\noindent En af de negative sider ved at bruge Git, var at gruppen ikke havde erfaring med værktøjet. Derfor skulle der bruges meget tid på at lære at bruge det effektivt. Efterhånden som der blev opnået større erfaring med brugen af Git, endte det med at være et meget værdifuldt værktøj for projektet.  

\subsection{Pivotal Tracker}
\label{section:pivotalTracker}
For at have et scrum board på nettet, som alle kunne tilgå blev Pivotal Tracker \cite{pivotalTracker} anvendt. Pivotal Tracker er et professionelt scrumværktøj, der indeholder mange funktioner. Det er blandt andet scrumboardet, grafer over for eksempel burndowncharts og statistikker. Programmet blev valgt ud fra en anbefaling fra vores vejleder.

\begin{figure}[H]
	\centering
	\includegraphics[width=\textwidth]{Projektgennemfoerelse/images/burnupchart}
	\caption{Autogenereret burnup chart fra Pivotal Tracker}
	\label{fig:burnup}
\end{figure}

\noindent På figur \ref{fig:burnup} ses et burnup chart fra Pivotal Tracker, taget for hele gruppens projektforløb. Denne burnup chart viser hvor mange point der er færdiggjort igennem forløbet.

\subsection{Facebook}
Til kommunikation mellem gruppensmedlemmer blev der oprettet en Facebook \cite{facebook} gruppe. Denne gruppe blev brugt til lave aftaler, fildeling og almen kommunikation. Dette værktøj var en essentiel del af gruppekommunikationen. Alle gruppemedlemmerne havde en forhåndsviden om facebook, og der skulle derfor ikke bruges tid på at lære et nyt værktøj. \newline

\noindent Da der skulle aftales arbejdsdage for påsken blev der oprettet en afstemmning. Med denne afstemmning fik hvert medlem tilkendegivet hvilke dage de havde til rådighed. Resultatet af afstemmningen var de to dage der havde størst opbakning. Dette ses i figur \ref{ref:fbpoll}, som er et udsnit af afstemningen.
\begin{figure}[H]
	\centering
	\includegraphics[scale=0.6]{Projektgennemfoerelse/images/fbpoll}
	\caption{Afstemning omkring arbejdsdage}
	\label{ref:fbpoll}
\end{figure}

\noindent Facebook gruppen fungerede som en opslagstavle, hvor der kunne dele informationer og meddele hinanden omkring hvordan dokumentation skulle håndteres. Under dette projekt er der blevet brugt LateX til dokumentation. Dermed skulle der lave aftaler omkring hvordan referencer skulle håndteres, når det afsnit man refererede til endnu ikke var skrevet. På figur \ref{ref:fblatex} ses, at løsningen til dette problem beskrives med en vejledning omkring hvordan referencer skal håndteres.

\begin{figure}[H]
	\centering
	\includegraphics[scale=0.6]{Projektgennemfoerelse/images/fblatex}
	\caption{Opslag omkring referencer}
	\label{ref:fblatex}
\end{figure}

\noindent Til at erstatte stå op møderne og logbøger er der blevet oprettet en gruppesamtale. I denne chat informerede man hinanden omkring det der var foretaget dagen før og det der skulle foretages på dagen. Et udsnit af denne samtale ses på figur \ref{ref:fbchat}. Denne form for kommunikation fungerede bedre end de andre forsøg der var blevet gjort på at holde daglige stå op møder. 

\begin{figure}[H]
	\centering
	\includegraphics[scale=0.6]{Projektgennemfoerelse/images/fbstandup}
	\caption{Udsnit af gruppesamtalen}
	\label{ref:fbchat}
\end{figure}
\newpage
\noindent Til dette projekt har Facebook været et vigtigt kommunikationsværktøj. Det blev muligt for gruppens medlemmer at kommunikere med hinanden på trods af varierende undervisningstimer og personlige skemaer. Ved at bruge Facebook påmindede man hinanden om at opretholde kommunikationen når man fik en notifikation på sin startside.

\chapter{Scrumkursus ved Systematic}
Som en del af projektet deltog gruppens medlemmer i et scrumkursus som udbydes af Systematic. Som udgangspunkt havde gruppen arbejdet med scrum i to måneder inden kursets start, dermed var der en forforståelse af hvordan scrum skulle anvendes i forbindelse med projektarbejdet. I de følgende afsnit vil de erfaringer som gruppen har gjort sig under kurset blive opsummeret.

\section{Scrum Spillet}
En af øvelserne til kurset var scrum spillet. Her blev gruppen præsenteret med en backlog af små, veldefinerede opgaver. Hver af disse opgaver havde en pointværdi. Når en opgave var fuldført og godkendt af produktejeren, kunne gruppen lægge denne pointværdi til den totale pointsum. Der blev udført tre sprint hver på 15 minutter. Før hvert sprint var der et sprintplanlægningsmøde på 10 minutter og efter hvert sprint var der aflagt fem minutter til retrospective. \newline

\noindent Under sprint planlægningsmøderne valgte gruppen de opgaver, som skulle laves i sprintet. Disse kunne enten udføres individuelt eller i grupper, og hvert gruppemedlem meldte sig selv ind på de opgaver de følte de kunne udføre. Dette betød at hvert medlem havde ansvar for sine egne opgaver, og dermed var der en forpligtelse til at udføre opgaven før sprintets afslutning. \newline

\noindent Under sprintene var der meget samarbejde omkring arbejdsopgaverne. Når et gruppemedlem konstaterede at der ikke var nok tid til at fuldføre en opgave, blev der hurtigt set på hvilke medlemmer der kunne assistere. Her var gruppen god til at holde overblik og være selvorganiserende. I og med opgaverne var veldefinerede, skete der sjældent misforståelser omkring opgavens natur, hvilket betød at arbejdet kunne startes hurtigt. Når der var tvivl omkring opgaverne kunne man konsultere en produktejer som opklarede tvivlen med en klar melding, som gruppen kunne forholde sig til og rette sig efter.  \newline

\noindent Til retrospective fik gruppen talt om hvad der fungerede og hvad der ikke fungerede for sprintet. Under første sprint var der ingen struktur omkring hvordan backloggen var sorteret, og hvor sprintopgaverne var placeret, hvilket skabte forvirring når der var pres på under selve sprintet. Herefter blev der opsat en struktur i form af lommer der var katagoriseret i \textit{Udførte opgaver} og \textit{Backlog}. I \textit{Udførte opgaver} lommen lå godkendte opgaver og i \textit{Backlog} lommen lå de opgaver der endnu ikke var påbegyndt. Disse opgaver var sorteret, så hurtige og lette opgaver lå øverst og kunne tages hvis der ikke var flere opgaver på sprintbackloggen. De sprintopgaver man havde ansvar på fik man i hånden, så man kunne give den til produktejeren når der skulle godkendes en opgave. \newline
Denne øvelse gav os erfaring med, hvor vigtigt det er at have et organiseret arbejdsrum, og veldefinerede opgaver i forbindelse med scrum og sprints.

\section{Perspektivering}
Under dette kursus har gruppen fået en praktisk forståelse omkring brugen af scrum. Vi oplevede at kommunikationen mellem gruppens medlemmer forbedres, når vi er fysisk sammen og har en fysisk og veldefineret sprintbacklog, som vi kan forholde os til. Det ville derfor gavne gruppen at have et grupperum hvor et fysisk scrumboard kunne sættes op. Dermed får vi muligheden for at visualisere de opgaver, der skal udføres i sprintet. \newline

\noindent I modsætning til projektarbejdet var der veldefinerede og simple opgaver for hvert sprint. I projektarbejdet har vi haft problemer med at formulere letforståelige opgaver, hvilket kun skaber forvirring når sprintbackloggen er fastsat, og arbejdet er i gang.\newline

\chapter{Udviklingsforløb}
I bilag (Bilag/Procesrapport/SprintBeskrivelser) ses en detaljeret beskrivelse af hvert sprintforløb, samt erfaringerne gjort under sprintet.

\chapter{Udfordringer ved anvendelse af scrum}
I forbindelse med at det er første gang, vi anvender scrum til et semester projekt, er der dele af scrum som vi har haft udfordringer med at få inddraget i udviklingsprocessen.

\section{Kommunikation}
En vigtig del af scrum er kommunikation. Da gruppen bestod af medlemmer fra forskellige studieretninger, var det ikke muligt at holde daglige stå op møder, hvilket gjorde det uoverskueligt at holde styr på, hvad de andre på teamet lavede. Som kommunikationsværktøj blev der oprettet en Facebook gruppe. Her kunne gruppens medlemmer aftale arbejdsdage og informere hinanden om vigtige hændelser. Dette værktøj blev ikke anvendt så godt som det kunne have været, dermed skete der en del miskommunikation omkring arbejdsdage og mødetider. \newline

\section{Organisering}
Det er ikke muligt for grupper på tredje semester at få et grupperum til semesterprojekt arbejde. Det ville have gjort en del for kommunikationen at have et fællesrum, hvor der kunne arbejdes og hvor scrumboardet kunne placeres. Da dette ikke var en mulighed blev scrumværktøjet Pivotal Tracker anvendt. Der var uklarhed om hvordan dette værktøj skulle anvendes og det tog et par sprint før dette blev afklaret. Her ville det have været en fordel at have et scrumboard med sticky notes, eller eventuelt at gøre brug af et mindre kompliceret værktøj til at holde et online scrum-board. Dette kunne for eksempel være Trello \cite{trello} eller lignende. 

\section{Sprintbackloggen}
Ved anvendelse af Pivotal Tracker's backlog blev der fra start opstillet en række af opgaver. I arbejdet med udviklingen af projektet, blev yderligere tilføjelser dog nødvendige. Her har scrum som agil udviklingsmodel sin fordel, idet, det er muligt at tilføje og ændre i funktionaliteterne for projektet undervejs. 
Grundet manglende erfaring med tidsestimering, har de fleste af sprintbackloggens opgaver været fejlestimerede. Det var en tendens til at udførelsestiden for en opgave blev groft undervurderet. Her kunne det muligvis have været en fordel at have prioriteret mere tid til sprintplanlægninsmøderne, for at få diskuteret opgaverne grundigt igennem. 


\chapter{Perspektivering}
Processtyringen af dette projekt har været præget af en usikkerhed, idét gruppen ikke havde en tidligere erfaring med brugen af scrum. Dette afspejles i de udfordringer gruppen har stødt på under projektforløbet, for eksempel manglende kommunikation, forhastede planlægningsmøder og fejl i tidsestimeringer. Dette har dog ikke forhindret gruppen i at afprøve nye metoder for at afhjælpe disse problemer. \newline

\noindent For at forbedre kommunikationen mellem gruppens medlemmer blev der forsøgt med forskellige metoder såsom logbøgerne og morgenmøderne. Vi oplevede dog at dette først blev forbedret, da Facebook samtalen blev indført. Det gjorde at alle fik en indsigt i hvad andre gruppemedlemmer skulle lave på dagen, og hvad der var blevet lavet dagen forinden. \newline

\noindent Der findes utallige metoder til tidsestimering, såsom estimering med t-shirt størrelser. Under dette projekt blev denne metode anvendt, da denne var et nemt udgangspunkt at starte fra. Det viste sig dog at estimeringerne var fejlvurderet da man ikke havde indsigt i hvor langt tid opgaver tager. T-shirt metoden var en god måde at gør det på, når man ikke har tidsestimeret før. Vi oplevede dog, at jo mere det bruges, jo lettere bliver det at lave nøjagtige tidsestimeringer. Et alternativ til dette er planning poker \cite{planningpoker}, som giver mulighed for flere tidsværdier til hver opgave. Dette kræver dog at man har bedre forståelse for hvor langt tid opgaverne tager og involverer flere diskussioner, når tidsværdierne er spredt over et større interval. 