\chapter{Konklusion}
Der blev i dette projekt brugt scrum som en udviklingsmodel, da denne model kan håndtere ændringer i krav til produktet eller design under projektforløbet. Derved var det muligt at arbejde iterativt, da man kunne tage hensyn for nye ændringer til produktet i næste sprint. \newline

 \noindent For hver sprintopstart blev der bestemt opgaver til udførelse samt tidsestimering af disse. Sprintvarigheden var på 3 uger, hvilket gjorde, at man fik et delmål for projektet der skulles udvikles i sprintet. Det gav en følelse af fremskridt efterhånden som projektetforløbt udviklede sig. Ved at arbejde i sprint blev projektet delt op i små dele. Dette var en fordel da, man løbende under projektet kan revidere sin plan for projektet, og lave passende ændringer. \newline 
 
 \noindent En del af scrum var daglige stå op møder. I løbet af projektet skiftede gruppen mellem flere forskellige måder at afholde disse møder. Efter flere forsøg fandt vi ud af, at det fungerede bedst når disse møder blev afholdt gennem en gruppesamtale på Facebook. Grundlaget for dette var, at gruppens medlemmer under semesteret, har haft forskellige skemaer og mødedage. Disse input gav alle et overblik, hvad vi hver især arbejdede med på dagen, og det der var foretaget dagen forinden. \newline
  
 \noindent I løbet af projektet har rollen som scrum master skiftet for hvert sprint. Dette har været en positiv oplevelse, idét alle der ville prøve rollen, havde chancen for dette. \newline
 
 \noindent Under dette projektforløb har gruppen fået indsigt i den praktiske anvendelse af scrum og hvordan dette kan anvendes i forbindelse med fremtidige semesterprojekter. 

 



 



