\chapter{Konklusion}
Der blev i dette projekt brugt scrum som en udviklingsmodel, da denne model kan håndtere ændringer i krav til produktet eller design under projektforløbet. Derved var det muligt at arbejde iterativt, da man kunne tage hensyn for nye ændringer i et nyt sprint. \newline

 \noindent For hver sprintopstart blev der bestemt opgaver til udførelse samt tidsestimering af disse. Sprintvarigheden var på 3 uger, hvilket gjorde, at man fik et delmål for projektet der skulles udvikles på i sprintet. Det gav en følelse af fremskridt efterhånden som projektet skred fremad. Ved at arbejde i sprint blev projektet delt op i små dele. Dette var en fordel da de små delmål var lettere at starte med end at starte med et stort mål. \newline 
 
 \noindent En del af scrum var daglige stå op møder. I løbet af projektet skiftede gruppen mellem forskelle måder at afholde disse møder. Efter forskellige forsøg fandt vi ud af, at det fungerede bedst når disse møder blev afholdt gennem en gruppesamtale på Facebook. Disse input gav alle et overblik over, hvad vi hver især arbejdede med på dagen, og det der var foretaget dagen forinden. \newline
  
 \noindent I løbet af projektet har rollen som scrummaster skiftet for hvert sprint. Dette har fungeret gnidningsfrit og alle der ville prøve rollen, havde chancen for dette. \newline
 
 \noindent Under dette projektforløb har gruppen fået indsigt i den praktiske anvendelse af scrum og hvordan dette kan anvendes i forbindelse med fremtidige semesterprojekter. 

 



 



