\chapter{Konklusion}
Der blev i dette projekt brugt scrum som en udviklingsmodel, da denne model kan håndtere ændringer i krav til produktet eller design under projektforløbet. Derved var det muligt at arbejde iterativt, da man kunne tage hensyn for de nye ændringer i det nye sprint. \newline
\noindent Under dette projektforløb har gruppen fået indsigt i den praktiske anvendelse af scrum.

 For hver sprintopstart blev der bestemt opgaver til udførelse samt tidsestimeret disse opgaver. Det med at der skulle laves sprint, ca 3 uger, gjord at man fik et del mål, man hele skulle nå, så man så småt kunne se at projekt skred frem ad, som havde en positiv effekt på os. I stedet for at det var et stort mål til sidst på semestret, var det godt med små del mål. \\Der skulle i scrum laves stå op møder hver dag, der blev fundet af for gruppen at det funger bedst med facebook chat, hvor man skrev hvad man skulle lave og hvad man har lavet. Så det gav os et større overblik over hvad der sket ved de andre. Så det var noget alle var gælder for det, som vi nok skulle have gjord fra starten af, for at opretholde  en god kommunikation imellem hinanden. igennem projektet har vi fundet en større viden om hvordan scrum funger og hvilke ting som man kan bruge i de projeket her på studiet. 
\\Så alt i alt har scrum nogle gode redskab, som man med sikkerhed vil tage med videre i sit undervisnings forløb her på Århus ingeniørhøjskole 
 



 



