\chapter{Konklusion}
Der blev i dette projekt brugt scrum som en udviklingsmodel, da denne model kan håndtere ændringer i krav til produktet eller design under projektforløbet. Derved var det muligt at arbejde iterativt, da man kunne tage hensyn for nye ændringer i et nyt sprint. \newline
\noindent Under dette projektforløb har gruppen fået indsigt i den praktiske anvendelse af scrum. 

 \noindent For hver sprintopstart blev der bestemt opgaver til udførelse samt tidsestimering af disse. Sprintvarigheden var på cirka 3 uger, hvilket gjorde, at man fik et delmål for projektet der skulles udvikles på i sprintet. Det gav en følelse af fremskridt efterhånden som projektet skred fremad. Opdelingen af projektet i små dele, var en fordel i forhold til et stort mål i slutningen. \\
 
 \noindent En del af scrum var daglige stå-op møder. I løbet af projektet skiftede vi mellem forskelle typer af møder. I sidste ende fandt vi ud af, at det fungerede bedst med daglige beskeder i en facebook chat. Disse input gav os et bedre overblik over, hvad vi hver især arbejede på den dag, og dagen forinden.
  
 \noindent Vi kunne ved fordel have brugt dette fra starten af for at opretholde en god kommunikation
 I løbet af projektet har vi skiftet scrummaster for hvert sprint. Dette har fungeret gnidningsfrit og alle der ville prøve rollen, havde chancen for dette. \\
 
 \noindent I gennem projektet har vi fået en større viden om hvordan scrum fungerer, og hvilke ting som man kan bruge i de projekter her på studiet. 
\\Så alt i alt har scrum nogle gode redskab, som man med sikkerhed vil tage med videre i sit undervisnings forløb her på Århus ingeniørhøjskole 
 



 



