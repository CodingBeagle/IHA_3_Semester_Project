\chapter{Forord}
Denne rapport er skrevet af projektgruppe 3 i forbindelse med tredje semesterprojekt på IHA. Gruppen består af følgende individer: Mikkel Nielsen (E), Pernille Kjeldgaard (E), Tenna Rasmussen (IKT), Daniel Vestergaard (IKT), Michael Kloock (IKT), Kasper Rieder (IKT) og Mia Konstmann (IKT). Gruppens vejleder er Gunvor Kirkelund. Rapporten afleveres fredag d. 27. maj 2016. 

\chapter{Indledning}
Under dette projekt er der gjort brug af den agile udviklingsmetode scrum\cite{scrumGuides}. Scrum er valgt til styring af dette projekt, da det egner sig til større projekter hvor både hardware og software indgår. Under et projektforløb kan det ske at krav til produktet eller design ændres, og dermed skal der kunne arbejdes iterativt, for at kunne inkorporere disse i en senere iteration. Da scrum egner sig til de behov gruppen søger i en udviklingsmetode, er denne blevet valgt til styring af dette projekt. \newline

\noindent Denne rapport omhandler hvordan gruppen har gjort brug af scrum, hvilke værktøjer der er taget i brug for at fremme udviklingsprocessen, samt de erfaringer gruppen har gjort sig under projektforløbet. \newline

\noindent Under projektforløbet har gruppen skiftet scrum master for hvert sprint. På tabel \ref{table:scrumMasters} ses fordelingen af scrum master roller over projektets sprints.

\begin{table}[H]
	\centering
	\begin{tabular}{|l|l|}
		\hline
		\textbf{Sprint} & \textbf{Scrum master} \\ \hline
		1               & Kasper Rieder        \\ \hline
		2               & Pernille Kjeldgaard  \\ \hline
		3               & Mia Konstmann        \\ \hline
		4               & Tenna Rasmussen      \\ \hline
	\end{tabular}
	\caption{Oversigt af scrum masters}
	\label{table:scrumMasters}
\end{table}

