\chapter{Udviklingsmodel}
I dette afsnit beskrives den udviklingsmodel der anvendes i forbindelse projektets forløb.

	\section{Rollebeskrivelser}
	Der er tre hovedroller i scrum; produktejeren, udviklingsholdet og scrum masteren. Disse tre roller udgør scrum teamet. Under dette projekt har gruppevejlederen fungeret som produktejeren, udviklingsholdet har bestået af projekt gruppen og rollen som scrum masteren er varetaget af forskellige medlemmer af gruppen.
	
	\subsection{Produktejer}
	Når der udvikles et produkt er der som regel en produktejer, der har stor interesse i det endelige produkt. Produktejeren definerer rammerne omkring produktet og prioriterer vigtigheden af opgaver på backloggen. \newline
	
	I dette projektforløb er rollen som produktejer en kunstig rolle, i den forstand at gruppen har sat rammerne omkring projektet og prioriterer opgaver under forløbet. Gruppens vejleder har påtaget sig rollen som produktejer, men har meget lidt indflydelse på udkommet af projektet. Til sprintenes afslutnings møder fungerede vejlederen som en konventionel produktejer. Her blev resultaterne af sprintet fremført for produktejeren, som skulle forholde sig kritisk i forhold til resultaterne.
	
	Under et sprintplanlægningsmøde havde produktejeren en reel indfyldelse på prioritering og opgaverne på sprintbackloggen. Inden mødet med produktejeren havde gruppen afholdt et møde og udvalgt de vigtigste opgaver til sprintet. Herefter blev mødet med produktejeren afholdt, som hjalp med prioriteringen af opgaver og gav foreslag til opgaver, der ikke var taget højde for.  
	
	\subsection{Udviklingshold}
	Udviklingsholdet består af en gruppe individer, der arbejder mod et samlet mål; et færdigt delprodukt for hvert sprint. Udviklingsholdet er selvorganiserende, dermed udpeges der ikke nogen holdleder. \newline
	
	I dette projekt bestod udviklingsholdet af gruppemedlemmerne, der hver især har forskellige egenskaber der giver værdi til holdet. Det er udenfor normerne at scrum masteren er en del af udviklingsholdet. Dette er dog tilfældet under dette projektforløb, da det ville hindre projektet at udelade en person fra udviklingsholdet på grund af sin rolle som scrum master. \par
	Under førløbet har udviklingsholdet været mere eller mindre selv organiseret med lille indflydelse af højere instanser. Der er dog nogle deadlines og krav, der skulle overholdes fra IHA's side.
	
	
	\subsection{Scrum master}
	En af hovedrollerne i scrum er scrum masteren. Scrum masterens vigtigste rolle er at sikre at processen i projektforløbet er bevaret. Med dette menes at scrum masteren skal sørge for at udviklingsholdet overholder reglerne for scrum ved at coache dem i at anvende scrum. Derudover skal scrum masteren fungere som bindeled mellem produktejeren og udviklingsholdet. Der er mange andre vigtige opgaver som scrum masteren typisk skal varetage sig, så som ordstyrer til daglige morgenmøder, hjælpe holdet med at afgøre hvad der skal udføres i et sprint og holde et alment overblik over gruppens fremskridt. Da scrum anvendes til et semester projekt og ikke et fuldtidprojekt, er der nogle dele af scrum master rollen vi har taget til os, nogle som vi anvender på vores egen måde og nogle som vi ser bort fra. \newline
	
	Under dette projekt overtog et nyt gruppemedlem rollen som scrum master for hvert sprint. Som scrum master havde man to ansvarsområder; holde overblik over logbøgerne og fungere som bindeled mellem produktejeren og udviklingsholdet. \newline
	
	Under projektforløbet blev der ikke afholdt daglige morgenmøder hvor scrum masteren kunne skabe et overblik over gruppen. Dette er blevet erstattet med logbøger, der blev udfyldt hver morgen inden dagens arbejde påbegyndes. Da der ikke arbejdes på projektet så koncentreret som scrum er bygget til, var daglige morgenmøde valgt fra grundet forskellige mødetider i gruppen. \par
	I stedet for at fungere som ordstyrer til morgenmøder, skulle scrum masteren læse hvert gruppemedlems logbog igennem. I disse logbøger blev der besvaret tre spørgsmål og noteret eventuelle bemærkninger. De tre spørgsmål, er de samme tre spørgsmål, man ville blive spurgt hvis der var afholdt morgenmøderne:
	
	\begin{itemize}
		\item Hvad lavede du i går?
		\item Hvad skal du lave i dag?
		\item Skal der bruges hjælp?
	\end{itemize} 
	
	Hvis der var opstået problemer for en af gruppens medlemmer var det scrum masterens ansvar at reagere på det. I og med at scrum masteren ikke har nogen reel ansvar for gruppen eller på produktet, har scrum masteren ikke noget ansvar til at løse problemerne. Derimod er scrum masteren med til at afhjælpe problemet ved at gøre resten af gruppen opmærksom på de nyopstandne problemer. 
	
	Som bindeled skulle scrum masteren kommunikere med produktejeren, som i dette tilfælde var gruppens vejleder. Til sprintplanlægningsmøder skulle gruppen bestemme hvad der skulle foretage i det næste sprint. Ved rigtig anvendelse af scrum ville produktejeren være til stede til disse møder for at få afstemt forventninger omkring det næste sprint. Derefter ville scrum masteren, med produktejerens ønsker i tankerne, sætte opgaver på sprintbackloggen sammen med udviklingsholdet. Da dette ville være en tidskrævende process, blev det bestemt at vejlederens tilstedeværelse ikke var nødvendig til denne process og blev først konsulteret efter backloggen var blevet udfyldt med opgaver. Til møderne fungerede scrum masteren som ordstyrer og oprettede opgaver på Pivitol Tracker efter gruppens ønsker. \par 
	Under forløbet er gruppens dokumenter til dokumentationsrapporten blevet reviewet af en anden gruppe. Her skulle scrum masteren samle dokumenterne, der skulle reviewes, og lave mødeindkaldelser til reviewmøderne.
	
	
	
	\section{Fejl ved anvendelse af scrum}
	I forbindelse med at det er første gang, der anvendes scrum til et semester projekt, er dele af scrum som vi har inddraget i udviklingsprocess blevet anvendt forkert og hindret processen. \newline
	
	\subsection{Kommunikation}
	En vigtig del af scrum er kommunikation. Der blev ikke afholdt daglige morgen møder og dette var en stor hindring for kommunikationen. Gruppensmedlemmer havde ikke overblik over hvad der foregik end det man selv var i gang med. Som kommunikationsværktøj blev der oprettet en Facebook gruppe. Her kunne gruppens medlemmer aftale arbejdsdage og informere hinanden om vigtige ting. Dette værktøj blev ikke anvendt så godt som det kunne have gjort, dermed skete der en del miskommunikation omkring arbejdsdage og hvornår man skulle mødes. \par
	Det er ikke muligt for grupper på tredje semester at få et grupperum. Dette ville have gjort en del for kommunikation at man havde et fælles møderum, hvor der kunne arbejdes og hvor scrumboardet kunne placeres. Da dette ikke var en mulighed blev scrumværktøjet Pivitol Tracker anvendt. Der var uklarhed om hvordan dette værktøj skulle anvendes og det tog et par sprint før dette blev afklaret. Her ville det have været en fordel at have et scrumboard med sticky notes, som man arbejder normalt i scrum. 
	
	\subsection{Sprintbackloggen}
	Anvendelsen af Pivitol Tracker's backlog gik galt fra starten, da der ikke blev lavet en veldefineret liste. For hvert sprint blev der lavet nye opgaver alt efter gruppens ønsker til det kommende sprint. Det at der ikke en veldefineret backlog betød at der ikke kunne udvælges opgaver derfra, da disse opgaver ført blev definieret da vi kom i tanke om dem. \par
	Under sprintplanlægningsmøderne manglede der struktur og der blev ikke aflagt nok tid til at afholde disse møde. Dette betød at opgaver ikke var veldefineret og disse blev ændret i løbet af sprintet, som er noget der ikke må ske når der køres et sprint. Udover at opgaverne ikke var veldefineret, var der opgaver der ikke var taget højde for. Dermed blev der arbejdet på opgaver, der ikke var på sprintbackloggen og tog tid væk fra de tidsestimerede opgaver. Dette førte til at alle opgaver op sprintbackloggen aldrig blev fuldtført og tidsplanen blev ved med at skride.  
