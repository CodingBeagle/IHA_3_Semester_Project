\documentclass{article}

% -- PREAMBLE START --
\usepackage[utf8]{inputenc}
\usepackage[T1]{fontenc}
\usepackage{lmodern} % load a font with all the characters

\usepackage{parskip}

\usepackage[danish]{isodate}

% Create front page info
\title{Logbog}
\author{Mia Konstmann}
\date{}
% -- PREAMBLE END --

\begin{document}
	\maketitle
	\tableofcontents
	
	\section{19/2/16}
	\textbf{Hvad har jeg lavet}
	Rettet på kravspec + vejledermøde
	
	\textbf{Hvad skal jeg lave}
	Ingenting
	
	\textbf{Skal der bruges hjælp?}
	Nej
	
	\textbf{Reflektioner}
	Nej.
	
	\section{22/2/16}
	\textbf{Hvad har jeg lavet}
	Kiggede på kasper lave struktur på pivitoltracker
	
	\textbf{Hvad skal jeg lave}
	Ingenting
	
	\textbf{Skal der bruges hjælp?}
	Nej
	
	\textbf{Reflektioner}
	Nej.
	
	\section{23/2/16}
	\textbf{Hvad har jeg lavet}
	Vejledermøde
	
	\textbf{Hvad skal jeg lave}
	Fundet ud af hvad der skal laves i sprintet
	
	\textbf{Skal der bruges hjælp?}
	Nej
	
	\textbf{Reflektioner}
	Nej.
	
	\section{24/2/16}
	\textbf{Hvad har jeg lavet}
	Ikke noget
	
	\textbf{Hvad skal jeg lave}
	Lave systemtest usecase med de andre
	
	\textbf{Skal der bruges hjælp?}
	Nej
	
	\textbf{Reflektioner}
	Nej.	
	
	\section{25/2/16}
	\textbf{Hvad har jeg lavet}
	Systemmtest usecase, acceptestspecifikation til ikke funktionelle krav og testUC
	
	\textbf{Hvad skal jeg lave}
	Ingenting. 
	
	\textbf{Skal der bruges hjælp?}
	Nej
	
	\textbf{Reflektioner}
	Nej.		
	
	
	\section{29/2/16}
	\textbf{Hvad har jeg lavet}
	Ingenting
	
	\textbf{Hvad skal jeg lave}
	Ingenting. 
	
	\textbf{Skal der bruges hjælp?}
	Nej
	
	\textbf{Reflektioner}
	Nej.	
	
	\section{1/3/16}
	\textbf{Hvad har jeg lavet}
	Ingenting
	
	\textbf{Hvad skal jeg lave}
	Vejledermøde, rettelser som kommer op til mødet
	
	\textbf{Skal der bruges hjælp?}
	Nej
	
	\textbf{Reflektioner}
	Nej.	
	
	\section{2/3/16}
	\textbf{Hvad har jeg lavet}
	Lavet IBD og BDD i fællesskab. Påbegyndt signal tabel.
	
	\textbf{Hvad skal jeg lave}
	Færddiggøre signal beskrivelse
	
	\textbf{Skal der bruges hjælp?}
	Nej
	
	\textbf{Reflektioner}
	Nej.	
	
	\section{3/3/16}
	\textbf{Hvad har jeg lavet}
	Ikke noget
	
	\textbf{Hvad skal jeg lave}
	Applikationsmodeller og kommunikationsprotokoller
	
	\textbf{Skal der bruges hjælp?}
	Nej
	
	\textbf{Reflektioner}
	Nej.			
	
	\section{4/3/16}
	\textbf{Hvad har jeg lavet}
	Har startet et udkast systembeskrivelse, lavet applikationsmodeller for hele systemet.
	
	\textbf{Hvad skal jeg lave}
	Ikke noget, holder weekend
	
	\textbf{Skal der bruges hjælp?}
	Nej
	
	\textbf{Reflektioner}
	Nej.	
	
	\section{7/3/16}
	\textbf{Hvad har jeg lavet}
	Ingenting
	
	\textbf{Hvad skal jeg lave}
	Prøve at fordele arbejdsopgaver og implementere dele af test usecase, da vi ikke har DOA i dag.
	
	\textbf{Skal der bruges hjælp?}
	Nej
	
	\textbf{Reflektioner}
	Nej.
	
	\section{8/3/16}
	\textbf{Hvad har jeg lavet}
	Undersøgt I2C og wii-nunchuck 
	
	\textbf{Hvad skal jeg lave}
	Lave mere af det samme og holde vejleder møde
	
	\textbf{Skal der bruges hjælp?}
	Ja, vi mangler en dims til at tilslutte nunchucken til PSoC 
	
	\textbf{Reflektioner}
	Nej.
	
	\section{9/3/16}
	\textbf{Hvad har jeg lavet}
	Hjulpet Kasper og Daniel med I2C nettet. 
	
	\textbf{Hvad skal jeg lave}
	Prøve at få kontakt med Wii-nunchuck over I2C
	
	\textbf{Skal der bruges hjælp?}
	Nej vi har fået dimsen til Wii'en (breakout adapter)
	Jo. Har problemer med at kommunikere med nunchuck
	
	\textbf{Reflektioner}
	Nej.
	
	\section{10/3/16}
	\textbf{Hvad har jeg lavet}
	Prøvede at få kontakt med Wii-nunchuck, men mislykkede. Der var forvirring med hvilken slave addresse nunchucken har. Har spurgt Daniel og Kasper om hjælp og vi prøver igen i morgen.
	
	\textbf{Hvad skal jeg lave}
	Ikke så meget. Måske undersøg lidt mere om nuchucken og I2C.
	
	\textbf{Skal der bruges hjælp?}
	Jo. Har problemer med at kommunikere med nunchuck
	
	\textbf{Reflektioner}
	Nej.		
	
	\section{11/3/16}
	\textbf{Hvad har jeg lavet}
	Ingenting
	
	\textbf{Hvad skal jeg lave}
	Prøve at få kontakt med Wii-nunchuck over I2C sammen med Daniel og Kasper
	
	\textbf{Skal der bruges hjælp?}
	Jo. Har problemer med at kommunikere med nunchuck
	
	\textbf{Reflektioner}
	Nej.
	
	\section{14/3/16}
	\textbf{Hvad har jeg lavet}
	Holdt review og retrospective møde
	
	\textbf{Hvad skal jeg lave}
	Dokumentation til aflevering og ny sprint opstart med userstories + stand up møde
	
	\textbf{Skal der bruges hjælp?}
	Nej, vi har fået Wii'en til at kommunikere med PSoC'en
	
	\textbf{Reflektioner}
	Nej.
	
	\section{15/3/16}
	\textbf{Hvad har jeg lavet}
	udkast til resume af første sprint og afsnittet om gruppe dannelse.
	Udvalg opgaver til andensprint's backlog
	stand up møde 
	
	\textbf{Hvad skal jeg lave}
	Flere user stories der omhandler dokumentations opgaver
	
	
	\textbf{Skal der bruges hjælp?}
	Jeg synes ikke vi har brug for mere hjælp til planlægning til det næste sprint, men de andre gør så vi holder sprint møde i morgen efter stand up mødet.
	
	\textbf{Reflektioner}
	Nej.
	
	\section{16/3/16}
	\textbf{Hvad har jeg lavet}
	SPI undervisning, added ting til sprint backlog 
	
	\textbf{Hvad skal jeg lave}
	Sprint opstart møde, foretage review af gruppe 2
	
	
	\textbf{Skal der bruges hjælp?}
	Nej
	
	\textbf{Reflektioner}
	Jeg troede at alle i gruppen skulle foretage review, men Tenna forduftede pludselige på trods af at vi lige havde siddet og snakket om det. 
	
	\section{17/3/16}
	\textbf{Hvad har jeg lavet}
	Ingenting
	
	\textbf{Hvad skal jeg lave}
	Foretage review møde med gruppe 2
	
	
	\textbf{Skal der bruges hjælp?}
	Nej
	
	\textbf{Reflektioner}
	Nej
	
	\section{18/3/16}
	\textbf{Hvad har jeg lavet}
	Lavet review af gruppe 2
	
	\textbf{Hvad skal jeg lave}
	Ingenting
	
	
	\textbf{Skal der bruges hjælp?}
	Nej
	
	\textbf{Reflektioner}
	Gruppen synes at vi greb reviewet an på en god måde og satte pris på de foreslag vi havde til deres dokumenter. Der var usikkerhed om hvordan versionshistorikken skulle styres for dokumenter. Vi skrev til Michael og fik svar.
	
	\section{19/3/16}
	\textbf{Hvad har jeg lavet}
	Ingenting
	
	\textbf{Hvad skal jeg lave}
	Lave API'er til I2C
	
	
	\textbf{Skal der bruges hjælp?}
	Nej
	
	\textbf{Reflektioner}
	Vi arbejder i denne weekend så vi kan holde ferie i fra torsdag af.
	
	\section{21/3/16}
	\textbf{Hvad har jeg lavet}
	Lavet API'er til wii nunchuck og assisteret Kasper og Daniel med mindre detaljer omkring I2C
	
	\textbf{Hvad skal jeg lave}
	Starte på exercise 6 i HAL for at kunne hjælpe Tenna med spi.
	
	\textbf{Skal der bruges hjælp?}
	Ja. Der er problemer med SPI øvelsen. Får hjælp af Kasper i morgen
	
	\textbf{Reflektioner}
	Jeg er træt af at lave SPI fordi Tenna tror at jeg vil stjæle hendes opgave. Man skal give lidt ansvar væk for at få hjælp. Det er lidt svært at hjælpe en der frygter at vi stjæler fra hende. 
	
	\section{22/3/16}
	\textbf{Hvad har jeg lavet}
	Arbejdet med SPI øvelsen i ti timer
	
	\textbf{Hvad skal jeg lave}
	Arbejde videre med SPI øvelsen
	
	\textbf{Skal der bruges hjælp?}
	Ja. Der er problemer med at tolke på en fejlkode der kommer fra spi\_sync(). Der er ikke dokumenteret nogen steder hvad -22 betyder.
	Der er skrevet til Peter omkring problemet og genoptager arbejdet når jeg har fået et svar.
	
	\textbf{Reflektioner}
	SPI øvelsen er generelt svært at lave, men jeg synes selv at jeg er kommet ret langt selv uden for meget hjælp. Dog er det tidskrævende at lavet noget som man ikke har fået undervisning i og man skal være udholden nok til at fortsætte. Den første del af øvelsen består kun af at builde en færdiglavet device driver og bruge den til en adc og hvor alt er skåret ud i pap. Det tog lidt tid, men set i bakspejlet er det ikke den vildeste opgave og kan ikke se hvorfor denne øvelse volder så mange problemer for nogen. Den anden del er dog lidt sværer, så her har jeg skrevet til Peter omkring hjælp. 
	
\end{document}