\documentclass{article}

% -- PREAMBLE START --
\usepackage[utf8]{inputenc}
\usepackage[T1]{fontenc}
\usepackage{lmodern} % load a font with all the characters

\usepackage{parskip}

\usepackage[danish]{isodate}

% Create front page info
\title{Logbog}
\author{Mia Konstmann}
\date{}
% -- PREAMBLE END --

\begin{document}
	\maketitle
	\tableofcontents
	
	\section{19/2/16}
	\textbf{Hvad har jeg lavet}
	Rettet på kravspec + vejledermøde
	
	\textbf{Hvad skal jeg lave}
	Ingenting
	
	\textbf{Skal der bruges hjælp?}
	Nej
	
	\textbf{Reflektioner}
	Nej.
	
	\section{22/2/16}
	\textbf{Hvad har jeg lavet}
	Kiggede på kasper lave struktur på pivitoltracker
	
	\textbf{Hvad skal jeg lave}
	Ingenting
	
	\textbf{Skal der bruges hjælp?}
	Nej
	
	\textbf{Reflektioner}
	Nej.
	
	\section{23/2/16}
	\textbf{Hvad har jeg lavet}
	Vejledermøde
	
	\textbf{Hvad skal jeg lave}
	Fundet ud af hvad der skal laves i sprintet
	
	\textbf{Skal der bruges hjælp?}
	Nej
	
	\textbf{Reflektioner}
	Nej.
	
	\section{24/2/16}
	\textbf{Hvad har jeg lavet}
	Ikke noget
	
	\textbf{Hvad skal jeg lave}
	Lave systemtest usecase med de andre
	
	\textbf{Skal der bruges hjælp?}
	Nej
	
	\textbf{Reflektioner}
	Nej.	
	
	\section{25/2/16}
	\textbf{Hvad har jeg lavet}
	Systemmtest usecase, acceptestspecifikation til ikke funktionelle krav og testUC
	
	\textbf{Hvad skal jeg lave}
	Ingenting. 
	
	\textbf{Skal der bruges hjælp?}
	Nej
	
	\textbf{Reflektioner}
	Nej.		
	
	
	\section{29/2/16}
	\textbf{Hvad har jeg lavet}
	Ingenting
	
	\textbf{Hvad skal jeg lave}
	Ingenting. 
	
	\textbf{Skal der bruges hjælp?}
	Nej
	
	\textbf{Reflektioner}
	Nej.	
	
	\section{1/3/16}
	\textbf{Hvad har jeg lavet}
	Ingenting
	
	\textbf{Hvad skal jeg lave}
	Vejledermøde, rettelser som kommer op til mødet
	
	\textbf{Skal der bruges hjælp?}
	Nej
	
	\textbf{Reflektioner}
	Nej.	
	
	\section{2/3/16}
	\textbf{Hvad har jeg lavet}
	Lavet IBD og BDD i fællesskab. Påbegyndt signal tabel.
	
	\textbf{Hvad skal jeg lave}
	Færddiggøre signal beskrivelse
	
	\textbf{Skal der bruges hjælp?}
	Nej
	
	\textbf{Reflektioner}
	Nej.	
	
	\section{3/3/16}
	\textbf{Hvad har jeg lavet}
	Ikke noget
	
	\textbf{Hvad skal jeg lave}
	Applikationsmodeller og kommunikationsprotokoller
	
	\textbf{Skal der bruges hjælp?}
	Nej
	
	\textbf{Reflektioner}
	Nej.			
	
	\section{4/3/16}
	\textbf{Hvad har jeg lavet}
	Har startet et udkast systembeskrivelse, lavet applikationsmodeller for hele systemet.
	
	\textbf{Hvad skal jeg lave}
	Ikke noget, holder weekend
	
	\textbf{Skal der bruges hjælp?}
	Nej
	
	\textbf{Reflektioner}
	Nej.	
	
	\section{7/3/16}
	\textbf{Hvad har jeg lavet}
	Ingenting
	
	\textbf{Hvad skal jeg lave}
	Prøve at fordele arbejdsopgaver og implementere dele af test usecase, da vi ikke har DOA i dag.
	
	\textbf{Skal der bruges hjælp?}
	Nej
	
	\textbf{Reflektioner}
	Nej.
	
	\section{8/3/16}
	\textbf{Hvad har jeg lavet}
	Undersøgt I2C og wii-nunchuck 
	
	\textbf{Hvad skal jeg lave}
	Lave mere af det samme og holde vejleder møde
	
	\textbf{Skal der bruges hjælp?}
	Ja, vi mangler en dims til at tilslutte nunchucken til PSoC 
	
	\textbf{Reflektioner}
	Nej.
	
	\section{9/3/16}
	\textbf{Hvad har jeg lavet}
	Hjulpet Kasper og Daniel med I2C nettet. 
	
	\textbf{Hvad skal jeg lave}
	Prøve at få kontakt med Wii-nunchuck over I2C
	
	\textbf{Skal der bruges hjælp?}
	Nej vi har fået dimsen til Wii'en (breakout adapter)
	Jo. Har problemer med at kommunikere med nunchuck
	
	\textbf{Reflektioner}
	Nej.
	
	\section{10/3/16}
	\textbf{Hvad har jeg lavet}
	Prøvede at få kontakt med Wii-nunchuck, men mislykkede. Der var forvirring med hvilken slave addresse nunchucken har. Har spurgt Daniel og Kasper om hjælp og vi prøver igen i morgen.
	
	\textbf{Hvad skal jeg lave}
	Ikke så meget. Måske undersøg lidt mere om nuchucken og I2C.
	
	\textbf{Skal der bruges hjælp?}
	Jo. Har problemer med at kommunikere med nunchuck
	
	\textbf{Reflektioner}
	Nej.		
	
	\section{11/3/16}
	\textbf{Hvad har jeg lavet}
	Ingenting
	
	\textbf{Hvad skal jeg lave}
	Prøve at få kontakt med Wii-nunchuck over I2C sammen med Daniel og Kasper
	
	\textbf{Skal der bruges hjælp?}
	Jo. Har problemer med at kommunikere med nunchuck
	
	\textbf{Reflektioner}
	Nej.
	
	\section{14/3/16}
	\textbf{Hvad har jeg lavet}
	Holdt review og retrospective møde
	
	\textbf{Hvad skal jeg lave}
	Dokumentation til aflevering og ny sprint opstart med userstories + stand up møde
	
	\textbf{Skal der bruges hjælp?}
	Nej, vi har fået Wii'en til at kommunikere med PSoC'en
	
	\textbf{Reflektioner}
	Nej.
	
	\section{15/3/16}
	\textbf{Hvad har jeg lavet}
	udkast til resume af første sprint og afsnittet om gruppe dannelse.
	Udvalg opgaver til andensprint's backlog
	stand up møde 
	
	\textbf{Hvad skal jeg lave}
	Flere user stories der omhandler dokumentations opgaver
	
	
	\textbf{Skal der bruges hjælp?}
	Jeg synes ikke vi har brug for mere hjælp til planlægning til det næste sprint, men de andre gør så vi holder sprint møde i morgen efter stand up mødet.
	
	\textbf{Reflektioner}
	Nej.
	
	\section{16/3/16}
	\textbf{Hvad har jeg lavet}
	SPI undervisning, added ting til sprint backlog 
	
	\textbf{Hvad skal jeg lave}
	Sprint opstart møde, foretage review af gruppe 2
	
	
	\textbf{Skal der bruges hjælp?}
	Nej
	
	\textbf{Reflektioner}
	Jeg troede at alle i gruppen skulle foretage review, men Tenna forduftede pludselige på trods af at vi lige havde siddet og snakket om det. 
	
	\section{17/3/16}
	\textbf{Hvad har jeg lavet}
	Ingenting
	
	\textbf{Hvad skal jeg lave}
	Foretage review møde med gruppe 2
	
	
	\textbf{Skal der bruges hjælp?}
	Nej
	
	\textbf{Reflektioner}
	Nej
	
	\section{18/3/16}
	\textbf{Hvad har jeg lavet}
	Lavet review af gruppe 2
	
	\textbf{Hvad skal jeg lave}
	Ingenting
	
	
	\textbf{Skal der bruges hjælp?}
	Nej
	
	\textbf{Reflektioner}
	Gruppen synes at vi greb reviewet an på en god måde og satte pris på de foreslag vi havde til deres dokumenter. Der var usikkerhed om hvordan versionshistorikken skulle styres for dokumenter. Vi skrev til Michael og fik svar.
	
	\section{19/3/16}
	\textbf{Hvad har jeg lavet}
	Ingenting
	
	\textbf{Hvad skal jeg lave}
	Lave API'er til I2C
	
	
	\textbf{Skal der bruges hjælp?}
	Nej
	
	\textbf{Reflektioner}
	Vi arbejder i denne weekend så vi kan holde ferie i fra torsdag af.
	
	\section{21/3/16}
	\textbf{Hvad har jeg lavet}
	Lavet API'er til wii nunchuck og assisteret Kasper og Daniel med mindre detaljer omkring I2C
	
	\textbf{Hvad skal jeg lave}
	Starte på exercise 6 i HAL for at kunne hjælpe Tenna med spi.
	
	\textbf{Skal der bruges hjælp?}
	Ja. Der er problemer med SPI øvelsen. Får hjælp af Kasper i morgen
	
	\textbf{Reflektioner}
	Jeg er træt af at lave SPI fordi Tenna tror at jeg vil stjæle hendes opgave. Man skal give lidt ansvar væk for at få hjælp. Det er lidt svært at hjælpe en der frygter at vi stjæler fra hende. 
	
	\section{22/3/16}
	\textbf{Hvad har jeg lavet}
	Arbejdet med SPI øvelsen i ti timer
	
	\textbf{Hvad skal jeg lave}
	Arbejde videre med SPI øvelsen
	
	\textbf{Skal der bruges hjælp?}
	Ja. Der er problemer med at tolke på en fejlkode der kommer fra spi\_sync(). Der er ikke dokumenteret nogen steder hvad -22 betyder.
	Der er skrevet til Peter omkring problemet og genoptager arbejdet når jeg har fået et svar.
	
	\textbf{Reflektioner}
	SPI øvelsen er generelt svært at lave, men jeg synes selv at jeg er kommet ret langt selv uden for meget hjælp. Dog er det tidskrævende at lavet noget som man ikke har fået undervisning i og man skal være udholden nok til at fortsætte. Den første del af øvelsen består kun af at builde en færdiglavet device driver og bruge den til en adc og hvor alt er skåret ud i pap. Det tog lidt tid, men set i bakspejlet er det ikke den vildeste opgave og kan ikke se hvorfor denne øvelse volder så mange problemer for nogen. Den anden del er dog lidt sværer, så her har jeg skrevet til Peter omkring hjælp. 
	
	\section{23/3/16}
	\textbf{Hvad har jeg lavet}
	Arbejdet med SPI øvelsen og startet på dokumentering af modul test til nunchuck med kasper
	Team building burger date med Kloock, Daniel og Kasper.
	
	\textbf{Hvad skal jeg lave}
	Arbejde videre med SPI øvelsen og færdiggøre dokumentering af modultest med Kasper
	
	\textbf{Skal der bruges hjælp?}
	Nej, har fået svar fra Peter og kan forhåbenligt færdiggøre SPI øvelsen
	
	\textbf{Reflektioner}
	Havde ønsket at vi var kommet lidt længere med projektet da vi har fri i denne uge. Holder ferie efter i dag.
	
	
	\section{29/3/16}
	\textbf{Hvad har jeg lavet}
	Færdiggjort SPI øvelsen, holdt ferie
	
	\textbf{Hvad skal jeg lave}
	Ingenting

	\textbf{Skal der bruges hjælp?}
	Nej
	
	\textbf{Reflektioner}
	Nej.
	
	\section{29/3/16}
	\textbf{Hvad har jeg lavet}
	Ingenting
	
	\textbf{Hvad skal jeg lave}
	Være med til vejledermøde
	
	\textbf{Skal der bruges hjælp?}
	Nej
	
	\textbf{Reflektioner}
	Nej.
	
	\section{30/3/16}
	\textbf{Hvad har jeg lavet}
	Ingenting
	
	\textbf{Hvad skal jeg lave}
	Ingenting
	
	\textbf{Skal der bruges hjælp?}
	Nej
	
	\textbf{Reflektioner}
	Nej.
		
	\section{31/3/16}
	\textbf{Hvad har jeg lavet}
	Ingenting
	
	\textbf{Hvad skal jeg lave}
	Ingenting
	
	\textbf{Skal der bruges hjælp?}
	Nej
	
	\textbf{Reflektioner}
	Nej.
			
	\section{31/3/16}
	\textbf{Hvad har jeg lavet}
	Ingenting
	
	\textbf{Hvad skal jeg lave}
	Ingenting. Holder weekend
	
	\textbf{Skal der bruges hjælp?}
	Nej
	
	\textbf{Reflektioner}
	Nej.
				
	\section{4/4/16}
	\textbf{Hvad har jeg lavet}
	Ingenting
	
	\textbf{Hvad skal jeg lave}
	Kigge lidt på SPI med Tenna
	
	\textbf{Skal der bruges hjælp?}
	Nej
	
	\textbf{Reflektioner}
	Nej.
	
	\section{5/4/16}
	\textbf{Hvad har jeg lavet}
	Prøvet på at sende fra devkit til PSoC
	
	\textbf{Hvad skal jeg lave}
	Arbejde videre med SPI fra 8-12
	
	\textbf{Skal der bruges hjælp?}
	Nej
	
	\textbf{Reflektioner}
	Nej.
	
	\section{5/4/16}
	\textbf{Hvad har jeg lavet}
	Prøvet på at sende fra devkit til PSoC
	
	\textbf{Hvad skal jeg lave}
	Arbejde videre med SPI fra 8-12
	
	\textbf{Skal der bruges hjælp?}
	Nej
	
	\textbf{Reflektioner}
	Nej.
	
	\section{6/4/16}
	\textbf{Hvad har jeg lavet}
	Arbejdet lidt med SPI
	
	\textbf{Hvad skal jeg lave}
	Afholde retrospective og reviewmøde med gunvor. 
	Sprint plaægningsmøde uden gunvor.
	
	\textbf{Skal der bruges hjælp?}
	Nej
	
	\textbf{Reflektioner}
	Brugte meget af tiden på at diskukere med Tenna og Daniel, der synes at det er ingen problemer i at implementere use case 1 efter dette sprint. Der vil ikke være nok tid til at udføre dette og lave en rapport der er god nok til at bestå. Det giver ikke projektet da vi allerede har implementeret kommunikations protokollerne. 
	
	\section{7/4/16}
	\textbf{Hvad har jeg lavet}
	Skrevet sprint resume afsnit med Mikkel.
	
	\textbf{Hvad skal jeg lave}
	Kigge lidt på Pivotal Tracker og rette lidt på backloggen. Undersøge scrummaster rollen.
	
	\textbf{Skal der bruges hjælp?}
	Nej
	
	\textbf{Reflektioner}
	Nej.
	
	\section{8/4/16}
	\textbf{Hvad har jeg lavet}
	Der blve rettet i backloggen i forhold til at vi ikke havde taget hensyn til at der skulle dokumenteres for GUI'en. Derudover er der lavet en userstory for færdiggørelsen af use case 2, da dette er målet for sprintet. \par
	Der er blevet på begyndt et afsnit om scrum masteren, da gunvor anbefalede hver sccrum master at gøre dette. Den virker ikke særlig personlig, så den vil måske kunne bruges i processrapporten. \par
	Læst logbøger igennem og har ikke bemærket andet end at der ikke bliver skrevet i dem så ofte som der burde.
	
	\textbf{Hvad skal jeg lave}
	Kigge afsnittet omkring scrum igennem og rette noget dokumentation igennem.
	
	\textbf{Skal der bruges hjælp?}
	Nej
	
	\textbf{Reflektioner}
	Nej.

	\section{11/4/16}
	\textbf{Hvad har jeg lavet}
	Der er blevet skrevet i process rapporten og rette på dokumentation med Kasper.
	
	\textbf{Hvad skal jeg lave}
	Arbejde på SPI med de andre på software
	
	\textbf{Skal der bruges hjælp?}
	Nej
	
	\textbf{Reflektioner}
	Nej.
	
	\section{12/4/16}
	\textbf{Hvad har jeg lavet}
	Der er blevet arbejdet med spi sammen med hele software holdet. 
	
	\textbf{Hvad skal jeg lave}
	Ingenting.
	
	\textbf{Skal der bruges hjælp?}
	Nej
	
	\textbf{Reflektioner}
	Der er kommet hul igennem med spi og meget af det bygger på det some Tenna har lavet. Det var der næsten, men med hele gruppen fik vi det til at fungere. Der var nogle forståelsesproblemer med spi, men sammen, har vi kunnet få det til at virke og forstået konceptet bag spi. Der er aftalt at der arbejdes videre på spi og forhåbenlig en færdiggørelse på fredag, dermed sker der ikke meget fra min side af resten af ugen.
	
	\section{13/4/16}
	\textbf{Hvad har jeg lavet}
	Ingenting
	
	\textbf{Hvad skal jeg lave}
	Ingenting.
	
	\textbf{Skal der bruges hjælp?}
	Nej
	
	\textbf{Reflektioner}
	Nej.
	
	\section{14/4/16}
	\textbf{Hvad har jeg lavet}
	Ingenting
	
	\textbf{Hvad skal jeg lave}
	Til scrum kursus ved systematic.
	
	\textbf{Skal der bruges hjælp?}
	Nej
	
	\textbf{Reflektioner}
	Det var lærerigt og vi fik nye erfaring med scrum. Manden havde mange gode ting at sige, men det gik en smule stærkt til tider. Vi oplevede den forskel vurdering havde i vores planlægning. Der var en der faldt i vandet. 
	
	\section{15/4/16}
	\textbf{Hvad har jeg lavet}
	Var til scrum kursus
	
	\textbf{Hvad skal jeg lave}
	Arbejde på spi dokumentation og produkt
	
	\textbf{Skal der bruges hjælp?}
	Nej
	
	\textbf{Reflektioner}
	Nej.
	
	\section{18/4/16}
	\textbf{Hvad har jeg lavet}
	Rettet arkitektur dokument igennem for software. Vejledermøde med Gunvor.
	
	\textbf{Hvad skal jeg lave}
	Arbejde på spi dokumentation og spi
	
	\textbf{Skal der bruges hjælp?}
	Nej
	
	\textbf{Reflektioner}
	Der er planlagt arbejdsdage i denne uge hvor vi sigter efter at have test usecasen færdig implementeret. Der skal arbejdes på en DSB aflevering i næste uge.
	

\end{document}