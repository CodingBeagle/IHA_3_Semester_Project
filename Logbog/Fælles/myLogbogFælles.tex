\documentclass{article}

% -- PREAMBLE START --
\usepackage[utf8]{inputenc}
\usepackage[T1]{fontenc}
\usepackage{lmodern} % load a font with all the characters

\usepackage{parskip}

\usepackage[danish]{isodate}

% Create front page info
\title{Logbog}
\author{Gruppe 3}
\date{}
% -- PREAMBLE END --

\begin{document}
	\maketitle
	\tableofcontents
	
	\section{16/02/2016 8.30-13.30}
	
	\textbf{Hvad har vi lavet}
	\begin{itemize}
		\item Rigt billede
		\item Lavet udkast til projektformulering
		\item Template til rapportskrivning med dertilhørende mappestruktur
		\item Product backlog
		\item Påbegyndt kravspec (Fully use case beskrivelse for brugsscenarie)
		\item Påbegyndt scrum iterationstabel
		\item Rettelser i tidsplan
		\item Sprint tidsplan
	\end{itemize}
	
	\textbf{Hvad skal vi lave}
	\begin{itemize}
		\item Kravspec
		\subitem Aktør-kontekstdiagram
		\subitem Aktør beskrivelser
		\subitem Use-case diagram
		\subitem Færddiggøre fully dressed usecase
		\subitem Ikke-funktionelle krav
		
		\item Opsætte Redmine Wiki (Kontakte søren)
		\item Scrum iterationstabel
		\item Gøre alt klar til at sende til Gunver (Inkludere git link)
		
	\end{itemize}
	\textbf{Skal der bruges hjælp?}
	Vi skal kontakte søren for at få ham til at lave en wiki side til vores semesterprojektgruppe. Wiki siden skal bruges som SCRUMBOARD til projektet.
	
	
	\textbf{Reflektioner}
	Vi har opsat en fælles Logbog. Denne logbog er kun til brug i opstartsfasen, hvor vi alle har arbejdet sammen. Når vi kommer ind i selve scrum-sprintene vil opgaverne blive opdelt og derefter noteres der i individuelle logbøger. 
	
	

\end{document}