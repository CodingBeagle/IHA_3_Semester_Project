\documentclass{article}

% -- PREAMBLE START --
\usepackage[utf8]{inputenc}
\usepackage[T1]{fontenc}
\usepackage{lmodern} % load a font with all the characters

\usepackage{parskip}

\usepackage[danish]{isodate}

% Create front page info
\title{Referat af Vejledermøde \#1}
\author{Gruppe 3 \\Vejleder: Gunvor Elisabeth Kirkelund}
\date{12.02.2016 - kl. 12:00}
% -- PREAMBLE END --

\begin{document}
	\maketitle
	
	\section{Fremmødte}
	\begin{itemize}
		\item Kasper R.
		\item Daniel J.
		\item Mia K.
		\item Pernille K.
		\item Mikkel N.
		\item Michael K.
		\item Tenna R. 
	\end{itemize}
	
	\section{Udeblevet med afbud}
	Ingen udeblev.
	
	\section{Udeblevet uden afbud}
	Ingen udeblev.
	
	\section{Dagsorden}
	
	\begin{itemize}
		\item{\textbf{Læringsmål}}
		\subitem{Iterativ Produktudvikling med Scrum.}
		\subitem{Det er vigtigt at arbejde imod et produkt der faktisk virker.}
		\item{\textbf{Rigt Billede}}
		\subitem{Fælles forståelse for produktet vi laver.}
		\subitem{Koncept-billede.}
		\subitem{Skitse af det samlede produkt.}
		\subitem{Kan sætte billeder ind, bruge blokke eller lave en "rigtig" tegning.}
		\item{\textbf{Scrum}}
		\subitem{Hver især fører en daglig logbog over hvad der er lavet dagen før og hvad der skal laves på dagen, samt om man har brug for hjælp. Fælles værktøj til at samle logbøger.}
		\subitem{Hav en Scrum master der ser logbøgerne igennem for problemer der skal mødeindkaldes til.}
		\subitem{Hver 3. uge bliver der holdt review og sprintet og retrospective.}
		\subitem{Anbefalet 3 uger sprint.}
		\subitem{Scrum master kan roteres for hvert sprint (Det er godt at prøve at være Scrum Master).}
		\subitem{Almindelig ugentlig møde med vejleder.}
		\subitem{Minimumskrav til Scrum: Udføre sprints, valg af Scrum Master, Scrumboard (Består af opgaverne man er igang med og hvornår de er færdige, samt et sprint review når sprintet er færdigt.)}
		\item{\textbf{Logbøerne består af}}
		\subitem{Hvad har jeg lavet.}
		\subitem{Hvad skal jeg lave.}
		\subitem{Skal der bruges hjælp?}
		\subitem{Eventuelt reflektioner.}
		\subitem{\textbf{Det er vigtigt at logbøgerne bliver ført hver dag!}}
		\item{\textbf{Husk}}
		\subitem{Point fås primært for software til embedded Linux.}
		\subitem{Dokumentation hedder nu \textit{bilag}.}
		\subitem{Rapport skal være mellem 30-50 sider og indeholde alle overvejelser om projektet (Vi gjorde følgende ting pga. af det her og dette var resultatet. Man skal kunne argumentere for valg). Projektstyring/Udviklingsprocessen skal også beskrives.}
		\subitem{Rapporten er den karakter gives ud fra.}
		\subitem{Der holdes møde på fredag kl 12:15.}
		\item{\textbf{To-Do}}
		\subitem{Lav en plan over hvornår hvert sprint forløber med alle datoer (en tidsplan).}
		\subitem{Udkast til backlog/Tasklist. Skriv tasks på i så små dele som muligt, samt hvor lang tid man tor det tager (brug eventuelt planning poker).}
		\subitem{Valg af Scrum Master.}
		\subitem{Sæt rapporten op fra starten og tænk over hvad der skal med i den.}
	\end{itemize}
	
\end{document}