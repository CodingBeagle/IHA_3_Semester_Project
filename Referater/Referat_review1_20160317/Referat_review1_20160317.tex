\documentclass{article}

% -- PREAMBLE START --
\usepackage[utf8]{inputenc}
\usepackage[T1]{fontenc}
\usepackage{lmodern} % load a font with all the characters

\usepackage{parskip}

\usepackage[danish]{isodate}

% Create front page info
\title{Referat af Vejledermøde \#...}
\author{Gruppe 3}
\date{12.02.2016 - kl. 12:00}
% -- PREAMBLE END --

\begin{document}
	\maketitle
	
	\section{Fremmødte}
	\begin{itemize}
		\item ...
		\item ...
	\end{itemize}
	
	\section{Udeblevet med afbud}
	\begin{itemize}
		\item ...
		\item ...
	\end{itemize}
	
	\section{Udeblevet uden afbud}
	\begin{itemize}
		\item ...
		\item ...
	\end{itemize}
	
	\section{Dagsorden}
	\begin{itemize}
		\item Side 3
		\subitem Kluntet, slikkanon til Goofy Candygun. Husk fodnoter! (youtube-link) 
		Spilstatistikker. 
		\item Controller i første halvdel af indledning, lige plidselig blvier det til wii-nunchuck. 
		\subitem Hvorfor slik? I det rige billede ser det lidt meningsløst ud, at man skyder slik op på en væg. 
		\item Skriv hvorfor den skal have en motor. Den skal kunne dreje frem og tilbage. 
		\item Det er meget specifikt, at vi skriver Devkit. Start med at skrive, at det skal være en indlejret linux-platform. 
		\item Horeunger. 
		\item Meget hårdt opdelt i HW og SW. Lyder som om, at vi har to grupper der arbejder. Iterativ proces, det behøver ikke være således. 
		\item Systembeskrivelse
		\subitem PSoC0, PSoC1, PSoC2 - PSoC-kittet har to chips, der hedder PSoC4 og PSoC5. 
		\subitem Det er upraktisk og dyrt, at have tre PSoC'er. 
		\subitem Sensoren er ikke med på figur 2. 
		\subitem Sensorer: Kim siger, at sensorer er aktører. Men kun hvis de kan starte en use case. Måske. 
		\item Dokumentation 
		\begin{itemize}
			\item Use case 1
			\subitem Hvorfor skal man selv fylde slik i og hvorfor skal det være præcist så mange skud. 
			\subitem Kunne være en god ide, at have sensor, der holder øje med slik. 
			\subitem At man skal fylde slik i og kan afslutte spil strider mod hinanden. 
			\subitem Husk termliste. 
			\item Use case 2
			\subitem Der står ikke noget om at PsoC'en finder ud af at nunchucken ikke er der. 
			\subitem Bedre navne til PSoC-blokke. 
			\subitem Det ville være smart, hvis use casen ikke stopper første gang den støder på en fejl. Hvis det er PSoC2 der stopper kan den stadig godt teste PSoC1. 
			\item Accepttest
			\subitem I use casen ser det ud som om man vælger to spillere midt i use casen. i accepttesten gør man det som det første. 
			\item Blokbeskrivelse skal komme efter diagrammer. 
			\subitem Vær konsekvent med skrifttype og kursiv! og stavemåder! 
			\item Signalbeskrivelse
			\subitem Beskriv hvad 0 volt er, referencespænding til stel. 
			\subitem Beskriv hvad høj og lav er (høj kan godt gå fra 3,3 til 5 volt)
			\item BDD
			\subitem Strømforsyning
			\item IBD 
			\subitem Strømforsyning, specificerer kun de porte hvor der sker en datatransmission. Så sætter en blok, hvor dette står. 
			\item side 20
			\subitem Al information sendes med strenge. Brug evt. enum i stedet.
			\subitem Brugergrænseflade. Nu er vi i denne state, så skal jeg se sådan ud. State machines. Sekvensdiagrammer er måske lidt for detaljeret til dette. 
			\subitem Klassediagram, ComProtocol. 
		\end{itemize}
		\subitem Spil spillet er en stor use case. Dele op i flere use cases. 
		\subitem 
	\end{itemize}
\end{document}