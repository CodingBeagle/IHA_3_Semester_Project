\documentclass{article}

% -- PREAMBLE START --
\usepackage[utf8]{inputenc}
\usepackage[T1]{fontenc}
\usepackage{lmodern} % load a font with all the characters

\usepackage{parskip}

\usepackage[danish]{isodate}

% Create front page info
\title{Referat af Vejledermøde \#...}
\author{Gruppe 3}
\date{29.03.2016 - kl. 08.15}
% -- PREAMBLE END --

\begin{document}
	\maketitle
	
	\section{Fremmødte}
	Mia, Kasper, Mikkel, Michael, Tenna, Daniel, Pernille
	
	\section{Udeblevet med afbud}
	Ingen. 
	
	\section{Udeblevet uden afbud}
	Ingen. 
	
	\section{Dagsorden}
	\begin{itemize}
		\item Valg af mødeleder
		
		
		\item Valg af referent
		Pernille. 
		
		\item Godkendelse af referat fra forrige møde 
		Der er ikke rigtig noget. 
		
		\item Opfølgning på aktionspunkter fra forrige møde
		\begin{itemize}
			\item Der er ikke rigtig nogle. 
		\end{itemize}
		
		\item Hvor er vi? 
		\begin{itemize}
			\item I2C-kommunikation
			\subitem Der er lavet en protokol.
			\subitem Begyndt på modultest. 
			\subitem Lige nu er der taget screenshots af analog discovery. Visuelle tests. Kan jeg læse ud fra det hvordan det er testet og er det testet grundigt nok? Er der nogle testscenarier, der ikke er udført? 
			\subitem Det er blevet besluttet, at bruge to PSoC'er i stedet for tre. 
			\subitem Længere forklaring hvorfor i dokumentationen, men kortere i rapporten - metodeafsnit. 
			\item Motorstyring
			\subitem H-broen virker. 
			\subitem 
			\item SPI
			\subitem Er kommet ud af jeg kan ingenting blokaden, så nu er aftener sat af til det. 
			\item GUI 
			\subitem Det går stille og roligt fremad. 
			\subitem Det til spillet er begyndt at arbejde på. 
		\end{itemize}
		\item Affyringsmekanisme
		\begin{itemize}
			\item Lost på hvordan man skal få den til at skyde af. 
			\item Er der nogle andre der har ideer til, hvordan det kan lade sig gøre. 
			\item Google helt vildt! Snak med Rasmus. 
		\end{itemize}
		\item Modultest H-bro - hvor meget skal med? 
		\begin{itemize}
			\item Aflæsning på et skop. 
			\item Teknisk set er det meningen, at man skal lave en fuld forsøgsrapport. Men den kan godt skrumpes en hel del. 
			\item Screenshots. 
		\end{itemize}
		\item Integrationstest og modultest
		\begin{itemize}
			\item Se tidligere punkt. 
			\item ...
		\end{itemize}
		\item Opfølgning på spørgsmål fra påsken
		\subitem Se tidligere punkt. 
		\item Gennemgang af tidsplan
		\begin{itemize}
			\item Sprintet slutter i næste uge. 
			\item Vi er on track og når nok at blive færdige til tiden. 
			\item Gunvor tænker ikke, at vi er bagud. 
		\end{itemize}
		\item Nye aktionspunkter til næste møde
		\begin{itemize}
			\item Færdiggøre sprint backloggen. 
			\item Send ting til Gunvor, så hun kan kigge det igennem inden review. 
		\end{itemize}
		\item Tidspunkt for næste møde
		\subitem Vi skal lave retrospective i næste uge efter sprintet er slut.
		\subitem Torsdag d. 7. april kl. 14.15 holder vi review og retrospective. 
		\subitem I Shannon. 
	\end{itemize}
\end{document}