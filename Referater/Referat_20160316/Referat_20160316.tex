\documentclass{article}

% -- PREAMBLE START --
\usepackage[utf8]{inputenc}
\usepackage[T1]{fontenc}
\usepackage{lmodern} % load a font with all the characters

\usepackage{parskip}

\usepackage[danish]{isodate}

% Create front page info
\title{Referat af Vejledermøde \#...}
\author{Gruppe 3}
\date{16.03.2016 - kl. 08:00}
% -- PREAMBLE END --

\begin{document}
	\maketitle
	
	\section{Fremmødte}
	\begin{itemize}
		\item Kasper
		\item Mikkel
		\item Mia
		\item Michael
		\item Tenna
		\item Pernille
		\item Daniel
	\end{itemize}
	
	\section{Udeblevet med afbud}
	\begin{itemize}
		\item Ingen
	\end{itemize}
	
	\section{Udeblevet uden afbud}
	\begin{itemize}
		\item Ingen
	\end{itemize}
	
	\section{Noter}
	\begin{itemize}
		\item Det er en god idé hvis at vi laver internt review på de ting vi sætter til færdige på pivotal tracker. (Én klikker finish, en anden reviewer og klikker accept)
		\item Hvis vi skal have et afsnit omkring udviklingsværktøjer skal det uddybe. Altså man skal beskrive hvad man har brugt et bestemt værktøj til, og hvorfor. (I dette afsnit kan det være godt at snakke om pivotaltracker, visual studio, eller brugen af analog discovery. Noteworthy mentions: Multisim, ultiboard, matlab)
		\item 
	\end{itemize}
\end{document}