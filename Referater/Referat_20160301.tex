\documentclass{article}

% -- PREAMBLE START --
\usepackage[utf8]{inputenc}
\usepackage[T1]{fontenc}
\usepackage{lmodern} % load a font with all the characters

\usepackage{parskip}

\usepackage[danish]{isodate}

% Create front page info
\title{Referat af Vejledermøde \#3}
\author{Gruppe 3}
\date{01.03.2016 - kl. 09:15}
% -- PREAMBLE END --

\begin{document}
	\maketitle
	
	\section{Fremmødte}
	\begin{itemize}
		\item Michael, Kasper, Mia, Mikkel, Pernille
	\end{itemize}
	
	\section{Udeblevet med afbud}
	\begin{itemize}
		\item 
	\end{itemize}
	
	\section{Udeblevet uden afbud}
	\begin{itemize}
		\item Tenna
	\end{itemize}
	
	\section{Dagsorden}
	\begin{itemize}
		\item Valg af mødeleder
		\subitem Kasper
		\item Valg af referent
		\subitem Pernille
		\item Godkendelse af referat fra forrige møde 
		\item Opfølgning på aktionspunkter fra forrige møde
		\item Gennemgang af arbejde indtil videre (use cases og accepttest specifikation)
		\begin{itemize}
			\item Pivotaltracker. Screenshots. 
			\item Struktur over hvordan afleveringen skal se ud. Dokumentationen er bilag. 
			\item Fully dressed use case. 
			\subitem Motor - visuel test. 
			\subitem Beskrive grænseflader korrekt. 
			\subitem I2C, logiske niveauer, hvorfra og hvortil forbindelsen går. Hvad ser vi hvis vi sætter et oscilloskop på. SPI. Hvor kan vi aflæse logisk høj og logisk lav. 
			\subitem BDD, atomic port. 
			\subitem Hellere for meget end for lidt ved grænseflader. 
		\end{itemize}
		\item Vejledning til de næste user stories
		\item Aktionspunkter til næste møde. Hvem gør hvad? 
		\subitem Grænseflader skal beskrives.
		\subitem Arkitektur: applikationsmodeller, BDD, IBD skal være gennemarbejdet. Grænsefladeblok for sig selv med atomare og komplekse porte i BDD. 
		\subitem Den 10. marts skal der være forbindelse igennem systemet. Koden til dette skal ikke skrives fra bunden men findes på nettet et sted. HW skal snart i gang. 
		\item Tidspunkt for næste møde: 
		\subitem Tirsdag den 8. marts kl. 8.15. 
	\end{itemize}
\end{document}