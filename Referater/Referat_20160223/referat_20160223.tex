\documentclass{article}

% -- PREAMBLE START --
\usepackage[utf8]{inputenc}
\usepackage[T1]{fontenc}
\usepackage{lmodern} % load a font with all the characters

\usepackage{parskip}

\usepackage[danish]{isodate}

% Create front page info
\title{Referat af Vejledermøde \#...}
\author{Gruppe 3}
\date{23.02.2016 - kl. 12:00}
% -- PREAMBLE END --

\begin{document}
	\maketitle
	
	\section{Fremmødte}
	\begin{itemize}
		\item Tenna
		\item Kasper 
		\item Michael 
		\item Mia 
		\item Mikkel 
		\item Pernille 
	\end{itemize}
	
	\section{Udeblevet med afbud}
	\begin{itemize}
		\item Daniel
	\end{itemize}
	
	\section{Udeblevet uden afbud}
	\begin{itemize}
		\item Ingen
	\end{itemize}
	
	\section{Dagsorden}
	\begin{itemize}
		\item Valg af mødeleder
		\subitem Kasper 
		\item Valg af referent
		\subitem Pernille
		\item Godkendelse af referat fra forrige møde 
		\subitem Godkendt. 
		\item Opfølgning på aktionspunkter fra forrige møde
		\begin{itemize}
			\item Vi har lavet opgaver og lagt dem ind i icebox på pivotaltracker. Så vi får en overordnet fremgang på pivottaltracker. 
			\item Backlog. Current er til nuværende sprint. 
			content...
		\end{itemize}
		\item Gennemgang af brug af pivotal tracker
		\item Gennemgang af foreløbig Productbacklog samt brug af user stories
		\item Valg af user stories til første sprint 
		\item Problemer i projektet 
		\subitem Integrationstest er svær. Jo før vi får hul igennem mellem de forskellige dele/blokke jo bedre. Use case der går ud på at få hul igennem, så at få delene til at virke sammen. Nunchuck direkte i Devkit måske. Første use case hedder initialiser use case. Hvad handler grænsefladen om og hvordan implementeres den? Meget simpel implementation. En driver, hvor vi kan se, at den kan læse værdier og udskrive dem på Devkit. Brugerens opgave er at tænde systemet og se at systemet virker. Devkittet skal også bekræfte, at det virker visuelt. Softwarefolk kan starte med at lave brugergrænseflade der tester. Hardwarefolk skal starte med at lave en H-bro til motor. 
		\item Første sprint 
		\subitem Det første sprint handler om at få hul igennem i systemet. 
		\item Gennemgang af tidsplan
		\subitem Vi er i gang med første sprint og har defineret opgaver til backloggen. 
		\item Aktionspunkter til næste møde. Hvem gør hvad? 
		\begin{itemize} 
			\item Use case der omhandler initialisering
			\item Beskrive grænseflader til use casen
		\end{itemize}
		\item Tidpsunkt for næste møde 
		\subitem Tirsdag den 1. marts kl. 9.15. 
		\item Evt. 
		\subitem Husk logbøger! 
	\end{itemize}
\end{document}