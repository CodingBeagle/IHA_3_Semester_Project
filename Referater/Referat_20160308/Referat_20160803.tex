\documentclass{article}

% -- PREAMBLE START --
\usepackage[utf8]{inputenc}
\usepackage[T1]{fontenc}
\usepackage{lmodern} % load a font with all the characters

\usepackage{parskip}

\usepackage[danish]{isodate}

% Create front page info
\title{Referat af Vejledermøde \#...}
\author{Gruppe 3}
\date{08.03.2016 - kl. 08:15}
% -- PREAMBLE END --

\begin{document}
	\maketitle
	
	\section{Fremmødte}
	\begin{itemize}
		\item Tenna, Daniel, Kasper, Mikkel, Mia, Michael, Pernille 
	\end{itemize}

	
	\section{Udeblevet med afbud}
	\begin{itemize}
		\item Ingen
	\end{itemize}
	
	\section{Udeblevet uden afbud}
	\begin{itemize}
		\item Ingen 
	\end{itemize}
	
	\section{Dagsorden}
	\begin{itemize}
			\item Valg af mødeleder
			\subitem Daniel
			\item Valg af referent
			Pernille
			\item Godkendelse af referat fra forrige møde 
			Godkendt.
			\item Opfølgning på aktionspunkter fra forrige møde
				\begin{itemize}
					\item Der er blevet lavet en applikationsmodel for UC2. 
					\item IBD og BDD, signalbeskrivelser
					\item implementering af SW
					\subitem I2C op at køre mellem to PSoC'er. 
					\subitem Find ud af hvad stikket til nunchuck er for et. Så er det lettere at finde det. 
				\end{itemize}
				
			\item Hjælp til logiske niveaur til signalbeskrivelse
				\begin{itemize}
					\item Signaltype high og low - CMOS? Definer hvad dødområde, høj og lav er for hvert signal. 
					\item 
				\end{itemize}
			
			\item Opdatering af diagrammer
			\subitem Vi skal opdatere diagrammer, når vi finder nye detaljer. Lav en ny version af diagrammet. 
			\subitem Overordnet flow i sekvensdiagrammet - hvad regner jeg med der sker, hvad regner jeg med der skal ske i koden. Fokuser mere på grænseflader i denne del af processen. 
				
			\item Gennemgang af tidsplan
				\begin{itemize}
					\item 
				\end{itemize}
				
			\item Nye aktionspunkter til næste møde
				\begin{itemize}
					\item Vi går til forelæsning om SPI på næste tirsdag. 
					\item Vi skal lave review og retrospective på et tidspunkt. 
					\item Vi arbejder videre med implementering 
					\item SPI kan først laves når vi har været til forelæsning på tirsdag
					\item Skrive ned inden retrospective, hvorfor vi ikke har nået det hele. 
				\end{itemize}
			\item Tidspunkt for næste møde
			\subitem Review: Her skal Gunvor se alt hvad vi har lavet i dette sprint. 
			\subitem Retrospective: Her vurderer vi hvordan sprintet har været. 
			\subitem Tidspunkt: Fredag den 11. marts kl. 12.15. 
			
			\item Evt. 
			
	\end{itemize}
\end{document}