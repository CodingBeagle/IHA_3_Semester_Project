\documentclass{article}

% -- PREAMBLE START --
\usepackage[utf8]{inputenc}
\usepackage[T1]{fontenc}
\usepackage{lmodern} % load a font with all the characters

\usepackage{parskip}

\usepackage[danish]{isodate}

% Create front page info
\title{Referat af Vejledermøde \#2}
\author{Gruppe 3}
\date{19.02.2016 - kl. 12:15}
% -- PREAMBLE END --

\begin{document}
	\maketitle
	
	\section{Fremmødte}
	\begin{itemize}
		\item Tenna
		\item Mikkel 
		\item Kasper
		\item Mia 
		\item Pernille
	\end{itemize}
	
	\section{Udeblevet med afbud}
	\begin{itemize}
		\item Daniel
		\item Michael
	\end{itemize}
	
	\section{Udeblevet uden afbud}
	\begin{itemize}
		\item Ingen
	\end{itemize}
	
	\section{Dagsorden}
	\begin{itemize}
	\item Valg af mødeleder
		\begin{itemize} 
			\item Kasper
		\end{itemize}
	\item Valg af referent
		\begin{itemize}
			\item Pernille 
		\end{itemize}
	\item Godkendelse af referat fra forrige møde 
		\subitem Godkendt 
	\item Opfølgning på aktionspunkter fra forrige møde
		\subitem Projektformulering
		\begin{itemize}
			\item Baggrund for valg. Argumenter for hvorfor vi vælger at lave en slikkanon. Referencer til om der er noget lignende på markedet. Avisartikler. Vi skal ikke lave ngoet der er lavet før. 
			\subitem Hvis der er nogen der har lavet det før så henvis til dette. 
			\item Man får en god fornemmelse af hvad det skal kunne. Skriv at det skal skrive på en embedded linuxplatform. Præmis for projektet. 
			\item Gerne alle præmisserne for projektet her. 
			\item Fint rigt billede. Scanningsapps - vektorgrafik. 
			\item Ret mange opgaver til HW i forhold til ikke så mange opgaver til SW. 
			\subitem Vild brugergrænseflade. 
			\subitem Små opgaver med godt defineret hvad SW-folk skal. 
			\subitem IKT'er på noget der nærmer sig HW. 
			\subitem Sørg for at opgaver ikke deles så hårdt op i HW og SW. 
			\item Begrund hvorfor det er et spil til to personer. 
		\end{itemize}
	\item Skal vi bruge \textit{BÅDE} motor OG sensorer?
		\begin{itemize}
			\item Det skal vi! 
			\item Kamera der kun er håndteret af IKT'er. Find målskiven med kamera. 
			\item motorstyring, PSoC med SPI og I2C. 
			\item snak med Rasmus fra mekanisk værksted om motor. 
		\end{itemize}
	\item Tjek af iterationstabel
		\begin{itemize}
			\item Bekskrivelse for hvert sprint
			\item Burn down chart 
			\item brug ikke for lang tid på at sætte timer på opgaver 
			\subitem brug evt. t-shirt-størrelser (S, M, L, XL) 
		\end{itemize}
	\item Tjek af sprint tidsplan
		\begin{itemize}
			\item I det sidste sprint fokuseres på rapportskrivning. 
		\end{itemize}
	\item Opbygning af fælles og individuel logbog
		\begin{itemize}
			\item Fælles logbog i starten inden sprintstart. 
			\item Husk at skrive individuel logbog hver eneste dag. Hvis man har læst referat fra sidste møde og det er det eneste man har lavet er det det, man skriver. 
			\item hvis vi arbejder i weekenden, så lav logbog. Hvis ikke så lad være. Det kan være en god ide at gøre det, så man holder rytmen. 
		\end{itemize}
	\item Scrum-master 
		\begin{itemize}
			\item Slå op, hvad en scrum-master er, så man kan svare på til eksamen hvad det er. 
			\item Kasper starter med at være scrummaster. 
			\item Skriv ind i logbogen, hvordan man vil være scrummaster i projektet. 
		\end{itemize}
	\item Gennemgang af tidsplan 
		\begin{itemize}
			\item Den er fin. 
		\end{itemize}
	\item Nye aktionspunkter til næste møde. Hvem gør hvad? 
		\begin{itemize}
			\item Liste med product backlog. 
			\item Bestem ting vi vil lave i dette sprint. 
			\item Få redmine til at virke. 
			\item Send liste med navne, studienummer, mail-adresse til Gunvor, så vi kan få licenser til scrum-værktøj. 
		\end{itemize}
	\item Tidspunkt for næste møde 
		\begin{itemize}
			\item Tirsdag d. 23. februar kl. 8.15. 
			\item Vi arbejder mandag med at finde opgaver til product backlog. 
			\subitem Mikkel og Pernille finder HW-opgaver. 
			\subitem Tenna, Mia, Daniel, Kasper, Michael finder SW-opgaver. 
		\end{itemize}
	\item Evt. 
		\begin{itemize}
			\item 
		\end{itemize}
	\end{itemize}
\end{document}