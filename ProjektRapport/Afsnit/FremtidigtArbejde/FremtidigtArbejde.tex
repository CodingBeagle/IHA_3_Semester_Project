\chapter{Fremtidigt arbejde}

Et problem ved produktet er, at affyringsmekanismen er meget tung i begge ender, så hvis der drejes for langt ud med Wii-nunchuck vipper den enten frem eller tilbage. Dette ville kunne have været afhjulpet med en H-bro. Denne ville nemlig så kunne have bremset affyringsmekanismen ved at være blevet sat til at køre i begge retninger i den tid, den skal holde stille. Hvis denne H-bro skulle medtages i systemet skulle der også skrives noget software, hvor H-broen blev sat til at køre i begge retninger på samme tid. 

Noget andet der kan arbejdes videre med, er noget i samme stil som med bremsefunktionen af affyringsmekanismen. Problemet er, at affyringsmekanismens motor ikke kører særligt stærkt. Det skyldes, at rotationsdetektoren ikke kunne nå at stoppe motoren, inden den havde kørt en omgang mere. For at afhjælpe det problem blev motorens dutycycle sat ned, men det gjorde, at affyringsmekanismen nu ikke længere kunne skyde særligt hårdt. Derfor ville det være en god ide, at bruge en H-bro til også at styre denne motor, så den kunne få en bremsefunktion, ligesom den vertikale drejefunktions. 