\chapter{Fremtidigt arbejde}

%Et problem ved produktet er, at affyringsmekanismen er meget tung i begge ender, så hvis der drejes for langt ud med Wii-nunchuck vipper den enten frem eller tilbage. Dette ville kunne have været afhjulpet med en H-bro. Denne ville nemlig så kunne have bremset affyringsmekanismen ved at være blevet sat til at køre i begge retninger i den tid, den skal holde stille. Hvis denne H-bro skulle medtages i systemet skulle der også skrives noget software, hvor H-broen blev sat til at køre i begge retninger på samme tid. 

%Noget andet der kan arbejdes videre med, er noget i samme stil som med bremsefunktionen af affyringsmekanismen. Problemet er, at affyringsmekanismens motor ikke kører særligt stærkt. Det skyldes, at rotationsdetektoren ikke kunne nå at stoppe motoren, inden den havde kørt en omgang mere. For at afhjælpe det problem blev motorens dutycycle sat ned, men det gjorde, at affyringsmekanismen nu ikke længere kunne skyde særligt hårdt. Derfor ville det være en god ide, at bruge en H-bro til også at styre denne motor, så den kunne få en bremsefunktion, ligesom den vertikale drejefunktions.


Eftersom motoren ikke er kraftig nok til at holde affyringsmekanismen stabilt i den vertikale akse, er dette noget der skal arbejdes på i fremtiden. Når kanonen vippes for langt frem eller tilbage, kan motoren ikke længere flytte affyringsmekanismen, og brugeren er nødt til at sætte kanonen tilbage til en start position. For at løse dette problem, kunne man geare motoren tilstrækkeligt, så den får en større trækkraft, hvilket burde gøre den i stand til at løfte den tunge affyringsmekanisme. \newline

\noindent Affyringsmekanismen på produktet, skubber nogle gange flere projektiler afsted, efter blot ét tryk på nunchucken. Dette skyldes at motoren der driver affyringsmekanismen stadig har momentum idét PWM signalet der driver motoren stoppes. For at afvikle dette problem, kunne der indsættes en H-bro før motoren, idét at denne, kan give motoren en "bremse" funktionalitet, der kan låse motorens rotation, og derved sørge for at kun et enkelt projektil bliver affyret. \newline 

\noindent I systemarkitekturen indeholde use case diagrammet en enkelt aktør, nemig brugeren. Idét at use case 2 omhandler en systemtest, som den almene bruger ikke vil gøre brug af, kunne det være fordelagtigt at tilføje en \textit{Admin} aktør der håndterer systemtesten. Derved interagerer brugeren kun med use case 1 - Spil Goofy Candygun 3000. \newline

\noindent Systemets brugergrænseflade er ikke færdig implementeret, idét at der ikke gives nogen point til spilleren, for at skyde med kanonen, og derved kan der ikke føres statistik og highscores for spillerne. For at implementere denne funktionalitet, kræves det at der laves en form for målskive, der kan registrere skud fra kanonen, og at denne information bliver sendt til brugergrænsefladen. Dette kunne evt. gøres ved at sende informationen igennem PSoC0, og så videre til Devkittet via SPI-forbindelsen. \newline

\noindent Til produktudviklingen er der blevet gjort brug af Devkit 8000 og PSoC udviklingsboard, som har ubrugte funktionaliteter. Skulle produktet produceres til salg, ville være uhensigtsmæssigt at gøre brug af udviklingsboards. I det endelige produkt skal komponenterne vælges ud fra de krav produktet har til funktionaliteter, så der ikke spildes penge på unødvendige funktionaliteter.