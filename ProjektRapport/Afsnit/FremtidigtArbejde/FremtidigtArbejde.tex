\chapter{Fremtidigt arbejde}


Noget, der kunne arbejdes videre på, er stabilisering af affyringsmekanismens vertikale bevægelighed, da kanonen er forholdsvis tung i begge ender. En ændring af H-broens opbygning, så spændingen over de to positive eller de to negative indgange kunne kortsluttes, ville medføre en mulighed for at motorbremse. Dermed ville den vertikale position af kanonen kunne låses under brug. En anden metode til at opnå bedre stabilitet, kunne være at geare mekanikken, så der opnås større trækkraft og dermed en langsommere, men mere stabil bevægelse.\newline

\noindent Som motorstyringen for affyringsmekanismen er opbygget, er der ikke mulighed for at motorbremse, og da affyringsmekanismen er gearet til at affyre med 5 gange motorens hastighed, har det været nødvendigt at sætte dutycyclen langt ned, for at der ikke skydes flere gange i træk, da motoren ved højere hastighed ikke når at stoppe, inden der igen skydes. Som resultat af den lave dutycycle, skyder kanonen ikke så kraftigt. En måde at opnå bremsefunktionalitet til motoren, så dutycyclen kunne sættes op og kanonen skyde kraftigere, ville være at ombygge motorstyringen, så den bestod af en H-bro med bremsefunktionalitet. \newline

\noindent Systemets brugergrænseflade kunne opdateres med mulighed for at vise pointtavler, idét at der ikke gives nogen point til spilleren, for at skyde med kanonen, og derved kan der ikke føres statistik og highscores for spillerne. For at implementere denne funktionalitet, kræves det, at der laves en form for målskive, der kan registrere skud fra kanonen, og at denne information bliver sendt til brugergrænsefladen. Dette kunne evt. gøres ved at sende informationen igennem PSoC0, og så videre til Devkittet via SPI-forbindelsen. \newline

 