\chapter{Fremtidigt arbejde}

Noget, der kunne arbejdes videre på, er stabilisering af affyringsmekanismens vertikale bevægelighed, da kanonen er forholdsvis tung i begge ender. En ændring af H-broens opbygning, så spændingen over de to positive eller de to negative indgange kunne kortsluttes, ville medføre en mulighed for at motorbremse. Dermed ville den vertikale position af kanonen kunne låses under brug. En anden metode til at opnå bedre stabilitet, kunne være at geare mekanikken, så der opnås større trækkraft.\newline

\noindent Affyringsmekanismen på produktet, skubber nogle gange flere projektiler afsted, efter blot ét tryk på nunchucken. Dette skyldes at motoren der driver affyringsmekanismen stadig har momentum idét PWM signalet der driver motoren stoppes. For at afvikle dette problem, kunne der indsættes en H-bro før motoren, idét at denne, kan give motoren en "bremse" funktionalitet, der kan låse motorens rotation, og derved sørge for at kun et enkelt projektil bliver affyret. \newline 

\noindent I systemarkitekturen indeholde use case diagrammet en enkelt aktør, nemlig brugeren. Idét at use case 2 omhandler en systemtest, som den almene bruger ikke vil gøre brug af, kunne det være fordelagtigt at tilføje en \textit{Admin} aktør der håndterer systemtesten. Derved interagerer brugeren kun med use case 1 - Spil Goofy Candygun 3000. \newline

\noindent Systemets brugergrænseflade kunne opdateres med mulighed for at vise pointtavler, idét at der ikke gives nogen point til spilleren, for at skyde med kanonen, og derved kan der ikke føres statistik og highscores for spillerne. For at implementere denne funktionalitet, kræves det, at der laves en form for målskive, der kan registrere skud fra kanonen, og at denne information bliver sendt til brugergrænsefladen. Dette kunne evt. gøres ved at sende informationen igennem PSoC0, og så videre til Devkittet via SPI-forbindelsen. \newline

\noindent Til produktudviklingen er der blevet gjort brug af Devkit 8000 og PSoC udviklingsboard, som har ubrugte funktionaliteter. Skulle produktet produceres til salg, ville være uhensigtsmæssigt at gøre brug af udviklingsboards. I det endelige produkt skal komponenterne vælges ud fra de krav produktet har til funktionaliteter, så der ikke spildes penge på unødvendige funktionaliteter.