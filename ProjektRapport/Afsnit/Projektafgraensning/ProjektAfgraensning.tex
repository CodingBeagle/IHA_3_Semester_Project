\chapter{Projektafgrænsning}

Ud fra MoSCoW-princippet er der udarbejdet en række krav efter prioriteringerne ’must have’, ’should have’, ’could have’ og ’won’t have’. Dette er for at gøre det tydeligt, hvad der er vigtigt, der bliver udviklet først, og hvad der godt kan vente til senere. Disse krav er som følger:
\begin{itemize}
	\item Produktet must have:
	\begin{itemize}
		\item En motor til styring af kanonen
		\item En grafisk brugergrænseflade
		\item En Wii-nunchuck til styring af motoren
		\item En kanon med en affyringsmekanisme
		\item En system test til diagnosering af fejl
	\end{itemize}
	\item Produktet should have:
	\begin{itemize}
		\item Et mål til registering af point
		\item En lokal ranglistestatistik
	\end{itemize}
	\item Produktet could have:
	\begin{itemize}
		\item En partymode-indstilling til over to spillere
		\item Trådløs Wii-nunchuckstyring
		\item Afspilning af lydeffekter
	\end{itemize}
	\item Produktet won’t have:
	\begin{itemize}
		\item Et batteri til brug uden strømforsyning
		\item Online ranglistestatistik
	\end{itemize}
\end{itemize}

"Must Have" kravende har højst prioritering i projektet. Det vil altså sige, at kravene under punkterne ’should have’ og ’could have’ har lavere prioritet. For at kravene under punktet 'must have' er opfyldt, skal use case 2, \textit{Test Kommunikationsprotokoller} implementeres. Derfor blev prioriteringen i dette projekt, at use case 2 skulle implementeres til fulde, inden der kunne startes på at implementere use case 1. Det havde dog også høj prioritet at have et produkt, der fysisk kunne styres, samt skyde, hvilket betød, at selvom affyringsmekanismen ikke indgår i use case 2, blev det alligevel prioriteret højt at få implementeret denne i systemet.






