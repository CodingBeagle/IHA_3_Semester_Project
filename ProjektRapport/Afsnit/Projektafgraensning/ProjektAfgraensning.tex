\chapter{Projektafgrænsning}
\label{afsnit:Projektafgraensning}
Ud fra MoSCoW-princippet \cite{moscow} er der udarbejdet en række krav efter prioriteringerne \textit{must have}, \textit{should have}, \textit{could have} og \textit{won’t have}. Dette tydeliggører, hvilke dele af systemet der er højest prioriteret. Disse krav er som følger:

\begin{itemize}
	\item Produktet must have:
	\begin{itemize}
		\item En motor til styring af kanonen
		\item En grafisk brugergrænseflade
		\item En Wii-nunchuck til styring af motoren
		\item En kanon med en affyringsmekanisme
		\item En system test til detektering af fejl i kommunikationsprotokoller
	\end{itemize}
	\item Produktet should have:
	\begin{itemize}
		\item Et mål til registering af point
		\item En lokal ranglistestatistik
	\end{itemize}
	\item Produktet could have:
	\begin{itemize}
		\item En multiplayer-indstilling til over to spillere
		\item Trådløs Wii-nunchuckstyring
		\item Afspilning af lydeffekter
	\end{itemize}
	\item Produktet won’t have:
	\begin{itemize}
		\item Et batteri til brug uden strømforsyning
		\item Online ranglistestatistik
	\end{itemize}
\end{itemize}

\noindent \textit{Must have} kravene har den højeste prioritering i projektet. Det vil altså sige, at kravene under punkterne \textit{should have} og \textit{could have} har lavere prioritet. For at kravene under punktet \textit{must have} er opfyldt, skal hardware for use case 1 samt hele use case 2 implementeres. Grunden til dette er, den grundlæggende kommunikation mellem udviklingsboardene er en essentiel del, idét kommunikationen er grundlaget for begge use cases.
