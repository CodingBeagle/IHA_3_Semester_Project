\chapter{Metode}
I arbejdet med projektet er det vigtigt at anvende gode analyse- og designmetoder. Dermed er det muligt at komme fra den indledende idé til det endelige produkt med lavere risiko for misforståelser og kommunikationsfejl undervejs. Det er også en stor fordel, hvis de metoder, der anvendes, er intuitive og har nogle fastlagte standarter. Det gør det muligt for udenforstående at sætte sig ind i, hvordan projektet er udviklet og designet. Dermed bliver projektet og dets produkt i højere grad uafhængigt af enkeltpersoner, og det bliver muligt at genskabe produktet.

\section{SysML}
Til dette projekt er der anvendt \textit{SysML} som visuelt modelleringsværktøj, til at skabe diagrammer. SysML blev valgt, da det er en anerkendt industristandard\cite{website:SysML} , hvilket betyder at det er mere universalt brugbart. Specifikationen for SysML kan findes i bilag, \textit{SysML Specification.pdf}.

SysML er et modelleringssprog, der bygger videre på det meget udbredte modelleringssprog, UML. Men hvor UML hovedsagligt er udviklet til brug i objekt orienteret software udvikling, er SysML i højere grad udviklet med fokus på beskrivelse af både software- og hardwaresystemer. Det gør det særdeles velegnet til dette projekt, som netop består af både software- og hardwaredele. Det er derfor også i store dele af arbejdet med analyse og design af produktet.

SysML specifikationen er omfattende og beskriver mange diagramtyper. Til dette projekt er der valgt diagramtyper alt efter deres nytteværdi for modellering af hardware og software, som vil summeres her.

Til modellering af hardware er der i rapport og dokumentation gjort brug af strukturdiagrammerne \textit{Block Definition Diagrams} (BDD) samt \textit{Internal Block Diagrams} (IDB). Disse diagrammer er brugt til at beskrive systemets hardwarekomponenter, deres signaler, og forbindelserne mellem dem.

Til modellering af software er der i rapport og dokumentation gjort brug af struktur- og adfærdsdiagrammerne \textit{Block Definition Diagrams}, \textit{Sequence Diagrams} (SD), og \textit{State Machine Diagrams} (SMD). Disse diagrammer er brugt til at beskrive systemets softwarekomponenter i form af klasser, relationer mellem klasserne, samt hvordan disse klasser interagerer med hinanden og hvilket tilstande de kan være i.

\subsection{Afvigelser fra standard brug}
I SysML sekvensdiagrammer bruges beskedudveksling typisk til at repræsentere metoder på de objekter der kommunikeres med. Denne fremgangsmåde bruges i denne raport, dog er der nogle sekvensdiagrammer der afviger ved at beskedudvekslingen repræsenterer handlinger påført på objekter. Et eksempel på denne afvigelse kan ses i afsnit \ref{afsnit:informationFlow}, figur \ref{fig:WiiNunchuckSekvensDiagram}. Denne afvigelse blev brugt for at kunne tydeliggøre interaktionen mellem systemets komponenter på en naturlig måde.