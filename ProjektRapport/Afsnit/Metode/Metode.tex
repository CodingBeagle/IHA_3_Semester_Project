\chapter{Metode}
I arbejdet med projektet er det vigtigt at anvende gode analyse- og designmetoder. Dermed er det muligt at komme fra den indledende idé til det endelige produkt med lavere risiko for misforståelser og kommunikationsfejl undervejs. Det er også en stor fordel, hvis de metoder, der anvendes, er intuitive og har nogle fastlagte standarter. Det gør det muligt for udenforstående at sætte sig ind i, hvordan projektet er udviklet og designet. Dermed bliver projektet og dets produkt i højere grad uafhængigt af enkeltpersoner, og det bliver muligt at genskabe produktet.\\ 
I forbindelse med udviklingen af Goofy Candygun 3000 er der anvendt use cases til at beskrive brugen af produktet. Gennem analyse af use cases er det muligt at skabe overblik over de nødvendige funktioner, som produktet skal have for at opfylde projektformuleringen. Som modellerings sprog til diagrammer er der anvendt UML og SysML, som netop har faslagte standarter, der gør de forskellige diagrammer lettere at sætte sig ind i, og som dermed bidrager til at give en god forståelse for opbygningen og designet af produktet.


\section{SysML}

\section{Software Design Principper}

**Cohesion, coupling**

\section{Designmetode}

**Afvigelser fra traditionelt vandfaldsmodel**