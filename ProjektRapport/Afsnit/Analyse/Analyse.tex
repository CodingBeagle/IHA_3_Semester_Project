\chapter{Analyse}
\label{afsnit:analyse}
Til projektets prototype er der brugt nogle grundlæggende hardwarekomponenter til realisering af systemets arkitektur. Disse hardwarekomponenter vil blive præsenteret følgende.

\section{DevKit8000}
DevKit8000 er en indlejret linux platform med et tilkoblet 4.3 tommer touch-display. Denne indlejrede linux platform blev valgt, da den allerede fra start understøtter interfacing med et touch-display. Dette kan bruges til systemets brugergrænseflade. DevKit8000 understøtter desuden de serielle kommunikationsbusser SPI og I2C, hvilket er typiske busser der bliver brugt til kommunikation med sensorer samt aktuatorer. \newline 
\noindent DevKit8000 platformen er også brugt gennem undervisning på IHA, hvilket betyder at der er god adgang til de compilers der skal bruges til platformen.

\section{Programmable System-on-Chip (PSoC)}
PSoC er en microcontroller der kan omprogrammeres via et medfølgende \textit{Integrated Development Environment} (IDE). PSoC'en understøtter multiple \textit{Seriel Communication Busses} (SCB), hvilket gør denne microcontroller ideel til dette system, da der skal kommunikeres med sensorer og aktuatorer.

\section{Wii-Nunchuck}
Til brugerstyring af systemets kanon er en Wii-Nunchuck controller valgt. Denne controller er valgt, da den har gode egenskaber til styring af en kanon. Wii-Nunchucken understøtter desuden en af PSoC'ens kommunikationsprotokoller, så den nemt kan kommunikere med resten af systemet.

\section{Motor}
Der blev testet og stillet krav for hvad motoren skulle kunne for at den kunne bruges i dette projekt. 
Kravene var som følger:
	\begin{itemize}
		\item  Skulle have en lave omdrejnings hastighed 
\item	Skulle være en DC motor
\item	Den skulle bruge over 8V, for at h-broen ville kunne funger med motoren (for ifølge multisim kunne den ikke klar en spænding under 8V)
\item	Skulle gerne kunne trække noget
\end{itemize}
Efter nogen test og læsning i datablade, blev der fundet frem til en motor, EV3 Large motor, som overholdte alle kravene som den var blevet stillet for. Motoren bruge en spænding på 9V og den har en omdrejnings hastighed på 175rpm, læs mere om den i databladet\#ref datablad, hvilket gjord den drejet pænt rundt. 

