\chapter{Resumé}
Denne rapport beskriver semesterprojektet for tredjesemesterstuderende, hvor en prototype for spillet Goofy Candygun 3000 er slut produktet. Ideen bag Goofy Candygun 3000 er et spilsystem hvor en eller to spillere skyder en slikkanon efter et mål for at opnå det største antal point. Brugeren interagerer med systemet gennem en touch skærm og en Wii-Nunchuck. Til projektet har IHA stillet nogle krav til hvilke elementer systemet skal indeholde (Bilag/Projektrapport/Semesterprojekt3Oplæg). Dette inkluderer en indlejret linux platform, en sensor, en aktuator og en PSoC platform. \newline

\noindent Følgende denne rapport er en procesrapport der beskriver gruppens brug af udviklingsmodellen scrum\cite{scrum}. Derudover beskriver procesrapporten udviklingsforløbet af projektet samt de værktøjer der er blevet anvendt til at fremme processen. \newline

\noindent Produktets prototype består af et netværk af PSoC4 udviklingsboards og en Wii-Nunchuck, som kommunikerer via I2C kommunikationsprotokollen. Brugeren interagerer med systemet gennem en grafisk brugergrænseflade på Devkit 8000's touch skærm. Via Wii-Nunchuck kan brugeren kontrollere kanonens sigte, samt affyre skud. \newline

\noindent Ikke alle dele af produktet er fuldt implementeret. Implementationen af hoved use casen, som omhandler selve spillet, er påbegyndt, men ikke færdigimplementeret. Gennem prioritering med MoSCoW princippet \cite{moscow} er kommunikationen mellem linux og PSoC platformene blevet prioriteret højest og dermed er use casen omhandlende systemtesten blevet færdigimplementeret.

\chapter{Abstract}
This paper describes the third semester project, in which a prototype for Goofy Candygun 3000 is the final product. The vision behind Goofy Candygun 3000 is game system where one or two players aim a candycannon at a target to score the highest amount of points. The system is controlled by the user through a touch screen and a Wii-Nunchuck. For this project IHA has required that the system includes an embedded linux platform, a sensor, an actuator and a PSoC platform. \newline

\noindent Mainly, the paper describes the design and implementation of a prototype for Goofy Candygun 3000. Following this paper, a report about the usage of the agile development framework, scrum, and the group's workprocess throughout the project is described. \newline 

\noindent The prototype consists of a network PSoC4 developmentboards, which communicate amongst each other by the I2C-bus. The user interacts with the system through a graphical userinterface on the Devkit 8000's touch screen. Using the Wii-Nunchuck the user controls cannon's aim, and fires the cannon. \newline

\noindent The prototype has not been fully implemented. The implementation of the main use case, which addresses gameplay functionalities, has not been fully implemented. Using the MoSCoW method, communication amongst the linux and PSoC platforms has been prioritized highly and therefore the use case addressing the systemtest has been implemented in full.\newline