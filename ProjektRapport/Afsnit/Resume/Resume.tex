\chapter{Resumé}
Denne rapport beskriver semesterprojektet for tredjesemesterstuderende, hvor en prototype for spillet Goofy Candygun 3000 er slut produktet. Produktet er et spilsystem, hvor en eller to spillere skyder efter et mål med en slik kanon for at få point.\newline

\noindent Produktets prototype består af et netværk af PSoC4 udviklingsboards og en Wii-Nunchuck, som kommunikerer via I2C kommunikationsprotokollen. Brugeren interagerer med systemet gennem en grafisk brugergrænseflade på Devkit 8000's touch skærm. Via Wii-Nunchuck kan brugeren kontrollere kanonens sigte, samt affyre skud. \newline

\noindent I løbet af projektet er der blevet implementereret et system, hvor brugeren, ved hjælp af en Nunchuck controller, kan styre en kanon og affyre et projektil. Nunchucken kommunikerer med et PSoC udviklingsboard, der behandler og videresender dataen til Devkit 8000 via SPI, og til en anden PSoC enhed, via I2C. Ud fra denne data, er PSoC netværket i stand til at styre og affyre slik kanonen. På Devkit 8000, er der blevet lavet en illustrativ brugergrænseflade, der viser produktets ønskede funktionaliteter. Brugergrænsefladen kan også starte en systemtest, der tester systemets kommunikationsprotokoller.\newline
 
\noindent Ved brug af MoSCoW prioritering, er systemets funktionaliteter blev prioriteret, og kun de vigtigste funktionaliteter er blevet implementeret i den endelige prototype. \newline

\chapter{Abstract}
This paper describes the third semester project, in which a prototype for Goofy Candygun 3000 is the final product. The product is a game, where the player(s) fires a candy cannon at a target to score points.  \newline

\noindent The prototype consists of a network of PSoC4 developmentboards, which communicate amongst each other by the I2C-bus. The user interacts with the system through a graphical userinterface on the Devkit 8000's touch screen. Using the Wii-Nunchuck the user controls cannon's aim, and fires the cannon. \newline

\noindent Over the course of the project, a system where the user, using a Nunchuck controller, can aim and fire a projectile from a candy cannon. The nunchuck communicates with a PSoC developmentboard, which processes and forwards the data to Devkit 8000 using SPI, and to a PSoC developmentboard using I2C. Using this data, the PSoC network is able to control and fire the candy cannon. On Devkit 8000, a demonstrative user interface has been implemented. This userinterface shows the wanted features of the product. From the userinterface the user can run a systemtest, which tests the communication protocols, used by the system.

\noindent MoSCoW has been used to prioritise the system functionalities, and only the most important ones has been implemented.