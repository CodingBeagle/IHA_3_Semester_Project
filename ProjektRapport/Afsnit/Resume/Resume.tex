\chapter{Resumé}
\chapter{Abstract}
%Abstract er sendt til Gunvor for gennemlæsning og vurdering af længden og hvilke dele, der evt. kan udelades.
This project aims to develop a candygun ment for use at parties and at leisure time. The finished product is to be controlled with a touchscreen for initiation of the system and a wii-nunchuck for adjusting and firing the gun. Furthermore, as demanded by predefined requirements,  an embedded linux platform is included. For this purpose the Devkit 8000 is chosen. Further predefined requirements include use of sensors and actuators and well as the use of a PSoC is required - here a PSoC4 is chosen. \\
Through the analysis and specification of use cases the functional demands of the system are determined, then prioritized by the use of the MoSCoW-method. In the development of the project the agile development framework, Scrum, is chosen. By working incremental and iterative in the course of sprints the development has been well structured and reflected upon. The architecture of the system is described by use of the SysML. Through analysis of the communication between sub-systems protocols for I2C- and SPI has been developed. \\ 
The end-result of this project is a prototype whose hardware include motor control of three motors and three detectors, one which makes use of a potentiometer to limit horizontal rotation of the gun, another using a photodiode for an optical detection in the firing of the gun. The last, a Wii-nunchuck reacting to user input. The hardware is closely connected to mechanic parts, who enable rotation and firing of the gun. As far as embedded software goes a big part is the implementation of the two communication systems, SPI and I2C. An eventbased graphical user interface makes up the touchscreen functionalities. And the programming of 3 PSoC4s connect the parts to complete the system. \\
All parts meet the requirements to test the communication protocols of the system. Playing of the game with shooting the gun meets the requirements to adjust and fire the gun, but is decoupled from the touchscreen initiation of the game, though an illustrational GUI for the game has been implemented. As a results requirements for the acceptance test, with few exceptions, has been met.
