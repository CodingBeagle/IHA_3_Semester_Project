\chapter{Resumé}
Denne rapport beskriver udviklingsprocessen bag semesterprojektet for tredjesemesterstuderende, hvor en prototype for Goofy Candygun 3000 er slut produktet. Ideen bag Goofy Candygun 3000 er et spilsystem hvor en eller to spillere sigter en slikkanon efter et mål for at opnå det største antal point. Brugeren interagerer med systemet gennem en touch skærm og en Wii-Nunchuck. Til projektet har IHA stillet nogle krav til hvilke elementer systemet skal indeholde. Dette inkluderer en indlejret linux platform, en sensor, en aktuator og en PSoC platform. \newline

\noindent Følgende denne rapport er en procesrapport der beskriver udviklingsmodellen, scrum, som anvendes under dette projektforløb. Derudover beskriver rapporten udviklingsforløbet af projektet samt de værktøjer der er blevet anvendt til at fremme processen. \newline

\noindent Prototypen består af et netværk af PSoC4 udviklingsboards, som kommunikerer via I2C kommunikationsprotokollen. Brugeren interagerer med systemet gennem en grafisk brugergrænseflade på Devkit 8000's touch skærm. Via Wii-Nunchuck kan brugeren kontrollere kanonens position samt afvikle skud. \newline

\noindent Grundet tidsbegrænsinger er dele af prototypen ikke blevet implementeret. Implementationen af hoved use casen, som omhandler selve spillet, er påbegyndt, dog er vigtige dele endnu ikke færdigimplementeret. Gennem prioritering med MoSCoW princippet er kommunikationen mellem linux og PSoC platformene blevet prioriteret højest og dermed er use casen omkring systemtesten blevet færdigimplementeret.

\chapter{Abstract}
%Abstract er sendt til Gunvor for gennemlæsning og vurdering af længden og hvilke dele, der evt. kan udelades.
%This project aims to develop a candygun ment for use at parties and at leisure time. The finished product is to be controlled with a touchscreen for initiation of the system and a wii-nunchuck for adjusting and firing the gun. Furthermore, as demanded by predefined requirements,  an embedded linux platform is included. For this purpose the Devkit 8000 is chosen. Further predefined requirements include use of sensors and actuators and well as the use of a PSoC is required - here a PSoC4 is chosen. \\
%Through the analysis and specification of use cases the functional demands of the system are determined, then prioritized by the use of the MoSCoW-method. In the development of the project the agile development framework, Scrum, is chosen. By working incremental and iterative in the course of sprints the development has been well structured and reflected upon. The architecture of the system is described by use of the SysML. Through analysis of the communication between sub-systems protocols for I2C- and SPI has been developed. \\ 
%The end-result of this project is a prototype whose hardware include motor control of three motors and three detectors, one which makes use of a potentiometer to limit horizontal rotation of the gun, another using a photodiode for an optical detection in the firing of the gun. The last, a Wii-nunchuck reacting to user input. The hardware is closely connected to mechanic parts, who enable rotation and firing of the gun. As far as embedded software goes a big part is the implementation of the two communication systems, SPI and I2C. An eventbased graphical user interface makes up the touchscreen functionalities. And the programming of 3 PSoC4s connect the parts to complete the system. \\
%All parts meet the requirements to test the communication protocols of the system. Playing of the game with shooting the gun meets the requirements to adjust and fire the gun, but is decoupled from the touchscreen initiation of the game, though an illustrational GUI for the game has been implemented. As a results requirements for the acceptance test, with few exceptions, has been met. \newline


This paper describes the development process behind the third semester project, in which a prototype for Goofy Candygun 3000 is the final product. The vision behind Goofy Candygun 3000 is game system where one or two players aim a candycannon at a target to score the highest amount of points. The system is controlled by the user through a touch screen and a Wii-Nunchuck. For this project IHA has required that the system includes an embedded linux platform, a sensor, an actuator and a PSoC platform. \newline

\noindent Mainly, the paper describes the design and implementation of a prototype for Goofy Candygun 3000. Following this paper, a report about the agile development framework, scrum, and the group's workprocess throughout the project cycle is described. \newline 

\noindent The prototype consists of a network PSoC4's, which communicate amongst each other via the I2C communication protocol. The user interacts with the system through a graphical interface on the Devkit 8000's touch screen. Via the Wii-Nunchuck the user controls cannon's position and launches the cannon. \newline

\noindent Due to time constraints parts of the prototype have not been implemented fully. The implementation of the main use case, which addresses game play functionalities, has been started although vital parts of the use case have not been implemented. Using the MoSCoW method, communication amongst the linux and PSoC platforms has been prioritized highly and therefore the use case addressing the systemtest has been implemented fully.    

