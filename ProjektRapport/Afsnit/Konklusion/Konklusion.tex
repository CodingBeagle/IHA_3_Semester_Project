\chapter{Konklusion}
Formålet med dette projektforløb var at udvikle \textit{Goofy Candy Gun 3000} spillet, hvor 1 til 2 spillere dyster om at ramme et mål med projektiler affyret fra en slikkanon styret med en Wii-Nunchuck. \newline

\noindent I indledningen af rapporten, blev Goofy Candygun 3000's funktionaliteter beskrevet, og igennem projektet er der blevet implementeret en række disse. Det er lykkedes at implementere en kanon, der kan styres af en Wii-nunchuck. Der er også blevet implementeret en brugergrænseflade, som brugeren kan interagere med i forbindelse med spillet. Brugergrænsefladen kan også starte en systemtest, der tester de interne kommunikationsprotokoller for systemet.\newline

\noindent Fra IHA's side, var der stillet en række krav til, hvad produktet skulle indeholde. Den endelige prototype inkorporerer alle disse krav. I systemet er der implementeret flere forskellige sensorer. Der er en Wii-nunchuck, som opfanger input på Nunchuck controlleren. Der er også implementeret en rotationsdetektor, der bliver brugt i forbindelse med kanonens affyringsmekanisme. Til at begrænse kanonens drejning, er der blevet implementeret en sensor der kan detektere dette. Systemet inkorporerer også flere aktuatorer, i form af kanonens motorer.\newline

\noindent Til projektet er der brugt en indlejret Linux platform i form af Devkit 8000. På denne er der implementeret en brugergrænseflade, hvor brugeren kan starte en systemtest.\newline

\noindent Til aflæsning og dekodning af Nunchucken er der gjort brug af en PSoC platform. \newline

\noindent Systemets funktionaliteter er prioriterede ved brug af en MoSCoW model. Alle funktionaliteterne der er prioriterede som \textit{must haves} er blevet implementeret. Funktionaliteterne under \textit{Should have} er ikke blevet fuldt implementeret, men der er dog opsat en brugergrænseflade der illustrerer mulighed for at se en lokal rangliste statistik, og vælge op til 2 spillere. Funktionaliterne under \textit{Could have} og \textit{Wont have} er ikke implementeret i dette projekt.