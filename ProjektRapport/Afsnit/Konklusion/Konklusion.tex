\chapter{Konklusion}
Formålet med dette projektforløb var at udvikle \textit{Goofy Candy Gun 3000} spillet, hvor 1 til 2 spillere dyster om at ramme et mål med projektiler affyret fra en slikkanon styret med en Wii-Nunchuck. \newline

\noindent Som nævnt i Resultater, afsnit \ref{afsnit:resultater}, endte projektet med en delvis implementering af use case 1 og en fuldtimplementeret use case 2. \newline

\noindent Til use case 1, blev der konstrueret en kanon, som kan styres af en Wii-Nunchuck. For at aflæse data fra Wii-Nunchuck og omsætte data til en bevægelse i motoren, er der blevet implementeret et netværk af PSoC udviklingboards, forbundet via en I2C-bus. Ifølge use case beskrivelsen, skulle der implementeres en brugergrænseflade til visning statistikker og point. Disse funktionaliteter er ikke implementeret, med der er udviklet en demonstrations GUI, der viser opsætningen, dog uden nogen reele funktionaliteter.  
\noindent Derudover skal der foretages nogle kalibreringer for kanonens motorstyring og affyringmekanisme for at use case 1's krav er opfyldt til fulde.\newline 

\noindent For use case 2, er der implementeret en brugergrænseflade, hvor brugeren kan starte en systemtest og følge dennes fremskridt og resultater gennem testen. Brugergrænsefladen gør brug af en SPI-driver til Devkit8000, som mulliggør kommunikationen med PSoC netværket via en SPI-bus. Denne use case er, ifølge den udfyldte accepttest, færdig-implementeret, uden mangler. \newline

\noindent For projektet, blev der foretaget en MOSCOW-analyse (se afsnit \ref{afsnit:Projektafgraensning} - Projektafgrænsning). I denne, blev mulige funktionaliteter prioriteret i en liste af \textit{must have}, \textit{could have}, \textit{should have} og \textit{won't have}. De højeste prioriterede funktionaliteter var:

\begin{itemize}
	\item En motor til styring af kanonen
	\item En grafisk brugergrænseflade
	\item En Wii-nunchuck til styring af motoren
	\item En kanon med affyringsmekanisme
	\item En system test til diagnosering af fejl
\end{itemize}
\noindent Igennem projektet er disse højest prioriterede funktionaliteter implementeret i forbindelse med use case 1 og 2. De resterene funktionaliteter, der har lavere prioritet, er ikke implementeret.