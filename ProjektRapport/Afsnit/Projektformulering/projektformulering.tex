\frontmatter
\maketitle
\newpage

\tableofcontents
\newpage
\listoffigures
\newpage

\mainmatter
\chapter{Projektformulering}
\subsection{Indledning}
I dette projektet skal der udvikles en slikkanon til spillet Goofy Candygun 2016. Denne slikkanon skal kunne skyde med slik. Dette kunne for eksempel være M\&M’s eller Skittle’s.

Goofy Candygun 2016 er et spil til to personer. Spillet går ud på at opnå flest point ved at ramme et mål. Hver spiller får et bestemt antal skud. Efter skuddene er opbrugt er vinderen den spiller med flest point.

Det endelige produkt omfatter:
\begin{itemize}
	\item{En brugergrænseflade, hvor spilstatistikker fremvises til deltagerne. Dette er blandt andet:}
	\subitem{Pointvisning}
	\subitem{Kanonens vinkel}
	\subitem{Antal resterende skud}
	\item{En motor, der drejer kanonen om forskellige akser}
	\subitem{Dette styres med en Wii-nunchuck}
	\item{Et mål, der kan registrere spillernes skud}
\end{itemize}

Et typisk brugerscenarie er at spillerne bestemmer antallet af skud for runden. Når dette er gjort, er spillet igang. Herefter går Wii-nunchucken på skift mellem spillerne for hvert skud. Dette fortsættes indtil skuddene er opbrugt. Vinderen er spilleren med flest point. Spilletsstatistikker vises løbende på brugergrænsefladen. 

\begin{itemize}
	\item{rigt billede + beskrivelse}
\end{itemize}

\subsection{MoSCoW}
\begin{itemize}
	\item{Produktet must have:}
	\subitem{En motor til styring af kanonen}
	\subitem{En brugergrænseflade til visning af statistikker}
	\subitem{En Wii-nunchuck til styring af motoren}
	\item{Produktet should have:}
	\subitem{Et mål til registering af point} 
	\item{Produktet could have:}
	\subitem{Party mode indstilling til over to spillere}
	\subitem{Trådløs Wii-nunchuck styring}
	\subitem{Afspilning af lydeffekter}
	\item{Produktet won't have:}
	\subitem{Et batteri til brug uden strømforsyning}
	\subitem{Online rangliste statistik}	
\end{itemize}


\subsection{Opdeling af gruppen}
\begin{itemize}
	\item{Ansvarsområder}
\end{itemize}