\chapter{Systemarkitektur}

Dette afsnit præsenterer systemets arkitektur i en grad der gør det muligt at forstå sammensætningen mellem dets hardware og software komponenter.

\section{Domænemodel}

På figur \ref{figure:domainModel} ses domænemodellen af systemet. Denne har til formål at præsentere forbindelserne mellem systemets komponenter, samt dets grænseflader.

\begin{figure}[H]
	\centering
	\includegraphics[width=\textwidth]{SystemArkitektur/images/domainModel}
	\caption{Systemets domænemodel}
	\label{figure:domainModel}
\end{figure}

\noindent Her repræsenteres hardware som blokke forbundet med associeringer. Associeringerne viser grænsefladerne mellem de forbundne hardware komponenter (Enten \textit{SPI} eller \textit{I2C}), samt retningen af kommunikationen. Af modellen fremstår konceptuelle kommandoer for grænsefladerne, som beskriver deres nødvændige attributter. \newline

\noindent Domænemodellen er brugt til at udlede grænseflader for systemet, samt potentielle hardware- og softwarekomponenter. Hvad der er udledt af domænemodellen i forhold til grænseflader og komponenter, og hvordan dette bruges omdiskuteres i de følgende arkitektur afsnit.

\section{Hardware}

\subsection{BDD}
\label{afsnit:BDD}
På figur \ref{figure:bddDiagram} ses BDD'et for systemet.

\begin{figure}[H]
	\begin{adjustwidth}{-3cm}{-\rightmargin}
	\centering
	\includegraphics[width=0.9\paperwidth]{SystemArkitektur/images/bddDiagram}
	\caption{BDD af systemets hardware}
	\label{figure:bddDiagram}
	\end{adjustwidth}
\end{figure}

\noindent Her vises alle hardwareblokke fra domænemodellen (figur \ref{figure:domainModel}) med nødvændige indgange og udgange for de fysiske signaler. Yderligere ses det at flow specifikationer er defineret for de ikke-atomare forbindelser \textit{I2C} samt \textit{SPI}, da disse er busser bestående af flere forbindelser. Der henvises til \textbf{IBD AFSNIT} for en detaljeret model af de fysiske forbindelser mellem hardwareblokkende.

\subsection{Blokbeskrivelse}
Følgende afsnit indeholder en blokbeskrivelse samt en flowspecifikation for I2C og SPI. I flowspecifikationen beskrives I2C og SPI forbindelserne mere detaljeret fra en masters synsvinkel.  I blokbeskrivelsen beskrives hver enkel blok, så man har en ide om hvad hver blok består af, hvis de består af flere ting og så skal det give et overblik over hvad hver blok skal bruges til i systemet \newline \newline

\begin{table}[H]
	\centering
	\begin{tabular}{|l|l|}
		\hline
		\textbf{Bloknavn}       & \textbf{Beskrivelse}                                                                                                                                                                                                                           \\ \hline
		Devkit 8000             & \begin{tabular}[c]{@{}l@{}}
		 DevKit 8000 er en embedded Linux platform med touch-skærm, \\ der bruges tilbrugergrænsefladen for produktet. Brugeren inter- \\ agerer med systemet og ser status for spillet via Devkit 8000.                                                      \end{tabular} \\ \hline
		Wii-Nunchuck            & \begin{tabular}[c]{@{}l@{}}Wii-Nunchuck er controlleren som brugeren styrer kanonens \\ retning med. \end{tabular} \\ \hline
		PSoC0                   & \begin{tabular}[c]{@{}l@{}}PSoC0 er PSoC hardware der indeholder software til I2C og \\ SPI kommunikationen og afkodning af Wii-Nunchuck data. \\ PSoC0 fungerer som I2C masterog SPI slave. Denne PSoC er \\ bindeleddet mellem brugergrænsefladen og restenaf systemets \\ hardware. \end{tabular} \\ \hline
		MotorControl                   & \begin{tabular}[c]{@{}l@{}}MotorControl blokken er Candy Gun 3000’s motorerer, der anvendes \\ til at bevæge kanonen. Denne blok består af H-bro blokken og \\ rotationsbegrænsnings blokken. \end{tabular} \\ \hline
		Motor &
		\begin{tabular}[c]{@{}l@{}}Motor er motorene der bruge til at bevæge platform og kanon. \end{tabular} \\ \hline
		H-bro &
		\begin{tabular}[c]{@{}l@{}}H-bro bruges til at styre mototrens rotationsretning \end{tabular} \\ \hline
			Rotationsbegrænsning &
			\begin{tabular}[c]{@{}l@{}}Rotationsbegrænsning er til at begrænse platformens rotation så \\ denne ikke kan dreje 360 grader. Den blok består af et potentiometer \\ og en ADC'en, som sidder internt på PSoC0  \end{tabular} \\ \hline
		PSoC1                   & \begin{tabular}[c]{@{}l@{}}PSoC1 er PSoC hardware der indeholder software til I2C \\ kommunikation og styring af Candy Gun 3000’s motorer. \\ PSoC1 fungerer som I2C slave. \end{tabular} \\ \hline
		SPI (FlowSpecification) &  \begin{tabular}[c]{@{}l@{}}SPI (FlowSpecification) beskriver signalerne der indgår i SPI \\ kommunikation. \end{tabular} \\ \hline
		I2C (FlowSpecification) &  \begin{tabular}[c]{@{}l@{}}I2C (FlowSpecification) beskriver signalerne der indgår i I2C \\ kommunikation. \end{tabular} \\ \hline	
		PSoC2  &  \begin{tabular}[c]{@{}l@{}}PSoC2 denne blok står for at sende et signal til\\ affyringsmekanismen om at den skal skyde og så skal den\\ også holde styre på at der kun bliver skydt en gang. \end{tabular} \\ \hline
		Projectile Launcher  &  \begin{tabular}[c]{@{}l@{}}Projectile Launcher denne blok er vores affyringsmekanismen \end{tabular} \\ \hline
	\end{tabular}
	\label{blokbeskrivelse}
	\caption{Blokbeskrivelse}
\end{table}

\subsection{IBD}
\label{afsnit:IBD}

På figur \ref{figure:ibdDiagram} ses IBD'et for systemet. Figuren viser hardwareblokkene med de fysiske forbindelser beskrevet i BDD'et (figur \ref{figure:bddDiagram}). 

\begin{figure}[H]
	\begin{adjustwidth}{-3cm}{-\rightmargin}
	\centering
	\includegraphics[width=0.9\paperwidth]{SystemArkitektur/images/jgjgj}
	\caption{IBD af systemets hardware}
	\label{figure:ibdDiagram}
	\end{adjustwidth}	
\end{figure}

Her vises alle hardwareblokke med de fysiske forbindelser beskrevet i BDD'et (figur \ref{figure:bddDiagram}). 

Det ses at systemet bliver påvirket af tre eksterne signaler: \textit{touch}, \textit{input}, samt \textit{voltage}. \textit{touch} er input fra brugeren når der interageres med brugergrænsefladen. \textit{input} er brugerens interaktion med Wii-Nunchuk. \textit{voltage} er forsyningsspænding til systemet.

\newpage
\subsection{Signalbeskrivelse}
\begin{longtable}{|>{\hspace{0pt}}p{3cm} | >{\hspace{0pt}}p{3cm} | p{2cm} | p{3cm} |}
	\hline                                                                                                                                                         
	\textbf{Blok-navn} & \textbf{Funktionsbeskrivelse} & \textbf{Signaler} & \textbf{Signalbeskrivelse} \\ \hline
	Devkit 8000 & Fungerer som grænseflade mellem bruger og systemet samt SPI master. & masterSPI & Type: SPI \newline Spændingsniveau: 0-5V \newline Hastighed: 1Mbps \newline Beskrivelse: SPI bussen hvori der sendes og modtages data.\\ \cline{3-4}
	& & touch & Type: touch \newline Beskrivelse: Brugertryk på Devkit 8000 touchdisplay. \\ \hline
	PSoC0 & Fungerer som I2C master for PSoC1 og Wii-Nunchuck samt SPI slave til Devkit 8000. & slaveSPI & Type: SPI \newline Spændingsniveau: 0-5V \newline Hastighed: 1Mbps \newline Beskrivelse: SPI bussen hvori der sendes og modtages data.\\ \cline{3-4}
	& & wiiMaster & Type: I2C \newline Spændingsniveau: 0-5V\newline Hastighed: 100Kpbs \newline Beskrivelse: I2C bussen hvor der modtages data fra Nunchuck.\\ \cline{3-4}
	& & motorMaster & Type: I2C \newline Spændingsniveau: 0-5V \newline Hastighed: 100kbit/sekund \newline Beskrivelse: I2C bussen hvor der sendes afkodet Nunchuck data til PSoC1.\\ \hline
	PSoC1 & Modtager nunchuckinput fra PSoC0 og omsætter dataene til PWM signaler. & motorSlave & Type: I2C \newline Spændingsniveau: 0-5V \newline Hastighed: 100kbit/sekund \newline Beskrivelse: Indeholder formatteret Wii-Nunchuck data som omsættes til PWM-signal. \\ \cline{3-4} 
	&& ShootISR & Type: voltage \newline Spændingsniveau: 0-5V \newline Beskrivelse:  giv et højt signal når den skal skyde.\\ \cline{3-4}   
	& & PWM & Type: PWM \newline Frekvens: 22kHz \newline PWM \%: 0-100\% \newline Spændingsniveau: 0-5V \newline Beskrivelse: PWM signal til styring af motorens hastighed. \\ \cline{3-4}
	& & PotenOut1 & Type: voltage \newline Spændingsniveau: en spænding 0V-5V alt efter hvad potentiometer står på \newline Beskrivelse: den spænding viser hvor motoren står henne\\ \cline{3-4}
	& & PotenOut2 & Type: voltage \newline Spændingsniveau: en spænding 0V-5V alt efter hvad potentiometer står på \newline Beskrivelse: den spænding viser hvor motoren står henne\\ \cline{3-4}
	& & PWM1 & Type: PWM \newline Frekvens: 3MHz \newline PWM\%: 0-100\% \newline Spædingsniveau: 0-5V \newline Beskrivelse: PWM signal til styring af motorens hastighed. \\ \cline{3-4}
	&& PWM2 & Type: PWM \newline Frekvens: 3MHz \newline PWM\%: 0-100\% \newline Spædingsniveau: 0-5V \newline Beskrivelse: PWM signal til styring af motorens hastighed. \\ \cline{3-4}
	
	
	&& PWM3 & Type: PWM \newline Frekvens: 3MHz \newline PWM\%: 0-100\% \newline Spædingsniveau: 0-5V \newline Beskrivelse: PWM signal til styring af motorens hastighed. \\ \cline{3-4}
	&& PWM4 & Type: PWM \newline Frekvens: 3MHz \newline PWM\%: 0-100\% \newline Spædingsniveau: 0-5V \newline Beskrivelse: PWM signal til styring af motorens hastighed. \\ \cline{3-4}
	& & motorVoltage & Type: voltage \newline Spændingsniveau: 9V \newline Beskrivelse: Strømforsyning til motoren\\ \cline{3-4}
	& & potentiometer & Type: voltage \newline Spændingsniveau: 5V \newline Beskrivelse: giver Rotationsbegrænsing 5V \\ \hline
	MotorControl & Den enhed der skal bevæge kanonen & PWM1 & Type: PWM \newline Frekvens: 3MHz \newline PWM\%: 0-100\% \newline Spædingsniveau: 0-5V \newline Beskrivelse: PWM signal til styring af motorens hastighed. \\ \cline{3-4}
	&& PWM2 & Type: PWM \newline Frekvens: 3MHz \newline PWM\%: 0-100\% \newline Spædingsniveau: 0-5V \newline Beskrivelse: PWM signal til styring af motorens hastighed. \\ \cline{3-4}
	
	
	&& PWM3 & Type: PWM \newline Frekvens: 3MHz \newline PWM\%: 0-100\% \newline Spædingsniveau: 0-5V \newline Beskrivelse: PWM signal til styring af motorens hastighed. \\ \cline{3-4}
	&& PWM4 & Type: PWM \newline Frekvens: 3MHz \newline PWM\%: 0-100\% \newline Spædingsniveau: 0-5V \newline Beskrivelse: PWM signal til styring af motorens hastighed. \\ \cline{3-4}
	& & motorVoltage & Type: voltage \newline Spændingsniveau: 9V \newline Beskrivelse: Strømforsyning til motoren\\ \cline{3-4}
	& & potentiometer & Type: voltage \newline Spændingsniveau: 5V \newline Beskrivelse: giver Rotationsbegrænsing 5V 
	\\ \cline{3-4}
	& & PotenOut1 & Type: voltage \newline Spændingsniveau: en spænding 0V-5V alt efter hvad potentiometer står på \newline Beskrivelse: den spænding viser hvor motoren står henne\\ \cline{3-4}
	& & PotenOut2 & Type: voltage \newline Spændingsniveau: en spænding 0V-5V alt efter hvad potentiometer står på \newline Beskrivelse: den spænding viser hvor motoren står henne
	\\ \hline
	Motor & Denne blok beskriver hvad motoren får. & motorvoltage & Type: voltage \newline Spændingsniveau: 0-5V  \newline Beskrivelse: giver spændning til motoren.(denne beskevelse glæder også for den anden motor) \\ \hline
	Wii-nunchuck & Den fysiske controller som brugeren styrer kanonen med. & wiiSlave & Type: I2C \newline Spændingsniveau: 0-5V \newline Hastighed: 100kbit/sekund \newline Beskrivelse: Kommunikationslinje mellem PSoC1 og Wii-Nunchuck. \\ \cline{3-4}
	& & userInput & Type: input \newline Beskrivelse: Brugerinput fra Wii-Nunchuck. \\ \hline
	SPI & Denne blok beskriver den ikke-atomiske SPI forbindelse. & MOSI & Type: CMOS \newline Spændingsniveau: 0-5V \newline Beskrivelse: Binært data der sendes fra master til slave. \\ \cline{3-4}
	& & MISO & Type: CMOS \newline Spændingsniveau: 0-5V \newline Beskrivelse: Binært data der sendes fra slave til master. \\ \cline{3-4}
	& & SCLK & Type: CMOS \newline Spændingsniveau: 0-5V \newline Hastighed: 1Mbps\newline Beskrivelse: Clock signalet fra master til slave, som bruges til at synkronisere den serielle kommunikation. \\ \cline{3-4}
	& & SS & Type: CMOS \newline Spændingsniveau: 0-5V \newline  Beskrivelse: Slave-Select, som bruges til at bestemme hvilken slave der skal kommunikeres med. \\ \hline
	I2C & Denne blok beskriver den ikke-atomiske I2C forbindelse. & SDA & Type: CMOS \newline Spændingsniveau: 0-5V \newline \newline Beskrivelse: Databussen mellem I2C masteren og I2C slaver. \\ \cline{3-4}
	& & SCL & Type: CMOS \newline Spændingsniveau: 0-5V \newline Hastighed: 100kbps \newline Beskrivelse: Clock signalet fra master til lyttende I2C slaver, som bruges til at synkronisere den serielle kommunikation. \\ \hline
	PSoC2& Denne blok bindeledet mellem afyrningsmeknismen og PSoC1. & ShootISR & Type: voltage \newline Spændingsniveau: 0-5V \newline Beskrivelse:  giv et højt signal når den skal skyde.  \\ \hline
	Projectile Launcher & Denne blok er vores afyrningsmeknismen. & PWM5 & Type: PWM \newline Spændingsniveau: 0-5V \newline Beskrivelse: giver signal til mosfet til at åbne så motoren køre en omgang. \\ \hline
	\caption{Tabel med signalbeskrivelse}
\end{longtable}


\section{Software}

\subsection{Software Allokering}
\label{afsnit:SoftwareAllokering}

Domænemodellen i figur \ref{figure:domainModel}, side \pageref{figure:domainModel}, præsenterer systemets hardwareblokke. På figur \ref{figure:allocationDiagram} ses et software allokeringsdiagram, som viser hvilke hardwareblokke der har softwaredele af systemet allokeret på sig. 

\begin{figure}[H]
	\centering
	\includegraphics[width=\textwidth]{SystemArkitektur/images/SoftwareAllocation}
	\caption{Systemets software allokeringer}
	\label{figure:allocationDiagram}
\end{figure}

Det kan her ses at systemet består af tre primære softwaredele: \textit{User Interface Software}, \textit{Nunchuck Polling Software}, \textit{Motor Control Software}. Disse er fordelt over de tre viste CPU'er.

På tabel \ref{tabel:softwareAllocationDescription} er hvert allokeret software komponent beskrevet.

\begin{table}[H]
	\centering
	\begin{tabular}{|ll|}
		\hline
		User Interface Software   & \begin{tabular}[c]{@{}l@{}}Dette allokerede software er brugergrænsefladen\\ som brugeren interagerer med på DevKit8000 touch-skærmen.\end{tabular}                                                  \\
		\rowcolor[HTML]{CBCEFB} 
		Nunchuck Polling Software & \begin{tabular}[c]{@{}l@{}}Dette allokerede software har til ansvar at polle\\ Nunchuck tilstanden og videresende det til\\ PSoC1.\end{tabular}                                                      \\
		Motor Control Software    & \cellcolor[HTML]{FFFFFF}\begin{tabular}[c]{@{}l@{}}Dette allokerede software har til ansvar at\\ bruge den pollede Nunchuck data fra PSoC0\\ til motorstyring samt affyringsmekanismen.\end{tabular} 
		\\ 
		\rowcolor[HTML]{CBCEFB} 
		Projectile Launcher Software & \begin{tabular}[c]{@{}l@{}}Dette allokerede software har til ansvar at aktivere\\ affyringsmekanismen når et \\ knaptryk detekteres på Nunchuck.\end{tabular} \\
		\hline
	\end{tabular}
	\caption{Beskrivelse af den allokerede software}
	\label{tabel:softwareAllocationDescription}
\end{table}

\subsection{Informationsflow i systemet}
\label{afsnit:informationFlow}
Dette afsnit har til formål at demonstrere sammenhængen mellem softwaren på CPU'erne og resten af systemet, samt at beskrive og identificere grænsefladerne brugt til kommunikation mellem dem. Yderligere vil klasseidentifikation også blive vist, hvor disse klasser vil specificeres i Design og Implementering.

\subsubsection{Wii-Nunchuck Information Flow}
En essentiel del af systemet er at kunne styre motoren ved brug af Wii-Nunchuck controlleren. På figur \ref{fig:WiiNunchuckSekvensDiagram} vises gennemløbet af Wii-Nunchuck input data fra Wii-Nunchuken til motoren, med de relevante CPU'er angivet. Her ses det at input data fra Wii-Nunchuck kontinuert bliver aflæst af PSoC0. Det bemærkes her at grænsefladen mellem PSoC0 og Wii-Nunchuck er en I2C bus. Efter at PSoC0 har aflæst input data'en, overføres den til PSoC1. Grænsefladen mellem disse to PSoCs er også en I2C bus. PSoC1 kan til slut oversætte modtaget input data til PWM signaler til motorstyring samt affyring.

\begin{figure}[H]
	\centering
	\includegraphics[width=\textwidth] {Systemarkitektur/images/SequenceDiagramUC1}
	\caption{Wii-Nunchuck Input Data Forløb}
	\label{fig:WiiNunchuckSekvensDiagram}
\end{figure}

\subsubsection{System Test}
Use Case 2 beskriver test af systemet før spillet startes. På figur \ref{fig:SystemTestSekvensDiagram} vises test sekvensen for en vellykket test. 

\begin{figure}[H]
	\centering
	\includegraphics[width=\textwidth] {Systemarkitektur/images/SequenceDiagramUC2}
	\caption{System Test Forløb}
	\label{fig:SystemTestSekvensDiagram}
\end{figure}

Kommunikationen mellem Devkit8000 og PSoC0 er initieret, og styret af, Devkittet. Da PSoC0 i dette design ikke har mulighed for at indikere, hvornår data er klar til aflæsning, skal Devkittet vente på en \textit{timeout}, før den aflæser data på PSoC0. Dette er muligt, fordi at system testen bliver afviklet sekventielt.

\subsection{Samlede Klassediagrammer}
På baggrund af sekvensdiagrammerne i afsnit \ref{afsnit:informationFlow} samt de detaljerede applikationsmodeller analyseret i DOKUMENTATION \textbf{\#ref}, er der udledt samlede klassediagrammer for hver individuel CPU i systemet. Disse er med til at identificere konceptuelle klasser og funktionaliter der skal overvejes i implementation og design af systemet, og har altså fungeret som et udgangspunkt til resten af udviklingsprocessen.

Figur \ref{fig:CompleteClassDiagramDevKit8000}, \ref{fig:CompleteClassDiagramPSoC0}, \ref{fig:CompleteClassDiagramPSoC1}, og \ref{fig:CompleteClassDiagramPSoC2} vises de samlede klassediagrammer for hver CPU.

For disse klassediagrammer er der konceptuelle klasser som gentager sig selv. Alle klassediagrammer har én klasse af typen \textit{controller}. Disse klasse indeholder alt funktionalitet der er nødvændig for at kunne implementere systemets Use Cases. Yderligere bliver klasserne \textit{I2C Interface} samt \textit{SPI Interface} gentaget. Disse repræsenterer klasser for de tilsvarende bustyper, I2C og SPI, og bruges af softwaren til at sende og modtage data på disse busser.

På figur \ref{fig:CompleteClassDiagramDevKit8000} ses det at DevKit8000 controlleren skal have funktionalitet til start af system test, i relation til Use Case 2. For at kunne udføre dette kommunikerer den med grænsefladen Graphical User Interface (\textit{GUI}), for at kunne vise resultater til brugeren. Desuden skal controlleren kommunikere med grænsefladen \textit{SPI Interface}, for at sende data ud til til resten af systemet via SPI bussen.
\begin{figure}[H]
	\centering
	\includegraphics[width=\textwidth] {Systemarkitektur/images/CompleteClassDiagramDevKit8000}
	\caption{Samlet Klassediagram for DevKit8000}
	\label{fig:CompleteClassDiagramDevKit8000}
\end{figure}

På figur \ref{fig:CompleteClassDiagramPSoC0} ses det at PSoC0 controlleren skal have funktionalitet til at starte tests af systemets busser. Yderligere skal den også kunne dekode og kalibrere data der kommer fra Nunchuck. I relation til disse funktionaliteter skal den kunne modtage og sende data på systemets I2C og SPI busser.
\begin{figure}[H]
	\centering
	\includegraphics[width=\textwidth] {Systemarkitektur/images/CompleteClassDiagramPSoC0}
	\caption{Samlet Klassediagram for PSoC0}
	\label{fig:CompleteClassDiagramPSoC0}
\end{figure}

På figur \ref{fig:CompleteClassDiagramPSoC1} ses det at PSoC1 controlleren kommunikerer med grænsefladerne \textit{I2C Interface}, \textit{Motor Control} samt \textit{PSoC2}. Controlleren skal kunne regulere fart på systemets motorstyring, for at kunne styre kanonen. Desuden skal den kunne sende en affyringsbesked til \textit{PSoC2}. For at regulere fart og affyre kanonen, skal controlleren kunne aflæse Nunchucken's tilstand fra \textit{I2C Interface}.
\begin{figure}[H]
	\centering
	\includegraphics[width=\textwidth] {Systemarkitektur/images/CompleteClassDiagramPSoC1}
	\caption{Samlet Klassediagram for PSoC1}
	\label{fig:CompleteClassDiagramPSoC1}
\end{figure}

På figur \ref{fig:CompleteClassDiagramPSoC2} ses det at PSoC2 skal have funktionalitet til at aktivere afskydning, som gøres ved at kommunikere med grænsefladen \textit{Projectile Launcher}. Afskydningen aktiveres af grænsefladen \textit{PSoC1}.
\begin{figure}[H]
	\centering
	\includegraphics[width=\textwidth] {Systemarkitektur/images/CompleteClassDiagramPSoC2}
	\caption{Samlet Klassediagram for PSoC2}
	\label{fig:CompleteClassDiagramPSoC2}
\end{figure}

\subsection{SPI Kommunikations Protokol}
I afsnit \ref{afsnit:IBD} \textbf{IBD} ses på figur \ref{figure:ibdDiagram} at Devkit8000 og PSoC0 kommunikerer via en SPI bus. Kommunikationen foregår ved at der sendes kommandotyper imellem de to enheder, SPI-master og SPI-slave. På tabel \ref{tabel:spiKommandoType} ses en de anvendte kommandotyper samt en kort beskrivelse for hver af disse.

Der er til systemet valgt en SPI bus mellem DevKit8000 og PSoC0, da der i dette tilfælde kun er brug for en bus med god support for én master og én slave. En SPI Bus skalerer ikke godt med multiple slaver i forhold til SPI, da en SPI bus skal have en fysisk forbindelse for hvert enkel slave der tilkobles. Bussen er dog simpel at implementere, og hver derfor ideel for denne grænseflade.

\begin{table}[H]
	\centering
	\resizebox{\textwidth}{!}{%
		\begin{tabular}{llll}
			\hline
			\multicolumn{1}{|l|}{Kommandotype}                                & \multicolumn{1}{l|}{Beskrivelse}                        & \multicolumn{1}{l|}{Binær Værdi} & \multicolumn{1}{l|}{Hex Værdi} \\ \hline
			\rowcolor[HTML]{CBCEFB} 
			{\color[HTML]{000000} START\_SPI\_TEST}                           & {\color[HTML]{000000} Sætter PSoC0 i 'SPI-TEST' mode}   & 1111 0001                        & 0xF1                           \\
			START\_I2C\_TEST                                                  & Sætter PSoC0 i 'I2C-TEST' mode                          & 1111 0010                        & 0xF2                           \\
			\rowcolor[HTML]{CBCEFB} 
			\begin{tabular}[c]{@{}l@{}}START\_NUN-\\ CHUCK\_TEST\end{tabular} & Sætter PSoC0 i 'NUNCHUCK-TEST' mode                     & 1111 0011                        & 0xF3                           \\
			SPI\_OK                                                           & Signalerer at SPI-testen blev gennemført uden fejl      & 1101 0001                        & 0xD1                           \\
			\rowcolor[HTML]{CBCEFB} 
			I2C\_OK                                                           & Signalerer at I2C-testen blev gennemført uden fejl      & 1101 0010                        & 0xD2                           \\
			I2C\_FAIL                                                         & Signalerer at I2C-testen fejlede           & 1100 0010                        & 0xC2                           \\
			\rowcolor[HTML]{CBCEFB} 
			NUNCHUCK\_OK                                                      & Signalerer at NUNCHUCK-testen blev gennemført uden fejl & 1101 0011                        & 0xD3                           \\
			NUNCHUCK\_FAIL                                                    & Signalerer at NUNCHUCK-testen fejlede      & 1100 0011                        & 0xC3                          
		\end{tabular}
	}
	\caption{SPI kommunikation kommandotyper}
	\label{tabel:spiKommandoType}
\end{table}

Kommandotyperne er udledt fra det samlede klassediagram for DevKit8000, figur \ref{fig:CompleteClassDiagramDevKit8000}. På figuren kan det ses at \textit{DevKit8000 Controller} skal kunne starte tests for SPI, I2C, samt Nunchuck og aflæse resultatet af disse.  

Kommunikation på en SPI-bus foregår ved bit-shifting. Dette betyder at indholdet af masterens transfer buffer bliver skiftet over i slavens read buffer og omvendt. Kommunikationen foregår i fuld duplex, derfor skal der foretages to transmissioner for at aflæse data fra en SPI-slave. Dette skyldes at slaven skal vide hvilken data der skal klargøres i transfer-bufferen og klargøre denne buffer før masteren kan aflæse denne. Her skal der tages højde for at en længere proces skal gennemføres før slaven har klargjort transfer-buffern. Derfor skal masteren, efter at have sendt en kommandotype, vente et bestemt stykke tid før der at aflæses fra slavens transfer-buffer. Denne sekvens er illustreret med et sekvensdiagram på figur \ref{figure:SDSpiSlaveRead}.

\begin{figure}[H]
	\centering
	\includegraphics[]{Systemarkitektur/images/SDSpiSlaveRead}
	\caption{Sekvensdiagram for aflæsning data fra en SPI-slave}
	\label{figure:SDSpiSlaveRead}
\end{figure} 

\subsubsection{Designvalg}
SPI kommunikations protokollen er designet ud fra det grundlag at DevKit8000, som SPI master, skal læse tilstande fra SPI slaven ved brug af en timeout model. Ved timeout model menes der at DevKit8000 sender en kommandotype ud, venter et bestemt antal sekunder, og herefter læser værdien fra SPI slaven. Denne model er modelleret på figur \ref{figure:SDSpiSlaveRead}

Et alternativt design ville være at generere et interrupt til SPI masteren når slaven havde data der skulle læses. Til dette system blev timeout modellen dog valgt, da det hardwaremæssigt var simplere at implementere, samt at alt nødvændig funktionalitet kan implementeres med den valgte model.

\subsection{I2C Kommunikations Protokol}
\label{afsnit:I2CProtokol}
I afsnit \ref{afsnit:IBD} \textbf{IBD} ses på figur \ref{figure:ibdDiagram} at tre hardwareblokke kommunikerer via en I2C bus. Til denne I2C kommunikation er der defineret en protokol, som bestemmer hvordan modtaget data skal fortolkes. Denne protokol beskrives følgende.

I2C bussen er brugt til disse forbindelser af to primære grunde. Først og fremmest er Nunchuck controlleren implementeret som en I2C Slave fra producentens side. Derfor skulle der gøres brug af en I2C bus i alle tilfælde. Yderligere skulle Nunchuken's input sendes videre til PSoC1, og I2C er en bus der skalerer godt med multiple enheder. Derfor var det naturligt at udvide I2C bussen mellem Nunchuck og PSoC0 med PSoC1.

I2C gør brug af en indbygget protokol, der anvender adressering af hardware-enheder til identificering af hvilken enhed der kommunikeres med. Derfor har hardwareblokkene som indgår i I2C kommunikationen fået tildelt adresser. På tabel \ref{table:I2CAddress} ses adresserne tildelt systemets PSoCs.

\begin{table}[H]
	\centering
	\begin{tabular}{llllllllll}
		\hline
		\multicolumn{1}{|l|}{I2C Adresse bits} & 7                        & 6                        & 5                        & 4 & 3 & 2 & \multicolumn{1}{l|}{1} & \multicolumn{1}{l|}{0 (R/W)} \\ \hline
		\rowcolor[HTML]{CBCEFB} 
		{\color[HTML]{000000} PSoC0}           & {\color[HTML]{000000} 0} & {\color[HTML]{000000} 0} & {\color[HTML]{000000} 0} & 1 & 0 & 0 & 0                      & 0/1                          \\
		PSoC1                                  & 0                        & 0                        & 0                        & 1 & 0 & 0 & 1                      & 0/1 \\
		\rowcolor[HTML]{CBCEFB} 
		{\color[HTML]{000000} Wii-Nunchuck}           & {\color[HTML]{000000} 1} & {\color[HTML]{000000} 0} & {\color[HTML]{000000} 1} & 0 & 0 & 1 & 0                      & 0/1                      
	\end{tabular}
	\caption{I2C bus addresser}
	\label{table:I2CAddress}
\end{table}

Da I2C dataudveksling sker bytevist, er kommunikations protokollen opbygget ved, at kommandoens type indikeres af den første modtagne byte. Herefter følger \textit{N}-antal bytes som er kommandoens tilhørende data. \textit{N} er et vilkårligt heltal og bruges i dette afsnit når der refereres til en mængde data-bytes der sendes med en kommandotype.

På tabel \ref{table:I2CKommandoer} ses de definerede kommandotyper og det tilsvarende antal af bytes der sendes ved dataveksling.

\begin{table}[H]
	\begin{adjustwidth}{-3cm}{-\rightmargin}
		\centering
		\resizebox{1.5\textwidth}{!}{
			\begin{tabular}{lllll}
				\hline
				\multicolumn{1}{|l|}{Kommandotype}    & \multicolumn{1}{l|}{Beskrivelse}                                            & \multicolumn{1}{l|}{Binær Værdi} & \multicolumn{1}{l|}{Hex Værdi} & \multicolumn{1}{l|}{Data Bytes}                                                                                         \\ \hline
				\rowcolor[HTML]{CBCEFB} 
				{\color[HTML]{000000} NunchchuckData} & {\color[HTML]{000000} Indeholder aflæst data fra Wii Nunchuck controlleren} & 0010 1010                        & 0xA2                           & \begin{tabular}[c]{@{}l@{}}Byte \#1 Analog X-værdi\\ Byte \#2 Analog Y-værdi\\ Byte \#3 Analog Buttonstate\end{tabular} \\                                                                               
			\end{tabular}
		}
		\caption{I2C kommunikation kommandotyper}
		\label{table:I2CKommandoer}
	\end{adjustwidth}
\end{table}

Kommandotyperne er primært udledt fra det samlede klassediagram for PSoC1, figur \ref{fig:CompleteClassDiagramPSoC1}. På figuren kan det ses at \textit{PSoC1 Controller} skal kunne aflæse input tilstand fra I2C Interface. På det samlede klassediagram for PSoC0, figur \ref{fig:CompleteClassDiagramPSoC0}, kan det også ses at PSoC0 Controller skal kunne udføre tilsvarende handling på I2C Interface. Denne funktionalitet er overført til kommandotypen \textit{NunchuckData}.

Kolonnerne \textit{Binær Værdi} og \textit{Hex Værdi} i tabel \ref{table:I2CKommandoer} viser kommandotypens unikke tal-ID i både binær- og hexadecimalform. Denne værdi sendes som den første byte, for at identificere kommandotypen.

Kommandoens type definerer antallet af databytes modtageren skal forvente og hvordan disse skal fortolkes. På figur \ref{fig:I2CProtokolEksempel} ses et sekvensdiagram der, med pseudo-kommandoer, demonstrerer forløbet mellem en I2C afsender og modtager ved brug af kommunikations protokollen.

\begin{figure}[H]
	\centering
	\includegraphics[width=\textwidth] {Systemarkitektur/images/I2CProtocol}
	\caption{Eksempel af I2C Protokol Forløb}
	\label{fig:I2CProtokolEksempel}
\end{figure}

På figur \ref{fig:I2CProtokolEksempel} ses at afsenderen starter en I2C transaktion, hvorefter typen af kommando sendes som den første byte. Efterfølgende sendes \textit{N} antal bytes, afhængig af hvor meget data den givne kommandotype har brug for at sende. Efter en afsluttet I2C transaktion læser I2C modtageren typen af kommando, hvor den herefter tolker \textit{N} antal modtagne bytes afhængig af den modtagne kommandotype.

\subsubsection{Designvalg}
Den primære idé bag valg af kommandotype metoden til I2C kommunikations protokollen er først og fremmest så modtageren kan differentiere mellem flere handlinger i systemet. En anden vigtig grund er at man, ved kommandotyper, kan associere et dynamisk antal bytes til hver kommando. Dvs, hvis en modtager får en kommandotype \textit{A}, vil den vide at den efterfølgende skal læse 5 bytes, hvormiod at en kommandotype \textit{B} ville betyde at der kun skulle læses 2 bytes.

En anden fordel ved kommandotype metoden er, at den er relativ simpel at implementere i kode, som vist ved pseudofunktionerne i figur \ref{fig:I2CProtokolEksempel}.


\subsection{Interface driver}




\subsection{Brugergrænseflade}
Ved hjælp af en brugergrænseflade, kan systemtesten styres fra DevKit8000.
Brugergrænsefladen er opbygget ud fra sekvensdiagrammet \ref{fig:SystemTestSekvensDiagram},
ud fra dette kunne brugergrænsefladen skitseres som figur \ref{fig:GUISkitse}.
Grænsefladen mellem DevKit8000 og PSoC0 er en SPI-bus som ses på figur \ref{afsnit:IBD}, og brugergrænsefladen er koblet til DevKit8000 ved hjælp af interfacedriveren.
Brugergrænsefladen skal sende startsekvensen til interfacedriveren, hvorefter dette sendes til PSoC0 og videre ud i systemet.
Brugergrænsefladen skal aflæse svaret fra systemtesten og printe dette ud i en konsol.

\begin{figure}[H]
	\centering
	\includegraphics[width=\textwidth] {Systemarkitektur/images/GUISkitse}
	\caption{Skitse af brugergrænseflade}
	\label{fig:GUISkitse}
\end{figure}

\subsubsection{Designvalg}
Brugergrænsefladen er designet ud fra den sekventielle struktur i usecase 2 \#Ref Dokumentation. Dette er løst ved hjælp af Event-Driven Programming.
Denne model drives ved hjælp af events, som i dette tilfælde aktiveres af brugeren ved knaptryk. Knapperne vil blive tildelt forskellige funktionaliteter, der faciliterer systemtesten.
Et alternativ kunne være trådbaseret design. Kompleksiteten i dette design ville være overvældende i forhold til den ønskede funktionalitet og blev derfor fravalgt.