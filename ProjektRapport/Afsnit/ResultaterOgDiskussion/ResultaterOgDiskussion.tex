%\chapter{Resultater og Diskussion}

\chapter{Resultater}
\label{afsnit:resultater}

Undervejs i projektet er der lavet modultests af de forskellige dele af produktet. Disse er alle gennemført med positivt resultat. Der er også foretaget integrationstest, som også blev godkendt. 

\noindent I slutningen af produktudviklingen blev der afviklet en accepttest i samarbejde med gruppens vejleder Gunvor Elisabeth Kirkelund. Den udfyldte accepttest kan ses i dokumentationen, afsnit 6 -\textit{Udfyldt Accepttest} side 124.

\noindent I tabel \ref{table:UseCaseResults} ses en opsummering af resultaterne fra produktets udførte accepttest. 

\begin{table}[H]
	\centering
\begin{tabular}{|l|l|}
	\hline
	Use Cases                                   & Resultat                                                                           \\ \hline
	Use Case 1 - Spil Goofy Candygun 3000       & \begin{tabular}[c]{@{}l@{}}Denne use case er\\ delvist implementeret.\end{tabular} \\ \hline
	Use Case 2 - Test kommunikationsprotokoller & \begin{tabular}[c]{@{}l@{}}Denne use case er\\ implementeret.\end{tabular}         \\ \hline
\end{tabular}
	\caption{Opnåede resultater for systemets use cases}
	\label{table:UseCaseResults}
\end{table}

\noindent Det ses, at hele use case 2 er fuldt implementeret og use case 1 er delvist implementeret. I use case 1 er det muligt at indstille kanonen med Wii-nunchuck og at udløse denne, så den affyrer et stykke slik. Der er også udviklet en illustrativ brugergrænseflade. Denne er kun en demo, da kommunikationsprotokollerne ikke er implementeret. 


%\noindent Jf. MoSCoW-analysen i afsnit \ref{afsnit:Projektafgraensning} skal det færdige produkt have en motor til styring af kanonen, en Wii-nunchuck til styring af motoren, en kanon med en affyringsmekanisme, en systemtest til diagnosticering af fejl og en grafisk brugergrænseflade. Det eneste, der ikke er implementeret, er den grafiske brugergrænseflade, som dog er implementeret med en demo. Dette gør, at der ikke er langt igen til at use case 1 er fuldt implementeret. 

%Denne er kun implementeret som en demo. Der er ingen bagomliggende funktionalitet til at vise spillets statistikker eller administrering af player-modes. Der er en slikkanon, som kan styres med en Nunchuck, og affyringsmekanismen kan aktiveres. Dog mangles kalibrering af motor og affyringsmekanisme, da motoren ikke kan bære kanonens vægt, og slik ikke kan affyres med tilstrækkelig kraft. \newline

\noindent Projektforløbet resulterede i en prototype hvor use case 1 og use case 2 er implementeret som beskrevet.

