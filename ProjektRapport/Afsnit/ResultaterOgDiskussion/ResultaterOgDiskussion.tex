%\chapter{Resultater og Diskussion}

\chapter{Resultater}
\label{afsnit:resultater}

I slutningen af produktudviklingen blev der afviklet en accepttest i samarbejde med gruppens vejleder Gunvor Elisabeth Kirkelund. Denne udfyldte accepttest kan ses i dokumentationen, afsnit 6-\textit{Udfyldt Accepttest} side 124.

\noindent På tabel \ref{table:UseCaseResults} ses en opsummering af resultaterne fra produktets udførte accepttests. 

\begin{table}[H]
	\centering
\begin{tabular}{|l|l|}
	\hline
	Use Cases                                   & Resultat                                                                           \\ \hline
	Use Case 1 - Spil Goofy Candygun 3000       & \begin{tabular}[c]{@{}l@{}}Denne use case er\\ delvist implementeret.\end{tabular} \\ \hline
	Use Case 2 - Test kommunikationsprotokoller & \begin{tabular}[c]{@{}l@{}}Denne use case er\\ implementeret.\end{tabular}         \\ \hline
\end{tabular}
	\caption{Opnåede resultater for systemets use cases}
	\label{table:UseCaseResults}
\end{table}

\noindent Use case 1 er delvist implementeret, da brugergrænsefladen er implementeret som en demo. Der er ingen bagomliggende funktionalitet til at vise spillets statistikker, samt administrering af player-modes. Der er en slik kanon som kan styres med en Nunchuck, og affyringsmekanismen kan aktiveres. Dog mangles kalibrering af motor og affyringsmekanisme, da motoren ikke kan bære kanonens vægt, og slik ikke kan affyres med tilstrækkelig kraft. \newline

\noindent Projektforløbet resulterede i en prototype hvor use case 1 og use case 2 er implementeret som beskrevet.

